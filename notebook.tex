
% Default to the notebook output style

    


% Inherit from the specified cell style.




    
\documentclass[11pt]{article}

    
    
    \usepackage[T1]{fontenc}
    % Nicer default font (+ math font) than Computer Modern for most use cases
    \usepackage{mathpazo}

    % Basic figure setup, for now with no caption control since it's done
    % automatically by Pandoc (which extracts ![](path) syntax from Markdown).
    \usepackage{graphicx}
    % We will generate all images so they have a width \maxwidth. This means
    % that they will get their normal width if they fit onto the page, but
    % are scaled down if they would overflow the margins.
    \makeatletter
    \def\maxwidth{\ifdim\Gin@nat@width>\linewidth\linewidth
    \else\Gin@nat@width\fi}
    \makeatother
    \let\Oldincludegraphics\includegraphics
    % Set max figure width to be 80% of text width, for now hardcoded.
    \renewcommand{\includegraphics}[1]{\Oldincludegraphics[width=.8\maxwidth]{#1}}
    % Ensure that by default, figures have no caption (until we provide a
    % proper Figure object with a Caption API and a way to capture that
    % in the conversion process - todo).
    \usepackage{caption}
    \DeclareCaptionLabelFormat{nolabel}{}
    \captionsetup{labelformat=nolabel}

    \usepackage{adjustbox} % Used to constrain images to a maximum size 
    \usepackage{xcolor} % Allow colors to be defined
    \usepackage{enumerate} % Needed for markdown enumerations to work
    \usepackage{geometry} % Used to adjust the document margins
    \usepackage{amsmath} % Equations
    \usepackage{amssymb} % Equations
    \usepackage{textcomp} % defines textquotesingle
    % Hack from http://tex.stackexchange.com/a/47451/13684:
    \AtBeginDocument{%
        \def\PYZsq{\textquotesingle}% Upright quotes in Pygmentized code
    }
    \usepackage{upquote} % Upright quotes for verbatim code
    \usepackage{eurosym} % defines \euro
    \usepackage[mathletters]{ucs} % Extended unicode (utf-8) support
    \usepackage[utf8x]{inputenc} % Allow utf-8 characters in the tex document
    \usepackage{fancyvrb} % verbatim replacement that allows latex
    \usepackage{grffile} % extends the file name processing of package graphics 
                         % to support a larger range 
    % The hyperref package gives us a pdf with properly built
    % internal navigation ('pdf bookmarks' for the table of contents,
    % internal cross-reference links, web links for URLs, etc.)
    \usepackage{hyperref}
    \usepackage{longtable} % longtable support required by pandoc >1.10
    \usepackage{booktabs}  % table support for pandoc > 1.12.2
    \usepackage[inline]{enumitem} % IRkernel/repr support (it uses the enumerate* environment)
    \usepackage[normalem]{ulem} % ulem is needed to support strikethroughs (\sout)
                                % normalem makes italics be italics, not underlines
    

    
    
    % Colors for the hyperref package
    \definecolor{urlcolor}{rgb}{0,.145,.698}
    \definecolor{linkcolor}{rgb}{.71,0.21,0.01}
    \definecolor{citecolor}{rgb}{.12,.54,.11}

    % ANSI colors
    \definecolor{ansi-black}{HTML}{3E424D}
    \definecolor{ansi-black-intense}{HTML}{282C36}
    \definecolor{ansi-red}{HTML}{E75C58}
    \definecolor{ansi-red-intense}{HTML}{B22B31}
    \definecolor{ansi-green}{HTML}{00A250}
    \definecolor{ansi-green-intense}{HTML}{007427}
    \definecolor{ansi-yellow}{HTML}{DDB62B}
    \definecolor{ansi-yellow-intense}{HTML}{B27D12}
    \definecolor{ansi-blue}{HTML}{208FFB}
    \definecolor{ansi-blue-intense}{HTML}{0065CA}
    \definecolor{ansi-magenta}{HTML}{D160C4}
    \definecolor{ansi-magenta-intense}{HTML}{A03196}
    \definecolor{ansi-cyan}{HTML}{60C6C8}
    \definecolor{ansi-cyan-intense}{HTML}{258F8F}
    \definecolor{ansi-white}{HTML}{C5C1B4}
    \definecolor{ansi-white-intense}{HTML}{A1A6B2}

    % commands and environments needed by pandoc snippets
    % extracted from the output of `pandoc -s`
    \providecommand{\tightlist}{%
      \setlength{\itemsep}{0pt}\setlength{\parskip}{0pt}}
    \DefineVerbatimEnvironment{Highlighting}{Verbatim}{commandchars=\\\{\}}
    % Add ',fontsize=\small' for more characters per line
    \newenvironment{Shaded}{}{}
    \newcommand{\KeywordTok}[1]{\textcolor[rgb]{0.00,0.44,0.13}{\textbf{{#1}}}}
    \newcommand{\DataTypeTok}[1]{\textcolor[rgb]{0.56,0.13,0.00}{{#1}}}
    \newcommand{\DecValTok}[1]{\textcolor[rgb]{0.25,0.63,0.44}{{#1}}}
    \newcommand{\BaseNTok}[1]{\textcolor[rgb]{0.25,0.63,0.44}{{#1}}}
    \newcommand{\FloatTok}[1]{\textcolor[rgb]{0.25,0.63,0.44}{{#1}}}
    \newcommand{\CharTok}[1]{\textcolor[rgb]{0.25,0.44,0.63}{{#1}}}
    \newcommand{\StringTok}[1]{\textcolor[rgb]{0.25,0.44,0.63}{{#1}}}
    \newcommand{\CommentTok}[1]{\textcolor[rgb]{0.38,0.63,0.69}{\textit{{#1}}}}
    \newcommand{\OtherTok}[1]{\textcolor[rgb]{0.00,0.44,0.13}{{#1}}}
    \newcommand{\AlertTok}[1]{\textcolor[rgb]{1.00,0.00,0.00}{\textbf{{#1}}}}
    \newcommand{\FunctionTok}[1]{\textcolor[rgb]{0.02,0.16,0.49}{{#1}}}
    \newcommand{\RegionMarkerTok}[1]{{#1}}
    \newcommand{\ErrorTok}[1]{\textcolor[rgb]{1.00,0.00,0.00}{\textbf{{#1}}}}
    \newcommand{\NormalTok}[1]{{#1}}
    
    % Additional commands for more recent versions of Pandoc
    \newcommand{\ConstantTok}[1]{\textcolor[rgb]{0.53,0.00,0.00}{{#1}}}
    \newcommand{\SpecialCharTok}[1]{\textcolor[rgb]{0.25,0.44,0.63}{{#1}}}
    \newcommand{\VerbatimStringTok}[1]{\textcolor[rgb]{0.25,0.44,0.63}{{#1}}}
    \newcommand{\SpecialStringTok}[1]{\textcolor[rgb]{0.73,0.40,0.53}{{#1}}}
    \newcommand{\ImportTok}[1]{{#1}}
    \newcommand{\DocumentationTok}[1]{\textcolor[rgb]{0.73,0.13,0.13}{\textit{{#1}}}}
    \newcommand{\AnnotationTok}[1]{\textcolor[rgb]{0.38,0.63,0.69}{\textbf{\textit{{#1}}}}}
    \newcommand{\CommentVarTok}[1]{\textcolor[rgb]{0.38,0.63,0.69}{\textbf{\textit{{#1}}}}}
    \newcommand{\VariableTok}[1]{\textcolor[rgb]{0.10,0.09,0.49}{{#1}}}
    \newcommand{\ControlFlowTok}[1]{\textcolor[rgb]{0.00,0.44,0.13}{\textbf{{#1}}}}
    \newcommand{\OperatorTok}[1]{\textcolor[rgb]{0.40,0.40,0.40}{{#1}}}
    \newcommand{\BuiltInTok}[1]{{#1}}
    \newcommand{\ExtensionTok}[1]{{#1}}
    \newcommand{\PreprocessorTok}[1]{\textcolor[rgb]{0.74,0.48,0.00}{{#1}}}
    \newcommand{\AttributeTok}[1]{\textcolor[rgb]{0.49,0.56,0.16}{{#1}}}
    \newcommand{\InformationTok}[1]{\textcolor[rgb]{0.38,0.63,0.69}{\textbf{\textit{{#1}}}}}
    \newcommand{\WarningTok}[1]{\textcolor[rgb]{0.38,0.63,0.69}{\textbf{\textit{{#1}}}}}
    
    
    % Define a nice break command that doesn't care if a line doesn't already
    % exist.
    \def\br{\hspace*{\fill} \\* }
    % Math Jax compatability definitions
    \def\TeX{\mbox{T\kern-.14em\lower.5ex\hbox{E}\kern-.115em X}}
    \def\LaTeX{\mbox{L\kern-.325em\raise.21em\hbox{$\scriptstyle{A}$}\kern-.17em}\TeX}
    \def\gt{>}
    \def\lt{<}
    % Document parameters
    \title{intro-ml-stepik}
    
    
    

    % Pygments definitions
    
\makeatletter
\def\PY@reset{\let\PY@it=\relax \let\PY@bf=\relax%
    \let\PY@ul=\relax \let\PY@tc=\relax%
    \let\PY@bc=\relax \let\PY@ff=\relax}
\def\PY@tok#1{\csname PY@tok@#1\endcsname}
\def\PY@toks#1+{\ifx\relax#1\empty\else%
    \PY@tok{#1}\expandafter\PY@toks\fi}
\def\PY@do#1{\PY@bc{\PY@tc{\PY@ul{%
    \PY@it{\PY@bf{\PY@ff{#1}}}}}}}
\def\PY#1#2{\PY@reset\PY@toks#1+\relax+\PY@do{#2}}

\expandafter\def\csname PY@tok@gd\endcsname{\def\PY@tc##1{\textcolor[rgb]{0.63,0.00,0.00}{##1}}}
\expandafter\def\csname PY@tok@gu\endcsname{\let\PY@bf=\textbf\def\PY@tc##1{\textcolor[rgb]{0.50,0.00,0.50}{##1}}}
\expandafter\def\csname PY@tok@gt\endcsname{\def\PY@tc##1{\textcolor[rgb]{0.00,0.27,0.87}{##1}}}
\expandafter\def\csname PY@tok@gs\endcsname{\let\PY@bf=\textbf}
\expandafter\def\csname PY@tok@gr\endcsname{\def\PY@tc##1{\textcolor[rgb]{1.00,0.00,0.00}{##1}}}
\expandafter\def\csname PY@tok@cm\endcsname{\let\PY@it=\textit\def\PY@tc##1{\textcolor[rgb]{0.25,0.50,0.50}{##1}}}
\expandafter\def\csname PY@tok@vg\endcsname{\def\PY@tc##1{\textcolor[rgb]{0.10,0.09,0.49}{##1}}}
\expandafter\def\csname PY@tok@vi\endcsname{\def\PY@tc##1{\textcolor[rgb]{0.10,0.09,0.49}{##1}}}
\expandafter\def\csname PY@tok@vm\endcsname{\def\PY@tc##1{\textcolor[rgb]{0.10,0.09,0.49}{##1}}}
\expandafter\def\csname PY@tok@mh\endcsname{\def\PY@tc##1{\textcolor[rgb]{0.40,0.40,0.40}{##1}}}
\expandafter\def\csname PY@tok@cs\endcsname{\let\PY@it=\textit\def\PY@tc##1{\textcolor[rgb]{0.25,0.50,0.50}{##1}}}
\expandafter\def\csname PY@tok@ge\endcsname{\let\PY@it=\textit}
\expandafter\def\csname PY@tok@vc\endcsname{\def\PY@tc##1{\textcolor[rgb]{0.10,0.09,0.49}{##1}}}
\expandafter\def\csname PY@tok@il\endcsname{\def\PY@tc##1{\textcolor[rgb]{0.40,0.40,0.40}{##1}}}
\expandafter\def\csname PY@tok@go\endcsname{\def\PY@tc##1{\textcolor[rgb]{0.53,0.53,0.53}{##1}}}
\expandafter\def\csname PY@tok@cp\endcsname{\def\PY@tc##1{\textcolor[rgb]{0.74,0.48,0.00}{##1}}}
\expandafter\def\csname PY@tok@gi\endcsname{\def\PY@tc##1{\textcolor[rgb]{0.00,0.63,0.00}{##1}}}
\expandafter\def\csname PY@tok@gh\endcsname{\let\PY@bf=\textbf\def\PY@tc##1{\textcolor[rgb]{0.00,0.00,0.50}{##1}}}
\expandafter\def\csname PY@tok@ni\endcsname{\let\PY@bf=\textbf\def\PY@tc##1{\textcolor[rgb]{0.60,0.60,0.60}{##1}}}
\expandafter\def\csname PY@tok@nl\endcsname{\def\PY@tc##1{\textcolor[rgb]{0.63,0.63,0.00}{##1}}}
\expandafter\def\csname PY@tok@nn\endcsname{\let\PY@bf=\textbf\def\PY@tc##1{\textcolor[rgb]{0.00,0.00,1.00}{##1}}}
\expandafter\def\csname PY@tok@no\endcsname{\def\PY@tc##1{\textcolor[rgb]{0.53,0.00,0.00}{##1}}}
\expandafter\def\csname PY@tok@na\endcsname{\def\PY@tc##1{\textcolor[rgb]{0.49,0.56,0.16}{##1}}}
\expandafter\def\csname PY@tok@nb\endcsname{\def\PY@tc##1{\textcolor[rgb]{0.00,0.50,0.00}{##1}}}
\expandafter\def\csname PY@tok@nc\endcsname{\let\PY@bf=\textbf\def\PY@tc##1{\textcolor[rgb]{0.00,0.00,1.00}{##1}}}
\expandafter\def\csname PY@tok@nd\endcsname{\def\PY@tc##1{\textcolor[rgb]{0.67,0.13,1.00}{##1}}}
\expandafter\def\csname PY@tok@ne\endcsname{\let\PY@bf=\textbf\def\PY@tc##1{\textcolor[rgb]{0.82,0.25,0.23}{##1}}}
\expandafter\def\csname PY@tok@nf\endcsname{\def\PY@tc##1{\textcolor[rgb]{0.00,0.00,1.00}{##1}}}
\expandafter\def\csname PY@tok@si\endcsname{\let\PY@bf=\textbf\def\PY@tc##1{\textcolor[rgb]{0.73,0.40,0.53}{##1}}}
\expandafter\def\csname PY@tok@s2\endcsname{\def\PY@tc##1{\textcolor[rgb]{0.73,0.13,0.13}{##1}}}
\expandafter\def\csname PY@tok@nt\endcsname{\let\PY@bf=\textbf\def\PY@tc##1{\textcolor[rgb]{0.00,0.50,0.00}{##1}}}
\expandafter\def\csname PY@tok@nv\endcsname{\def\PY@tc##1{\textcolor[rgb]{0.10,0.09,0.49}{##1}}}
\expandafter\def\csname PY@tok@s1\endcsname{\def\PY@tc##1{\textcolor[rgb]{0.73,0.13,0.13}{##1}}}
\expandafter\def\csname PY@tok@dl\endcsname{\def\PY@tc##1{\textcolor[rgb]{0.73,0.13,0.13}{##1}}}
\expandafter\def\csname PY@tok@ch\endcsname{\let\PY@it=\textit\def\PY@tc##1{\textcolor[rgb]{0.25,0.50,0.50}{##1}}}
\expandafter\def\csname PY@tok@m\endcsname{\def\PY@tc##1{\textcolor[rgb]{0.40,0.40,0.40}{##1}}}
\expandafter\def\csname PY@tok@gp\endcsname{\let\PY@bf=\textbf\def\PY@tc##1{\textcolor[rgb]{0.00,0.00,0.50}{##1}}}
\expandafter\def\csname PY@tok@sh\endcsname{\def\PY@tc##1{\textcolor[rgb]{0.73,0.13,0.13}{##1}}}
\expandafter\def\csname PY@tok@ow\endcsname{\let\PY@bf=\textbf\def\PY@tc##1{\textcolor[rgb]{0.67,0.13,1.00}{##1}}}
\expandafter\def\csname PY@tok@sx\endcsname{\def\PY@tc##1{\textcolor[rgb]{0.00,0.50,0.00}{##1}}}
\expandafter\def\csname PY@tok@bp\endcsname{\def\PY@tc##1{\textcolor[rgb]{0.00,0.50,0.00}{##1}}}
\expandafter\def\csname PY@tok@c1\endcsname{\let\PY@it=\textit\def\PY@tc##1{\textcolor[rgb]{0.25,0.50,0.50}{##1}}}
\expandafter\def\csname PY@tok@fm\endcsname{\def\PY@tc##1{\textcolor[rgb]{0.00,0.00,1.00}{##1}}}
\expandafter\def\csname PY@tok@o\endcsname{\def\PY@tc##1{\textcolor[rgb]{0.40,0.40,0.40}{##1}}}
\expandafter\def\csname PY@tok@kc\endcsname{\let\PY@bf=\textbf\def\PY@tc##1{\textcolor[rgb]{0.00,0.50,0.00}{##1}}}
\expandafter\def\csname PY@tok@c\endcsname{\let\PY@it=\textit\def\PY@tc##1{\textcolor[rgb]{0.25,0.50,0.50}{##1}}}
\expandafter\def\csname PY@tok@mf\endcsname{\def\PY@tc##1{\textcolor[rgb]{0.40,0.40,0.40}{##1}}}
\expandafter\def\csname PY@tok@err\endcsname{\def\PY@bc##1{\setlength{\fboxsep}{0pt}\fcolorbox[rgb]{1.00,0.00,0.00}{1,1,1}{\strut ##1}}}
\expandafter\def\csname PY@tok@mb\endcsname{\def\PY@tc##1{\textcolor[rgb]{0.40,0.40,0.40}{##1}}}
\expandafter\def\csname PY@tok@ss\endcsname{\def\PY@tc##1{\textcolor[rgb]{0.10,0.09,0.49}{##1}}}
\expandafter\def\csname PY@tok@sr\endcsname{\def\PY@tc##1{\textcolor[rgb]{0.73,0.40,0.53}{##1}}}
\expandafter\def\csname PY@tok@mo\endcsname{\def\PY@tc##1{\textcolor[rgb]{0.40,0.40,0.40}{##1}}}
\expandafter\def\csname PY@tok@kd\endcsname{\let\PY@bf=\textbf\def\PY@tc##1{\textcolor[rgb]{0.00,0.50,0.00}{##1}}}
\expandafter\def\csname PY@tok@mi\endcsname{\def\PY@tc##1{\textcolor[rgb]{0.40,0.40,0.40}{##1}}}
\expandafter\def\csname PY@tok@kn\endcsname{\let\PY@bf=\textbf\def\PY@tc##1{\textcolor[rgb]{0.00,0.50,0.00}{##1}}}
\expandafter\def\csname PY@tok@cpf\endcsname{\let\PY@it=\textit\def\PY@tc##1{\textcolor[rgb]{0.25,0.50,0.50}{##1}}}
\expandafter\def\csname PY@tok@kr\endcsname{\let\PY@bf=\textbf\def\PY@tc##1{\textcolor[rgb]{0.00,0.50,0.00}{##1}}}
\expandafter\def\csname PY@tok@s\endcsname{\def\PY@tc##1{\textcolor[rgb]{0.73,0.13,0.13}{##1}}}
\expandafter\def\csname PY@tok@kp\endcsname{\def\PY@tc##1{\textcolor[rgb]{0.00,0.50,0.00}{##1}}}
\expandafter\def\csname PY@tok@w\endcsname{\def\PY@tc##1{\textcolor[rgb]{0.73,0.73,0.73}{##1}}}
\expandafter\def\csname PY@tok@kt\endcsname{\def\PY@tc##1{\textcolor[rgb]{0.69,0.00,0.25}{##1}}}
\expandafter\def\csname PY@tok@sc\endcsname{\def\PY@tc##1{\textcolor[rgb]{0.73,0.13,0.13}{##1}}}
\expandafter\def\csname PY@tok@sb\endcsname{\def\PY@tc##1{\textcolor[rgb]{0.73,0.13,0.13}{##1}}}
\expandafter\def\csname PY@tok@sa\endcsname{\def\PY@tc##1{\textcolor[rgb]{0.73,0.13,0.13}{##1}}}
\expandafter\def\csname PY@tok@k\endcsname{\let\PY@bf=\textbf\def\PY@tc##1{\textcolor[rgb]{0.00,0.50,0.00}{##1}}}
\expandafter\def\csname PY@tok@se\endcsname{\let\PY@bf=\textbf\def\PY@tc##1{\textcolor[rgb]{0.73,0.40,0.13}{##1}}}
\expandafter\def\csname PY@tok@sd\endcsname{\let\PY@it=\textit\def\PY@tc##1{\textcolor[rgb]{0.73,0.13,0.13}{##1}}}

\def\PYZbs{\char`\\}
\def\PYZus{\char`\_}
\def\PYZob{\char`\{}
\def\PYZcb{\char`\}}
\def\PYZca{\char`\^}
\def\PYZam{\char`\&}
\def\PYZlt{\char`\<}
\def\PYZgt{\char`\>}
\def\PYZsh{\char`\#}
\def\PYZpc{\char`\%}
\def\PYZdl{\char`\$}
\def\PYZhy{\char`\-}
\def\PYZsq{\char`\'}
\def\PYZdq{\char`\"}
\def\PYZti{\char`\~}
% for compatibility with earlier versions
\def\PYZat{@}
\def\PYZlb{[}
\def\PYZrb{]}
\makeatother


    % Exact colors from NB
    \definecolor{incolor}{rgb}{0.0, 0.0, 0.5}
    \definecolor{outcolor}{rgb}{0.545, 0.0, 0.0}



    
    % Prevent overflowing lines due to hard-to-break entities
    \sloppy 
    % Setup hyperref package
    \hypersetup{
      breaklinks=true,  % so long urls are correctly broken across lines
      colorlinks=true,
      urlcolor=urlcolor,
      linkcolor=linkcolor,
      citecolor=citecolor,
      }
    % Slightly bigger margins than the latex defaults
    
    \geometry{verbose,tmargin=1in,bmargin=1in,lmargin=1in,rmargin=1in}
    
    

    \begin{document}
    
    
    \maketitle
    
    

    
    \subsubsection{\texorpdfstring{Notebook по курсу
\href{https://stepik.org/course/4852}{"Введение в машинное
обучение"}}{Notebook по курсу "Введение в машинное обучение"}}\label{notebook-ux43fux43e-ux43aux443ux440ux441ux443-ux432ux432ux435ux434ux435ux43dux438ux435-ux432-ux43cux430ux448ux438ux43dux43dux43eux435-ux43eux431ux443ux447ux435ux43dux438ux435}

    \subsubsection{Модуль 1.4 Pandas,
DataFrames}\label{ux43cux43eux434ux443ux43bux44c-1.4-pandas-dataframes}

    \begin{Verbatim}[commandchars=\\\{\}]
{\color{incolor}In [{\color{incolor}8}]:} \PY{k+kn}{import} \PY{n+nn}{pandas} \PY{k}{as} \PY{n+nn}{pd}
        \PY{k+kn}{import} \PY{n+nn}{numpy} \PY{k}{as} \PY{n+nn}{np}
\end{Verbatim}


    \paragraph{Чтение CSV
файла}\label{ux447ux442ux435ux43dux438ux435-csv-ux444ux430ux439ux43bux430}

    \begin{Verbatim}[commandchars=\\\{\}]
{\color{incolor}In [{\color{incolor}9}]:} \PY{n}{students\PYZus{}perfomance} \PY{o}{=} \PY{n}{pd}\PY{o}{.}\PY{n}{read\PYZus{}csv}\PY{p}{(}\PY{l+s+s2}{\PYZdq{}}\PY{l+s+s2}{https://stepik.org/media/attachments/course/4852/StudentsPerformance.csv}\PY{l+s+s2}{\PYZdq{}}\PY{p}{)}
\end{Verbatim}


    \begin{Verbatim}[commandchars=\\\{\}]
{\color{incolor}In [{\color{incolor}10}]:} \PY{n}{students\PYZus{}perfomance}\PY{o}{.}\PY{n}{head}\PY{p}{(}\PY{l+m+mi}{3}\PY{p}{)} \PY{c+c1}{\PYZsh{}вывод 3 первых строк}
\end{Verbatim}


\begin{Verbatim}[commandchars=\\\{\}]
{\color{outcolor}Out[{\color{outcolor}10}]:}    gender race/ethnicity parental level of education     lunch  \textbackslash{}
         0  female        group B           bachelor's degree  standard   
         1  female        group C                some college  standard   
         2  female        group B             master's degree  standard   
         
           test preparation course  math score  reading score  writing score  
         0                    none          72             72             74  
         1               completed          69             90             88  
         2                    none          90             95             93  
\end{Verbatim}
            
    \begin{Verbatim}[commandchars=\\\{\}]
{\color{incolor}In [{\color{incolor}11}]:} \PY{n}{students\PYZus{}perfomance}\PY{o}{.}\PY{n}{iloc}\PY{p}{[}\PY{l+m+mi}{0}\PY{p}{:}\PY{l+m+mi}{2}\PY{p}{,} \PY{l+m+mi}{0}\PY{p}{:}\PY{l+m+mi}{5}\PY{p}{]} \PY{c+c1}{\PYZsh{}вывод 0\PYZhy{}2 строк, по 0\PYZhy{}5 столбцам}
\end{Verbatim}


\begin{Verbatim}[commandchars=\\\{\}]
{\color{outcolor}Out[{\color{outcolor}11}]:}    gender race/ethnicity parental level of education     lunch  \textbackslash{}
         0  female        group B           bachelor's degree  standard   
         1  female        group C                some college  standard   
         
           test preparation course  
         0                    none  
         1               completed  
\end{Verbatim}
            
    \begin{Verbatim}[commandchars=\\\{\}]
{\color{incolor}In [{\color{incolor}12}]:} \PY{n}{students\PYZus{}perfomance}\PY{o}{.}\PY{n}{iloc}\PY{p}{[}\PY{p}{[}\PY{l+m+mi}{1}\PY{p}{,} \PY{l+m+mi}{3}\PY{p}{,} \PY{l+m+mi}{5}\PY{p}{]}\PY{p}{,} \PY{p}{[}\PY{l+m+mi}{0}\PY{p}{,} \PY{l+m+mi}{5}\PY{p}{,} \PY{l+m+mi}{6}\PY{p}{]}\PY{p}{]} \PY{c+c1}{\PYZsh{}вывод 1, 3, 5 строк, по столбцам 0, 5, 6}
\end{Verbatim}


\begin{Verbatim}[commandchars=\\\{\}]
{\color{outcolor}Out[{\color{outcolor}12}]:}    gender  math score  reading score
         1  female          69             90
         3    male          47             57
         5  female          71             83
\end{Verbatim}
            
    \begin{Verbatim}[commandchars=\\\{\}]
{\color{incolor}In [{\color{incolor}13}]:} \PY{n}{students\PYZus{}perfomance\PYZus{}with\PYZus{}names} \PY{o}{=} \PY{n}{students\PYZus{}perfomance}\PY{o}{.}\PY{n}{iloc}\PY{p}{[}\PY{p}{[}\PY{l+m+mi}{0}\PY{p}{,} \PY{l+m+mi}{3}\PY{p}{,} \PY{l+m+mi}{4}\PY{p}{]}\PY{p}{]}
\end{Verbatim}


    \begin{Verbatim}[commandchars=\\\{\}]
{\color{incolor}In [{\color{incolor}14}]:} \PY{n}{students\PYZus{}perfomance\PYZus{}with\PYZus{}names}
\end{Verbatim}


\begin{Verbatim}[commandchars=\\\{\}]
{\color{outcolor}Out[{\color{outcolor}14}]:}    gender race/ethnicity parental level of education         lunch  \textbackslash{}
         0  female        group B           bachelor's degree      standard   
         3    male        group A          associate's degree  free/reduced   
         4    male        group C                some college      standard   
         
           test preparation course  math score  reading score  writing score  
         0                    none          72             72             74  
         3                    none          47             57             44  
         4                    none          76             78             75  
\end{Verbatim}
            
    \begin{Verbatim}[commandchars=\\\{\}]
{\color{incolor}In [{\color{incolor}15}]:} \PY{c+c1}{\PYZsh{}присвоить индексы}
         \PY{n}{students\PYZus{}perfomance\PYZus{}with\PYZus{}names}\PY{o}{.}\PY{n}{index} \PY{o}{=} \PY{p}{[}\PY{l+s+s1}{\PYZsq{}}\PY{l+s+s1}{Sultan}\PY{l+s+s1}{\PYZsq{}}\PY{p}{,} \PY{l+s+s1}{\PYZsq{}}\PY{l+s+s1}{Madina}\PY{l+s+s1}{\PYZsq{}}\PY{p}{,} \PY{l+s+s1}{\PYZsq{}}\PY{l+s+s1}{Amina}\PY{l+s+s1}{\PYZsq{}}\PY{p}{]}
\end{Verbatim}


    \begin{Verbatim}[commandchars=\\\{\}]
{\color{incolor}In [{\color{incolor}16}]:} \PY{c+c1}{\PYZsh{}вывод по названию индекса и столбцов}
         \PY{n}{students\PYZus{}perfomance\PYZus{}with\PYZus{}names}\PY{o}{.}\PY{n}{loc}\PY{p}{[}\PY{p}{[}\PY{l+s+s1}{\PYZsq{}}\PY{l+s+s1}{Sultan}\PY{l+s+s1}{\PYZsq{}}\PY{p}{,} \PY{l+s+s1}{\PYZsq{}}\PY{l+s+s1}{Madina}\PY{l+s+s1}{\PYZsq{}}\PY{p}{]}\PY{p}{,} \PY{p}{[}\PY{l+s+s1}{\PYZsq{}}\PY{l+s+s1}{gender}\PY{l+s+s1}{\PYZsq{}}\PY{p}{]}\PY{p}{]}
\end{Verbatim}


\begin{Verbatim}[commandchars=\\\{\}]
{\color{outcolor}Out[{\color{outcolor}16}]:}         gender
         Sultan  female
         Madina    male
\end{Verbatim}
            
    \subsubsection{Задача из модуля (шаг -
10)}\label{ux437ux430ux434ux430ux447ux430-ux438ux437-ux43cux43eux434ux443ux43bux44f-ux448ux430ux433---10}

    \begin{Verbatim}[commandchars=\\\{\}]
{\color{incolor}In [{\color{incolor}29}]:} \PY{n}{titanic\PYZus{}passengers} \PY{o}{=} \PY{n}{pd}\PY{o}{.}\PY{n}{read\PYZus{}csv}\PY{p}{(}\PY{l+s+s1}{\PYZsq{}}\PY{l+s+s1}{https://stepik.org/media/attachments/course/4852/titanic.csv}\PY{l+s+s1}{\PYZsq{}}\PY{p}{)}
\end{Verbatim}


    \begin{Verbatim}[commandchars=\\\{\}]
{\color{incolor}In [{\color{incolor}35}]:} \PY{c+c1}{\PYZsh{}кортеж (кол\PYZhy{}во строк, кол\PYZhy{}во столбцов)}
         \PY{n}{titanic\PYZus{}passengers}\PY{o}{.}\PY{n}{shape}
\end{Verbatim}


\begin{Verbatim}[commandchars=\\\{\}]
{\color{outcolor}Out[{\color{outcolor}35}]:} (891, 12)
\end{Verbatim}
            
    \begin{Verbatim}[commandchars=\\\{\}]
{\color{incolor}In [{\color{incolor}38}]:} \PY{c+c1}{\PYZsh{}вывод кол\PYZhy{}во столбцов по типу данных}
         \PY{n}{titanic\PYZus{}passengers}\PY{o}{.}\PY{n}{get\PYZus{}dtype\PYZus{}counts}\PY{p}{(}\PY{p}{)}
\end{Verbatim}


\begin{Verbatim}[commandchars=\\\{\}]
{\color{outcolor}Out[{\color{outcolor}38}]:} float64    2
         int64      5
         object     5
         dtype: int64
\end{Verbatim}
            
    \subsubsection{Модуль 1.5 Фильтрация
данных}\label{ux43cux43eux434ux443ux43bux44c-1.5-ux444ux438ux43bux44cux442ux440ux430ux446ux438ux44f-ux434ux430ux43dux43dux44bux445}

    \begin{Verbatim}[commandchars=\\\{\}]
{\color{incolor}In [{\color{incolor}1}]:} \PY{k+kn}{import} \PY{n+nn}{pandas} \PY{k}{as} \PY{n+nn}{pd}
        \PY{k+kn}{import} \PY{n+nn}{numpy} \PY{k}{as} \PY{n+nn}{np}
\end{Verbatim}


    \begin{Verbatim}[commandchars=\\\{\}]
{\color{incolor}In [{\color{incolor}68}]:} \PY{n}{students\PYZus{}perfomance} \PY{o}{=} \PY{n}{pd}\PY{o}{.}\PY{n}{read\PYZus{}csv}\PY{p}{(}\PY{l+s+s2}{\PYZdq{}}\PY{l+s+s2}{https://stepik.org/media/attachments/course/4852/StudentsPerformance.csv}\PY{l+s+s2}{\PYZdq{}}\PY{p}{)}
\end{Verbatim}


    У \emph{DataFrame} есть атрибута \emph{loc}, который поддерживает также
\emph{boolean array}, например \textbf{df.loc{[}{[}True, False{]}{]}}

    Выберем данные где, пол равен \textbf{female}

    \begin{Verbatim}[commandchars=\\\{\}]
{\color{incolor}In [{\color{incolor}18}]:} \PY{c+c1}{\PYZsh{} students\PYZus{}perfomance.gender \PYZhy{} Pandas Series}
         \PY{c+c1}{\PYZsh{} students\PYZus{}perfomance[\PYZsq{}gender\PYZsq{}] \PYZhy{} Pandas Series}
         
         \PY{c+c1}{\PYZsh{}students\PYZus{}perfomance.gender == \PYZsq{}female\PYZsq{} }
         \PY{n}{query} \PY{o}{=} \PY{n}{students\PYZus{}perfomance}\PY{p}{[}\PY{l+s+s1}{\PYZsq{}}\PY{l+s+s1}{gender}\PY{l+s+s1}{\PYZsq{}}\PY{p}{]} \PY{o}{==} \PY{l+s+s1}{\PYZsq{}}\PY{l+s+s1}{female}\PY{l+s+s1}{\PYZsq{}} \PY{c+c1}{\PYZsh{}все pandas series и сравнить ее с некоторым значением}
         \PY{c+c1}{\PYZsh{} query \PYZhy{} Pandas Series, из True и False}
         \PY{c+c1}{\PYZsh{}students\PYZus{}perfomance[query].head(5) }
         \PY{c+c1}{\PYZsh{} Можем обратиться по loc или как словарю}
         \PY{n}{students\PYZus{}perfomance}\PY{o}{.}\PY{n}{loc}\PY{p}{[}\PY{n}{query}\PY{p}{]}\PY{o}{.}\PY{n}{head}\PY{p}{(}\PY{l+m+mi}{5}\PY{p}{)}
         \PY{n}{students\PYZus{}perfomance}\PY{o}{.}\PY{n}{loc}\PY{p}{[}\PY{n}{query}\PY{p}{,} \PY{p}{[}\PY{l+s+s1}{\PYZsq{}}\PY{l+s+s1}{gender}\PY{l+s+s1}{\PYZsq{}}\PY{p}{,} \PY{l+s+s1}{\PYZsq{}}\PY{l+s+s1}{reading score}\PY{l+s+s1}{\PYZsq{}}\PY{p}{]}\PY{p}{]}\PY{o}{.}\PY{n}{head}\PY{p}{(}\PY{l+m+mi}{5}\PY{p}{)} \PY{c+c1}{\PYZsh{} Можем выбрать и разные слобцы}
\end{Verbatim}


\begin{Verbatim}[commandchars=\\\{\}]
{\color{outcolor}Out[{\color{outcolor}18}]:}    gender  reading score
         0  female             72
         1  female             90
         2  female             95
         5  female             83
         6  female             95
\end{Verbatim}
            
    \begin{Verbatim}[commandchars=\\\{\}]
{\color{incolor}In [{\color{incolor}26}]:} \PY{c+c1}{\PYZsh{} Посмотреть среднее значение}
         \PY{n}{students\PYZus{}perfomance}\PY{o}{.}\PY{n}{loc}\PY{p}{[}\PY{p}{:}\PY{p}{,}\PY{l+s+s1}{\PYZsq{}}\PY{l+s+s1}{writing score}\PY{l+s+s1}{\PYZsq{}}\PY{p}{]}\PY{o}{.}\PY{n}{mean}\PY{p}{(}\PY{p}{)}
         
         \PY{c+c1}{\PYZsh{}или}
         
         \PY{n}{mean\PYZus{}writing\PYZus{}score} \PY{o}{=} \PY{n}{students\PYZus{}perfomance}\PY{p}{[}\PY{l+s+s1}{\PYZsq{}}\PY{l+s+s1}{writing score}\PY{l+s+s1}{\PYZsq{}}\PY{p}{]}\PY{o}{.}\PY{n}{mean}\PY{p}{(}\PY{p}{)}
         \PY{n}{mean\PYZus{}writing\PYZus{}score}
\end{Verbatim}


\begin{Verbatim}[commandchars=\\\{\}]
{\color{outcolor}Out[{\color{outcolor}26}]:} 68.054
\end{Verbatim}
            
    \begin{Verbatim}[commandchars=\\\{\}]
{\color{incolor}In [{\color{incolor}28}]:} \PY{n}{students\PYZus{}perfomance}\PY{o}{.}\PY{n}{loc}\PY{p}{[}\PY{n}{students\PYZus{}perfomance}\PY{p}{[}\PY{l+s+s1}{\PYZsq{}}\PY{l+s+s1}{writing score}\PY{l+s+s1}{\PYZsq{}}\PY{p}{]} \PY{o}{\PYZgt{}} \PY{n}{mean\PYZus{}writing\PYZus{}score}\PY{p}{]}\PY{o}{.}\PY{n}{head}\PY{p}{(}\PY{l+m+mi}{5}\PY{p}{)}
\end{Verbatim}


\begin{Verbatim}[commandchars=\\\{\}]
{\color{outcolor}Out[{\color{outcolor}28}]:}    gender race/ethnicity parental level of education     lunch  \textbackslash{}
         0  female        group B           bachelor's degree  standard   
         1  female        group C                some college  standard   
         2  female        group B             master's degree  standard   
         4    male        group C                some college  standard   
         5  female        group B          associate's degree  standard   
         
           test preparation course  math score  reading score  writing score  
         0                    none          72             72             74  
         1               completed          69             90             88  
         2                    none          90             95             93  
         4                    none          76             78             75  
         5                    none          71             83             78  
\end{Verbatim}
            
    \subparagraph{Комбинация условий для
фильтрации}\label{ux43aux43eux43cux431ux438ux43dux430ux446ux438ux44f-ux443ux441ux43bux43eux432ux438ux439-ux434ux43bux44f-ux444ux438ux43bux44cux442ux440ux430ux446ux438ux438}

Выражения в Pandas, \emph{and} - \textbf{\&}, \emph{or} -
\textbf{\textbar{}}

    \begin{Verbatim}[commandchars=\\\{\}]
{\color{incolor}In [{\color{incolor}29}]:} \PY{n}{query} \PY{o}{=} \PY{p}{(}\PY{n}{students\PYZus{}perfomance}\PY{p}{[}\PY{l+s+s1}{\PYZsq{}}\PY{l+s+s1}{writing score}\PY{l+s+s1}{\PYZsq{}}\PY{p}{]} \PY{o}{\PYZgt{}} \PY{n}{mean\PYZus{}writing\PYZus{}score}\PY{p}{)} \PY{o}{\PYZam{}} \PY{p}{(}\PY{n}{students\PYZus{}perfomance}\PY{p}{[}\PY{l+s+s1}{\PYZsq{}}\PY{l+s+s1}{gender}\PY{l+s+s1}{\PYZsq{}}\PY{p}{]} \PY{o}{==} \PY{l+s+s1}{\PYZsq{}}\PY{l+s+s1}{female}\PY{l+s+s1}{\PYZsq{}}\PY{p}{)}
         
         \PY{n}{students\PYZus{}perfomance}\PY{o}{.}\PY{n}{loc}\PY{p}{[}\PY{n}{query}\PY{p}{]}\PY{o}{.}\PY{n}{head}\PY{p}{(}\PY{l+m+mi}{5}\PY{p}{)}
\end{Verbatim}


\begin{Verbatim}[commandchars=\\\{\}]
{\color{outcolor}Out[{\color{outcolor}29}]:}    gender race/ethnicity parental level of education     lunch  \textbackslash{}
         0  female        group B           bachelor's degree  standard   
         1  female        group C                some college  standard   
         2  female        group B             master's degree  standard   
         5  female        group B          associate's degree  standard   
         6  female        group B                some college  standard   
         
           test preparation course  math score  reading score  writing score  
         0                    none          72             72             74  
         1               completed          69             90             88  
         2                    none          90             95             93  
         5                    none          71             83             78  
         6               completed          88             95             92  
\end{Verbatim}
            
    \paragraph{Фильтрация через
query}\label{ux444ux438ux43bux44cux442ux440ux430ux446ux438ux44f-ux447ux435ux440ux435ux437-query}

    \begin{Verbatim}[commandchars=\\\{\}]
{\color{incolor}In [{\color{incolor}69}]:} \PY{c+c1}{\PYZsh{} переименуем колонки для удобной фильтрации}
         
         \PY{n}{students\PYZus{}perfomance}\PY{o}{.}\PY{n}{columns} \PY{o}{=} \PY{p}{[}\PY{n}{x}\PY{o}{.}\PY{n}{replace}\PY{p}{(}\PY{l+s+s1}{\PYZsq{}}\PY{l+s+s1}{ }\PY{l+s+s1}{\PYZsq{}}\PY{p}{,} \PY{l+s+s1}{\PYZsq{}}\PY{l+s+s1}{\PYZus{}}\PY{l+s+s1}{\PYZsq{}}\PY{p}{)} \PY{k}{for} \PY{n}{x} \PY{o+ow}{in} \PY{n}{students\PYZus{}perfomance}\PY{p}{]}
         \PY{n}{students\PYZus{}perfomance}\PY{o}{.}\PY{n}{head}\PY{p}{(}\PY{p}{)}
\end{Verbatim}


\begin{Verbatim}[commandchars=\\\{\}]
{\color{outcolor}Out[{\color{outcolor}69}]:}    gender race/ethnicity parental\_level\_of\_education         lunch  \textbackslash{}
         0  female        group B           bachelor's degree      standard   
         1  female        group C                some college      standard   
         2  female        group B             master's degree      standard   
         3    male        group A          associate's degree  free/reduced   
         4    male        group C                some college      standard   
         
           test\_preparation\_course  math\_score  reading\_score  writing\_score  
         0                    none          72             72             74  
         1               completed          69             90             88  
         2                    none          90             95             93  
         3                    none          47             57             44  
         4                    none          76             78             75  
\end{Verbatim}
            
    \begin{Verbatim}[commandchars=\\\{\}]
{\color{incolor}In [{\color{incolor}71}]:} \PY{n}{students\PYZus{}perfomance}\PY{o}{.}\PY{n}{query}\PY{p}{(}\PY{l+s+s2}{\PYZdq{}}\PY{l+s+s2}{writing\PYZus{}score \PYZgt{} 74}\PY{l+s+s2}{\PYZdq{}}\PY{p}{)}\PY{o}{.}\PY{n}{head}\PY{p}{(}\PY{p}{)}
\end{Verbatim}


\begin{Verbatim}[commandchars=\\\{\}]
{\color{outcolor}Out[{\color{outcolor}71}]:}    gender race/ethnicity parental\_level\_of\_education     lunch  \textbackslash{}
         1  female        group C                some college  standard   
         2  female        group B             master's degree  standard   
         4    male        group C                some college  standard   
         5  female        group B          associate's degree  standard   
         6  female        group B                some college  standard   
         
           test\_preparation\_course  math\_score  reading\_score  writing\_score  
         1               completed          69             90             88  
         2                    none          90             95             93  
         4                    none          76             78             75  
         5                    none          71             83             78  
         6               completed          88             95             92  
\end{Verbatim}
            
    \begin{Verbatim}[commandchars=\\\{\}]
{\color{incolor}In [{\color{incolor}74}]:} \PY{n}{students\PYZus{}perfomance}\PY{o}{.}\PY{n}{query}\PY{p}{(}\PY{l+s+s2}{\PYZdq{}}\PY{l+s+s2}{writing\PYZus{}score \PYZgt{} 74 \PYZam{} gender == }\PY{l+s+s2}{\PYZsq{}}\PY{l+s+s2}{male}\PY{l+s+s2}{\PYZsq{}}\PY{l+s+s2}{\PYZdq{}}\PY{p}{)}\PY{o}{.}\PY{n}{head}\PY{p}{(}\PY{p}{)} \PY{c+c1}{\PYZsh{}комбинированное условие}
\end{Verbatim}


\begin{Verbatim}[commandchars=\\\{\}]
{\color{outcolor}Out[{\color{outcolor}74}]:}    gender race/ethnicity parental\_level\_of\_education         lunch  \textbackslash{}
         4    male        group C                some college      standard   
         16   male        group C                 high school      standard   
         24   male        group D           bachelor's degree  free/reduced   
         34   male        group E                some college      standard   
         35   male        group E          associate's degree      standard   
         
            test\_preparation\_course  math\_score  reading\_score  writing\_score  
         4                     none          76             78             75  
         16                    none          88             89             86  
         24               completed          74             71             80  
         34                    none          97             87             82  
         35               completed          81             81             79  
\end{Verbatim}
            
    \begin{Verbatim}[commandchars=\\\{\}]
{\color{incolor}In [{\color{incolor}75}]:} \PY{n}{writing\PYZus{}score\PYZus{}query} \PY{o}{=} \PY{l+m+mi}{80}
         
         \PY{n}{students\PYZus{}perfomance}\PY{o}{.}\PY{n}{query}\PY{p}{(}\PY{l+s+s2}{\PYZdq{}}\PY{l+s+s2}{writing\PYZus{}score \PYZgt{} @writing\PYZus{}score\PYZus{}query}\PY{l+s+s2}{\PYZdq{}}\PY{p}{)}\PY{o}{.}\PY{n}{head}\PY{p}{(}\PY{p}{)} \PY{c+c1}{\PYZsh{}фильтрация через переменную}
\end{Verbatim}


\begin{Verbatim}[commandchars=\\\{\}]
{\color{outcolor}Out[{\color{outcolor}75}]:}     gender race/ethnicity parental\_level\_of\_education     lunch  \textbackslash{}
         1   female        group C                some college  standard   
         2   female        group B             master's degree  standard   
         6   female        group B                some college  standard   
         16    male        group C                 high school  standard   
         34    male        group E                some college  standard   
         
            test\_preparation\_course  math\_score  reading\_score  writing\_score  
         1                completed          69             90             88  
         2                     none          90             95             93  
         6                completed          88             95             92  
         16                    none          88             89             86  
         34                    none          97             87             82  
\end{Verbatim}
            
    \paragraph{Отобрать колонки по каким нибудь
признакам}\label{ux43eux442ux43eux431ux440ux430ux442ux44c-ux43aux43eux43bux43eux43dux43aux438-ux43fux43e-ux43aux430ux43aux438ux43c-ux43dux438ux431ux443ux434ux44c-ux43fux440ux438ux437ux43dux430ux43aux430ux43c}

    \begin{Verbatim}[commandchars=\\\{\}]
{\color{incolor}In [{\color{incolor}80}]:} \PY{c+c1}{\PYZsh{}по питоновски}
         
         \PY{n}{column\PYZus{}names} \PY{o}{=} \PY{n+nb}{list}\PY{p}{(}\PY{n}{students\PYZus{}perfomance}\PY{p}{)}
         \PY{n}{column\PYZus{}names} \PY{c+c1}{\PYZsh{}список наименований столбцов}
         
         \PY{n}{score\PYZus{}columns} \PY{o}{=} \PY{p}{[}\PY{n}{i} \PY{k}{for} \PY{n}{i} \PY{o+ow}{in} \PY{n+nb}{list}\PY{p}{(}\PY{n}{students\PYZus{}perfomance}\PY{p}{)} \PY{k}{if} \PY{l+s+s1}{\PYZsq{}}\PY{l+s+s1}{score}\PY{l+s+s1}{\PYZsq{}} \PY{o+ow}{in} \PY{n}{i}\PY{p}{]}
         \PY{n}{students\PYZus{}perfomance}\PY{p}{[}\PY{n}{score\PYZus{}columns}\PY{p}{]}\PY{o}{.}\PY{n}{head}\PY{p}{(}\PY{p}{)}
\end{Verbatim}


\begin{Verbatim}[commandchars=\\\{\}]
{\color{outcolor}Out[{\color{outcolor}80}]:}    math\_score  reading\_score  writing\_score
         0          72             72             74
         1          69             90             88
         2          90             95             93
         3          47             57             44
         4          76             78             75
\end{Verbatim}
            
    \begin{Verbatim}[commandchars=\\\{\}]
{\color{incolor}In [{\color{incolor} }]:} \PY{c+c1}{\PYZsh{}по Pandas}
        
        \PY{n}{s} \PY{o}{=} \PY{n}{students\PYZus{}perfomance}\PY{o}{.}\PY{n}{filter}\PY{p}{(}\PY{n}{like}\PY{o}{=}\PY{l+s+s1}{\PYZsq{}}\PY{l+s+s1}{score}\PY{l+s+s1}{\PYZsq{}}\PY{p}{)}
        \PY{n}{s}\PY{o}{.}\PY{n}{head}\PY{p}{(}\PY{p}{)}
        
        \PY{c+c1}{\PYZsh{}s = students\PYZus{}perfomance.filter(like=\PYZsq{}row\PYZus{}name\PYZsq{}, axis=0) \PYZsh{}axis = 0 \PYZhy{} фильтрация по строкам}
        \PY{c+c1}{\PYZsh{}s.head()}
\end{Verbatim}


    \subsubsection{Задача из модуля (шаг -
6)}\label{ux437ux430ux434ux430ux447ux430-ux438ux437-ux43cux43eux434ux443ux43bux44f-ux448ux430ux433---6}

    У какой доли студентов из
\href{'https://stepik.org/media/attachments/course/4852/StudentsPerformance.csv'}{датасэта}
в колонке lunch указано free/reduced?

Формат ответа десятичная дробь, например, 0.25

    \begin{Verbatim}[commandchars=\\\{\}]
{\color{incolor}In [{\color{incolor}31}]:} \PY{k+kn}{import} \PY{n+nn}{pandas} \PY{k}{as} \PY{n+nn}{pd}
         
         \PY{n}{df} \PY{o}{=} \PY{n}{pd}\PY{o}{.}\PY{n}{read\PYZus{}csv}\PY{p}{(}\PY{l+s+s1}{\PYZsq{}}\PY{l+s+s1}{https://stepik.org/media/attachments/course/4852/StudentsPerformance.csv}\PY{l+s+s1}{\PYZsq{}}\PY{p}{)}
\end{Verbatim}


    \textbf{Вариант по питоновски}

    \begin{Verbatim}[commandchars=\\\{\}]
{\color{incolor}In [{\color{incolor}40}]:} \PY{n}{query} \PY{o}{=} \PY{n}{df}\PY{p}{[}\PY{l+s+s1}{\PYZsq{}}\PY{l+s+s1}{lunch}\PY{l+s+s1}{\PYZsq{}}\PY{p}{]} \PY{o}{==} \PY{l+s+s1}{\PYZsq{}}\PY{l+s+s1}{free/reduced}\PY{l+s+s1}{\PYZsq{}}
         \PY{n}{free\PYZus{}reduced\PYZus{}count} \PY{o}{=} \PY{n}{df}\PY{o}{.}\PY{n}{loc}\PY{p}{[}\PY{n}{query}\PY{p}{]}\PY{o}{.}\PY{n}{shape}\PY{p}{[}\PY{l+m+mi}{0}\PY{p}{]} \PY{c+c1}{\PYZsh{}количество студентов (строк), где free/reduced}
         \PY{n}{all\PYZus{}count} \PY{o}{=} \PY{n}{df}\PY{o}{.}\PY{n}{shape}\PY{p}{[}\PY{l+m+mi}{0}\PY{p}{]} \PY{c+c1}{\PYZsh{}количество всех студентов}
         
         \PY{n}{result} \PY{o}{=} \PY{n}{free\PYZus{}reduced\PYZus{}count} \PY{o}{/} \PY{n}{all\PYZus{}count}
         \PY{n}{result}
\end{Verbatim}


\begin{Verbatim}[commandchars=\\\{\}]
{\color{outcolor}Out[{\color{outcolor}40}]:} 0.355
\end{Verbatim}
            
    \textbf{Вариант со статистикой - Нормалицазия}

    \begin{Verbatim}[commandchars=\\\{\}]
{\color{incolor}In [{\color{incolor}45}]:} \PY{n}{df}\PY{p}{[}\PY{l+s+s1}{\PYZsq{}}\PY{l+s+s1}{lunch}\PY{l+s+s1}{\PYZsq{}}\PY{p}{]}\PY{o}{.}\PY{n}{value\PYZus{}counts}\PY{p}{(}\PY{p}{)}
         \PY{n}{df}\PY{p}{[}\PY{l+s+s1}{\PYZsq{}}\PY{l+s+s1}{lunch}\PY{l+s+s1}{\PYZsq{}}\PY{p}{]}\PY{o}{.}\PY{n}{value\PYZus{}counts}\PY{p}{(}\PY{n}{normalize}\PY{o}{=}\PY{k+kc}{True}\PY{p}{)}
\end{Verbatim}


\begin{Verbatim}[commandchars=\\\{\}]
{\color{outcolor}Out[{\color{outcolor}45}]:} standard        0.645
         free/reduced    0.355
         Name: lunch, dtype: float64
\end{Verbatim}
            
    \subsubsection{Задача из модуля (шаг -
7)}\label{ux437ux430ux434ux430ux447ux430-ux438ux437-ux43cux43eux434ux443ux43bux44f-ux448ux430ux433---7}

Как различается
\href{https://pandas.pydata.org/pandas-docs/stable/reference/api/pandas.DataFrame.mean.html}{среднее}
и
\href{https://pandas.pydata.org/pandas-docs/stable/reference/api/pandas.DataFrame.var.html}{дисперсия}
оценок по предметам у групп
\href{https://stepik.org/media/attachments/course/4852/StudentsPerformance.csv}{студентов}
со стандартным или урезанным ланчем?

    \begin{Verbatim}[commandchars=\\\{\}]
{\color{incolor}In [{\color{incolor}46}]:} \PY{k+kn}{import} \PY{n+nn}{pandas} \PY{k}{as} \PY{n+nn}{pd}
         
         \PY{n}{df} \PY{o}{=} \PY{n}{pd}\PY{o}{.}\PY{n}{read\PYZus{}csv}\PY{p}{(}\PY{l+s+s1}{\PYZsq{}}\PY{l+s+s1}{https://stepik.org/media/attachments/course/4852/StudentsPerformance.csv}\PY{l+s+s1}{\PYZsq{}}\PY{p}{)}
\end{Verbatim}


    \begin{Verbatim}[commandchars=\\\{\}]
{\color{incolor}In [{\color{incolor}48}]:} \PY{n}{df}\PY{o}{.}\PY{n}{describe}\PY{p}{(}\PY{p}{)} \PY{c+c1}{\PYZsh{} показывает основные статистические характеристики}
\end{Verbatim}


\begin{Verbatim}[commandchars=\\\{\}]
{\color{outcolor}Out[{\color{outcolor}48}]:}        math score  reading score  writing score
         count  1000.00000    1000.000000    1000.000000
         mean     66.08900      69.169000      68.054000
         std      15.16308      14.600192      15.195657
         min       0.00000      17.000000      10.000000
         25\%      57.00000      59.000000      57.750000
         50\%      66.00000      70.000000      69.000000
         75\%      77.00000      79.000000      79.000000
         max     100.00000     100.000000     100.000000
\end{Verbatim}
            
    \paragraph{Среднее оценок по предметам со стандартным
обедом}\label{ux441ux440ux435ux434ux43dux435ux435-ux43eux446ux435ux43dux43eux43a-ux43fux43e-ux43fux440ux435ux434ux43cux435ux442ux430ux43c-ux441ux43e-ux441ux442ux430ux43dux434ux430ux440ux442ux43dux44bux43c-ux43eux431ux435ux434ux43eux43c}

    \begin{Verbatim}[commandchars=\\\{\}]
{\color{incolor}In [{\color{incolor}59}]:} \PY{n}{query1} \PY{o}{=} \PY{n}{df}\PY{p}{[}\PY{l+s+s1}{\PYZsq{}}\PY{l+s+s1}{lunch}\PY{l+s+s1}{\PYZsq{}}\PY{p}{]} \PY{o}{==} \PY{l+s+s1}{\PYZsq{}}\PY{l+s+s1}{standard}\PY{l+s+s1}{\PYZsq{}}
         \PY{n}{df}\PY{o}{.}\PY{n}{loc}\PY{p}{[}\PY{n}{query1}\PY{p}{,} \PY{p}{[}\PY{l+s+s1}{\PYZsq{}}\PY{l+s+s1}{lunch}\PY{l+s+s1}{\PYZsq{}}\PY{p}{,} \PY{l+s+s1}{\PYZsq{}}\PY{l+s+s1}{math score}\PY{l+s+s1}{\PYZsq{}}\PY{p}{,} \PY{l+s+s1}{\PYZsq{}}\PY{l+s+s1}{reading score}\PY{l+s+s1}{\PYZsq{}}\PY{p}{,} \PY{l+s+s1}{\PYZsq{}}\PY{l+s+s1}{writing score}\PY{l+s+s1}{\PYZsq{}}\PY{p}{]}\PY{p}{]}\PY{o}{.}\PY{n}{mean}\PY{p}{(}\PY{p}{)}
\end{Verbatim}


\begin{Verbatim}[commandchars=\\\{\}]
{\color{outcolor}Out[{\color{outcolor}59}]:} math score       70.034109
         reading score    71.654264
         writing score    70.823256
         dtype: float64
\end{Verbatim}
            
    \paragraph{Среднее оценок по предметам со урезанным
обедом}\label{ux441ux440ux435ux434ux43dux435ux435-ux43eux446ux435ux43dux43eux43a-ux43fux43e-ux43fux440ux435ux434ux43cux435ux442ux430ux43c-ux441ux43e-ux443ux440ux435ux437ux430ux43dux43dux44bux43c-ux43eux431ux435ux434ux43eux43c}

    \begin{Verbatim}[commandchars=\\\{\}]
{\color{incolor}In [{\color{incolor}58}]:} \PY{n}{query2} \PY{o}{=} \PY{n}{df}\PY{p}{[}\PY{l+s+s1}{\PYZsq{}}\PY{l+s+s1}{lunch}\PY{l+s+s1}{\PYZsq{}}\PY{p}{]} \PY{o}{==} \PY{l+s+s1}{\PYZsq{}}\PY{l+s+s1}{free/reduced}\PY{l+s+s1}{\PYZsq{}}
         \PY{n}{df}\PY{o}{.}\PY{n}{loc}\PY{p}{[}\PY{n}{query2}\PY{p}{,} \PY{p}{[}\PY{l+s+s1}{\PYZsq{}}\PY{l+s+s1}{lunch}\PY{l+s+s1}{\PYZsq{}}\PY{p}{,} \PY{l+s+s1}{\PYZsq{}}\PY{l+s+s1}{math score}\PY{l+s+s1}{\PYZsq{}}\PY{p}{,} \PY{l+s+s1}{\PYZsq{}}\PY{l+s+s1}{reading score}\PY{l+s+s1}{\PYZsq{}}\PY{p}{,} \PY{l+s+s1}{\PYZsq{}}\PY{l+s+s1}{writing score}\PY{l+s+s1}{\PYZsq{}}\PY{p}{]}\PY{p}{]}\PY{o}{.}\PY{n}{mean}\PY{p}{(}\PY{p}{)}
\end{Verbatim}


\begin{Verbatim}[commandchars=\\\{\}]
{\color{outcolor}Out[{\color{outcolor}58}]:} math score       58.921127
         reading score    64.653521
         writing score    63.022535
         dtype: float64
\end{Verbatim}
            
    \paragraph{Дисперсия оценок по предметам со стандартным
обедом}\label{ux434ux438ux441ux43fux435ux440ux441ux438ux44f-ux43eux446ux435ux43dux43eux43a-ux43fux43e-ux43fux440ux435ux434ux43cux435ux442ux430ux43c-ux441ux43e-ux441ux442ux430ux43dux434ux430ux440ux442ux43dux44bux43c-ux43eux431ux435ux434ux43eux43c}

    \begin{Verbatim}[commandchars=\\\{\}]
{\color{incolor}In [{\color{incolor}64}]:} \PY{n}{query1} \PY{o}{=} \PY{n}{df}\PY{p}{[}\PY{l+s+s1}{\PYZsq{}}\PY{l+s+s1}{lunch}\PY{l+s+s1}{\PYZsq{}}\PY{p}{]} \PY{o}{==} \PY{l+s+s1}{\PYZsq{}}\PY{l+s+s1}{standard}\PY{l+s+s1}{\PYZsq{}}
         \PY{n}{df}\PY{o}{.}\PY{n}{loc}\PY{p}{[}\PY{n}{query1}\PY{p}{,} \PY{p}{[}\PY{l+s+s1}{\PYZsq{}}\PY{l+s+s1}{lunch}\PY{l+s+s1}{\PYZsq{}}\PY{p}{,} \PY{l+s+s1}{\PYZsq{}}\PY{l+s+s1}{math score}\PY{l+s+s1}{\PYZsq{}}\PY{p}{,} \PY{l+s+s1}{\PYZsq{}}\PY{l+s+s1}{reading score}\PY{l+s+s1}{\PYZsq{}}\PY{p}{,} \PY{l+s+s1}{\PYZsq{}}\PY{l+s+s1}{writing score}\PY{l+s+s1}{\PYZsq{}}\PY{p}{]}\PY{p}{]}\PY{o}{.}\PY{n}{var}\PY{p}{(}\PY{p}{)}
\end{Verbatim}


\begin{Verbatim}[commandchars=\\\{\}]
{\color{outcolor}Out[{\color{outcolor}64}]:} math score       186.418089
         reading score    191.285560
         writing score    205.620887
         dtype: float64
\end{Verbatim}
            
    \paragraph{Дисперсия оценок по предметам со урезанным
обедом}\label{ux434ux438ux441ux43fux435ux440ux441ux438ux44f-ux43eux446ux435ux43dux43eux43a-ux43fux43e-ux43fux440ux435ux434ux43cux435ux442ux430ux43c-ux441ux43e-ux443ux440ux435ux437ux430ux43dux43dux44bux43c-ux43eux431ux435ux434ux43eux43c}

    \begin{Verbatim}[commandchars=\\\{\}]
{\color{incolor}In [{\color{incolor}61}]:} \PY{n}{query2} \PY{o}{=} \PY{n}{df}\PY{p}{[}\PY{l+s+s1}{\PYZsq{}}\PY{l+s+s1}{lunch}\PY{l+s+s1}{\PYZsq{}}\PY{p}{]} \PY{o}{==} \PY{l+s+s1}{\PYZsq{}}\PY{l+s+s1}{free/reduced}\PY{l+s+s1}{\PYZsq{}}
         \PY{n}{df}\PY{o}{.}\PY{n}{loc}\PY{p}{[}\PY{n}{query2}\PY{p}{,} \PY{p}{[}\PY{l+s+s1}{\PYZsq{}}\PY{l+s+s1}{lunch}\PY{l+s+s1}{\PYZsq{}}\PY{p}{,} \PY{l+s+s1}{\PYZsq{}}\PY{l+s+s1}{math score}\PY{l+s+s1}{\PYZsq{}}\PY{p}{,} \PY{l+s+s1}{\PYZsq{}}\PY{l+s+s1}{reading score}\PY{l+s+s1}{\PYZsq{}}\PY{p}{,} \PY{l+s+s1}{\PYZsq{}}\PY{l+s+s1}{writing score}\PY{l+s+s1}{\PYZsq{}}\PY{p}{]}\PY{p}{]}\PY{o}{.}\PY{n}{var}\PY{p}{(}\PY{p}{)}
\end{Verbatim}


\begin{Verbatim}[commandchars=\\\{\}]
{\color{outcolor}Out[{\color{outcolor}61}]:} math score       229.824270
         reading score    221.871139
         writing score    238.202881
         dtype: float64
\end{Verbatim}
            
    \subsubsection{Модуль 1.6 Группировка данных и
агрегация}\label{ux43cux43eux434ux443ux43bux44c-1.6-ux433ux440ux443ux43fux43fux438ux440ux43eux432ux43aux430-ux434ux430ux43dux43dux44bux445-ux438-ux430ux433ux440ux435ux433ux430ux446ux438ux44f}

    \begin{Verbatim}[commandchars=\\\{\}]
{\color{incolor}In [{\color{incolor}3}]:} \PY{k+kn}{import} \PY{n+nn}{pandas} \PY{k}{as} \PY{n+nn}{pd}
        \PY{k+kn}{import} \PY{n+nn}{numpy} \PY{k}{as} \PY{n+nn}{np}
        
        \PY{n}{students\PYZus{}perfomance} \PY{o}{=} \PY{n}{pd}\PY{o}{.}\PY{n}{read\PYZus{}csv}\PY{p}{(}\PY{l+s+s1}{\PYZsq{}}\PY{l+s+s1}{https://stepik.org/media/attachments/course/4852/StudentsPerformance.csv}\PY{l+s+s1}{\PYZsq{}}\PY{p}{)}
\end{Verbatim}


    \begin{Verbatim}[commandchars=\\\{\}]
{\color{incolor}In [{\color{incolor}5}]:} \PY{n}{students\PYZus{}perfomance}\PY{o}{.}\PY{n}{head}\PY{p}{(}\PY{p}{)}
\end{Verbatim}


\begin{Verbatim}[commandchars=\\\{\}]
{\color{outcolor}Out[{\color{outcolor}5}]:}    gender race/ethnicity parental level of education         lunch  \textbackslash{}
        0  female        group B           bachelor's degree      standard   
        1  female        group C                some college      standard   
        2  female        group B             master's degree      standard   
        3    male        group A          associate's degree  free/reduced   
        4    male        group C                some college      standard   
        
          test preparation course  math score  reading score  writing score  
        0                    none          72             72             74  
        1               completed          69             90             88  
        2                    none          90             95             93  
        3                    none          47             57             44  
        4                    none          76             78             75  
\end{Verbatim}
            
    \begin{Verbatim}[commandchars=\\\{\}]
{\color{incolor}In [{\color{incolor}14}]:} \PY{c+c1}{\PYZsh{}посчитаем среднее score\PYZhy{}ов в зависивости от полов студентов}
         
         \PY{n}{students\PYZus{}perfomance}\PY{o}{.}\PY{n}{groupby}\PY{p}{(}\PY{l+s+s1}{\PYZsq{}}\PY{l+s+s1}{gender}\PY{l+s+s1}{\PYZsq{}}\PY{p}{)}
         \PY{c+c1}{\PYZsh{}результат сгруппированный DataFrame}
\end{Verbatim}


\begin{Verbatim}[commandchars=\\\{\}]
{\color{outcolor}Out[{\color{outcolor}14}]:} <pandas.core.groupby.groupby.DataFrameGroupBy object at 0x7f5f15d9a780>
\end{Verbatim}
            
    \begin{Verbatim}[commandchars=\\\{\}]
{\color{incolor}In [{\color{incolor}15}]:} \PY{n}{students\PYZus{}perfomance}\PY{o}{.}\PY{n}{groupby}\PY{p}{(}\PY{l+s+s1}{\PYZsq{}}\PY{l+s+s1}{gender}\PY{l+s+s1}{\PYZsq{}}\PY{p}{)}\PY{o}{.}\PY{n}{mean}\PY{p}{(}\PY{p}{)} \PY{c+c1}{\PYZsh{}пременили агрегацию mean}
         \PY{c+c1}{\PYZsh{}среднее значение количественных переменных}
\end{Verbatim}


\begin{Verbatim}[commandchars=\\\{\}]
{\color{outcolor}Out[{\color{outcolor}15}]:}         math score  reading score  writing score
         gender                                          
         female   63.633205      72.608108      72.467181
         male     68.728216      65.473029      63.311203
\end{Verbatim}
            
    \begin{Verbatim}[commandchars=\\\{\}]
{\color{incolor}In [{\color{incolor}18}]:} \PY{c+c1}{\PYZsh{}какие переменные использовать для агрегации}
         \PY{n}{students\PYZus{}perfomance}\PY{o}{.}\PY{n}{groupby}\PY{p}{(}\PY{l+s+s1}{\PYZsq{}}\PY{l+s+s1}{gender}\PY{l+s+s1}{\PYZsq{}}\PY{p}{)}\PY{o}{.}\PY{n}{aggregate}\PY{p}{(}\PY{p}{\PYZob{}}\PY{l+s+s1}{\PYZsq{}}\PY{l+s+s1}{math score}\PY{l+s+s1}{\PYZsq{}}\PY{p}{:} \PY{l+s+s1}{\PYZsq{}}\PY{l+s+s1}{mean}\PY{l+s+s1}{\PYZsq{}}\PY{p}{\PYZcb{}}\PY{p}{)}
\end{Verbatim}


\begin{Verbatim}[commandchars=\\\{\}]
{\color{outcolor}Out[{\color{outcolor}18}]:}         math score
         gender            
         female   63.633205
         male     68.728216
\end{Verbatim}
            
    \begin{Verbatim}[commandchars=\\\{\}]
{\color{incolor}In [{\color{incolor}20}]:} \PY{c+c1}{\PYZsh{}можно также использовать несколько переменных}
         \PY{n}{students\PYZus{}perfomance}\PY{o}{.}\PY{n}{groupby}\PY{p}{(}\PY{l+s+s1}{\PYZsq{}}\PY{l+s+s1}{gender}\PY{l+s+s1}{\PYZsq{}}\PY{p}{)}\PY{o}{.}\PY{n}{aggregate}\PY{p}{(}\PY{p}{\PYZob{}}\PY{l+s+s1}{\PYZsq{}}\PY{l+s+s1}{math score}\PY{l+s+s1}{\PYZsq{}}\PY{p}{:} \PY{l+s+s1}{\PYZsq{}}\PY{l+s+s1}{mean}\PY{l+s+s1}{\PYZsq{}}\PY{p}{,} \PY{l+s+s1}{\PYZsq{}}\PY{l+s+s1}{reading score}\PY{l+s+s1}{\PYZsq{}}\PY{p}{:} \PY{l+s+s1}{\PYZsq{}}\PY{l+s+s1}{mean}\PY{l+s+s1}{\PYZsq{}}\PY{p}{\PYZcb{}}\PY{p}{)}
\end{Verbatim}


\begin{Verbatim}[commandchars=\\\{\}]
{\color{outcolor}Out[{\color{outcolor}20}]:}         math score  reading score
         gender                           
         female   63.633205      72.608108
         male     68.728216      65.473029
\end{Verbatim}
            
    \begin{Verbatim}[commandchars=\\\{\}]
{\color{incolor}In [{\color{incolor}21}]:} \PY{c+c1}{\PYZsh{}в результате мы получаем DataFrame с индексами, которые мы использовали для группировки}
         \PY{n}{students\PYZus{}perfomance}\PY{o}{.}\PY{n}{groupby}\PY{p}{(}\PY{l+s+s1}{\PYZsq{}}\PY{l+s+s1}{gender}\PY{l+s+s1}{\PYZsq{}}\PY{p}{)}\PY{o}{.}\PY{n}{aggregate}\PY{p}{(}\PY{p}{\PYZob{}}\PY{l+s+s1}{\PYZsq{}}\PY{l+s+s1}{math score}\PY{l+s+s1}{\PYZsq{}}\PY{p}{:} \PY{l+s+s1}{\PYZsq{}}\PY{l+s+s1}{mean}\PY{l+s+s1}{\PYZsq{}}\PY{p}{,} \PY{l+s+s1}{\PYZsq{}}\PY{l+s+s1}{reading score}\PY{l+s+s1}{\PYZsq{}}\PY{p}{:} \PY{l+s+s1}{\PYZsq{}}\PY{l+s+s1}{mean}\PY{l+s+s1}{\PYZsq{}}\PY{p}{\PYZcb{}}\PY{p}{)}
\end{Verbatim}


\begin{Verbatim}[commandchars=\\\{\}]
{\color{outcolor}Out[{\color{outcolor}21}]:}         math score  reading score
         gender                           
         female   63.633205      72.608108
         male     68.728216      65.473029
\end{Verbatim}
            
    \begin{Verbatim}[commandchars=\\\{\}]
{\color{incolor}In [{\color{incolor}23}]:} \PY{c+c1}{\PYZsh{}as\PYZus{}index=False, с другими индексами}
         \PY{n}{students\PYZus{}perfomance}\PY{o}{.}\PY{n}{groupby}\PY{p}{(}\PY{l+s+s1}{\PYZsq{}}\PY{l+s+s1}{gender}\PY{l+s+s1}{\PYZsq{}}\PY{p}{,} \PY{n}{as\PYZus{}index}\PY{o}{=}\PY{k+kc}{False}\PY{p}{)}\PYZbs{}
             \PY{o}{.}\PY{n}{aggregate}\PY{p}{(}\PY{p}{\PYZob{}}\PY{l+s+s1}{\PYZsq{}}\PY{l+s+s1}{math score}\PY{l+s+s1}{\PYZsq{}}\PY{p}{:} \PY{l+s+s1}{\PYZsq{}}\PY{l+s+s1}{mean}\PY{l+s+s1}{\PYZsq{}}\PY{p}{,} \PY{l+s+s1}{\PYZsq{}}\PY{l+s+s1}{reading score}\PY{l+s+s1}{\PYZsq{}}\PY{p}{:} \PY{l+s+s1}{\PYZsq{}}\PY{l+s+s1}{mean}\PY{l+s+s1}{\PYZsq{}}\PY{p}{\PYZcb{}}\PY{p}{)}
\end{Verbatim}


\begin{Verbatim}[commandchars=\\\{\}]
{\color{outcolor}Out[{\color{outcolor}23}]:}    gender  math score  reading score
         0  female   63.633205      72.608108
         1    male   68.728216      65.473029
\end{Verbatim}
            
    \begin{Verbatim}[commandchars=\\\{\}]
{\color{incolor}In [{\color{incolor}24}]:} \PY{n}{students\PYZus{}perfomance}\PY{o}{.}\PY{n}{groupby}\PY{p}{(}\PY{l+s+s1}{\PYZsq{}}\PY{l+s+s1}{gender}\PY{l+s+s1}{\PYZsq{}}\PY{p}{,} \PY{n}{as\PYZus{}index}\PY{o}{=}\PY{k+kc}{False}\PY{p}{)}\PYZbs{}
             \PY{o}{.}\PY{n}{aggregate}\PY{p}{(}\PY{p}{\PYZob{}}\PY{l+s+s1}{\PYZsq{}}\PY{l+s+s1}{math score}\PY{l+s+s1}{\PYZsq{}}\PY{p}{:} \PY{l+s+s1}{\PYZsq{}}\PY{l+s+s1}{mean}\PY{l+s+s1}{\PYZsq{}}\PY{p}{,} \PY{l+s+s1}{\PYZsq{}}\PY{l+s+s1}{reading score}\PY{l+s+s1}{\PYZsq{}}\PY{p}{:} \PY{l+s+s1}{\PYZsq{}}\PY{l+s+s1}{mean}\PY{l+s+s1}{\PYZsq{}}\PY{p}{\PYZcb{}}\PY{p}{)} \PYZbs{}
             \PY{o}{.}\PY{n}{rename}\PY{p}{(}\PY{n}{columns} \PY{o}{=} \PY{p}{\PYZob{}}\PY{l+s+s1}{\PYZsq{}}\PY{l+s+s1}{math score}\PY{l+s+s1}{\PYZsq{}}\PY{p}{:} \PY{l+s+s1}{\PYZsq{}}\PY{l+s+s1}{mean\PYZus{}math\PYZus{}score}\PY{l+s+s1}{\PYZsq{}}\PY{p}{,} \PY{l+s+s1}{\PYZsq{}}\PY{l+s+s1}{reading score}\PY{l+s+s1}{\PYZsq{}}\PY{p}{:} \PY{l+s+s1}{\PYZsq{}}\PY{l+s+s1}{mean\PYZus{}reading\PYZus{}score}\PY{l+s+s1}{\PYZsq{}}\PY{p}{\PYZcb{}}\PY{p}{)}
\end{Verbatim}


\begin{Verbatim}[commandchars=\\\{\}]
{\color{outcolor}Out[{\color{outcolor}24}]:}    gender  mean\_math\_score  mean\_reading\_score
         0  female        63.633205           72.608108
         1    male        68.728216           65.473029
\end{Verbatim}
            
    \subparagraph{Tips по
комментариям}\label{tips-ux43fux43e-ux43aux43eux43cux43cux435ux43dux442ux430ux440ux438ux44fux43c}

Кстати, если подавать значения для ключей в виде списка, то можно,
во-первых, выводить несколько статистик для одной переменной, а,
во-вторых, колонки будут иметь названия в соответствии со статистикой.
Вот такой код

    \begin{Verbatim}[commandchars=\\\{\}]
{\color{incolor}In [{\color{incolor}27}]:} \PY{n}{students\PYZus{}perfomance}\PY{o}{.}\PY{n}{groupby}\PY{p}{(}\PY{l+s+s1}{\PYZsq{}}\PY{l+s+s1}{gender}\PY{l+s+s1}{\PYZsq{}}\PY{p}{,} \PY{n}{as\PYZus{}index}\PY{o}{=}\PY{k+kc}{False}\PY{p}{)}\PYZbs{}
             \PY{o}{.}\PY{n}{aggregate}\PY{p}{(}\PY{p}{\PYZob{}}\PY{l+s+s1}{\PYZsq{}}\PY{l+s+s1}{math score}\PY{l+s+s1}{\PYZsq{}}\PY{p}{:} \PY{p}{[}\PY{l+s+s1}{\PYZsq{}}\PY{l+s+s1}{mean}\PY{l+s+s1}{\PYZsq{}}\PY{p}{,} \PY{l+s+s1}{\PYZsq{}}\PY{l+s+s1}{count}\PY{l+s+s1}{\PYZsq{}}\PY{p}{,} \PY{l+s+s1}{\PYZsq{}}\PY{l+s+s1}{std}\PY{l+s+s1}{\PYZsq{}}\PY{p}{]}\PY{p}{,}\PY{l+s+s1}{\PYZsq{}}\PY{l+s+s1}{reading score}\PY{l+s+s1}{\PYZsq{}}\PY{p}{:} \PY{p}{[}\PY{l+s+s1}{\PYZsq{}}\PY{l+s+s1}{std}\PY{l+s+s1}{\PYZsq{}}\PY{p}{,} \PY{l+s+s1}{\PYZsq{}}\PY{l+s+s1}{min}\PY{l+s+s1}{\PYZsq{}}\PY{p}{,} \PY{l+s+s1}{\PYZsq{}}\PY{l+s+s1}{max}\PY{l+s+s1}{\PYZsq{}}\PY{p}{]}\PY{p}{\PYZcb{}}\PY{p}{)}
\end{Verbatim}


\begin{Verbatim}[commandchars=\\\{\}]
{\color{outcolor}Out[{\color{outcolor}27}]:}    gender math score                  reading score         
                         mean count        std           std min  max
         0  female  63.633205   518  15.491453     14.378245  17  100
         1    male  68.728216   482  14.356277     13.931832  23  100
\end{Verbatim}
            
    \begin{Verbatim}[commandchars=\\\{\}]
{\color{incolor}In [{\color{incolor}29}]:} \PY{n}{students\PYZus{}perfomance}\PY{o}{.}\PY{n}{groupby}\PY{p}{(}\PY{l+s+s1}{\PYZsq{}}\PY{l+s+s1}{gender}\PY{l+s+s1}{\PYZsq{}}\PY{p}{,} \PY{n}{as\PYZus{}index}\PY{o}{=}\PY{k+kc}{False}\PY{p}{)}\PYZbs{}
             \PY{o}{.}\PY{n}{aggregate}\PY{p}{(}\PY{p}{\PYZob{}}\PY{l+s+s1}{\PYZsq{}}\PY{l+s+s1}{math score}\PY{l+s+s1}{\PYZsq{}}\PY{p}{:} \PY{p}{[}\PY{l+s+s1}{\PYZsq{}}\PY{l+s+s1}{mean}\PY{l+s+s1}{\PYZsq{}}\PY{p}{]}\PY{p}{,}\PY{l+s+s1}{\PYZsq{}}\PY{l+s+s1}{reading score}\PY{l+s+s1}{\PYZsq{}}\PY{p}{:} \PY{p}{[}\PY{l+s+s1}{\PYZsq{}}\PY{l+s+s1}{std}\PY{l+s+s1}{\PYZsq{}}\PY{p}{,} \PY{l+s+s1}{\PYZsq{}}\PY{l+s+s1}{min}\PY{l+s+s1}{\PYZsq{}}\PY{p}{,} \PY{l+s+s1}{\PYZsq{}}\PY{l+s+s1}{max}\PY{l+s+s1}{\PYZsq{}}\PY{p}{]}\PY{p}{\PYZcb{}}\PY{p}{)}
\end{Verbatim}


\begin{Verbatim}[commandchars=\\\{\}]
{\color{outcolor}Out[{\color{outcolor}29}]:}    gender math score reading score         
                         mean           std min  max
         0  female  63.633205     14.378245  17  100
         1    male  68.728216     13.931832  23  100
\end{Verbatim}
            
    \paragraph{Группировать по нескольким
переменным}\label{ux433ux440ux443ux43fux43fux438ux440ux43eux432ux430ux442ux44c-ux43fux43e-ux43dux435ux441ux43aux43eux43bux44cux43aux438ux43c-ux43fux435ux440ux435ux43cux435ux43dux43dux44bux43c}

    \begin{Verbatim}[commandchars=\\\{\}]
{\color{incolor}In [{\color{incolor}33}]:} \PY{n}{students\PYZus{}perfomance}\PY{o}{.}\PY{n}{groupby}\PY{p}{(}\PY{p}{[}\PY{l+s+s1}{\PYZsq{}}\PY{l+s+s1}{gender}\PY{l+s+s1}{\PYZsq{}}\PY{p}{,} \PY{l+s+s1}{\PYZsq{}}\PY{l+s+s1}{lunch}\PY{l+s+s1}{\PYZsq{}}\PY{p}{]}\PY{p}{,} \PY{n}{as\PYZus{}index}\PY{o}{=}\PY{k+kc}{False}\PY{p}{)}\PYZbs{}
             \PY{o}{.}\PY{n}{aggregate}\PY{p}{(}\PY{p}{\PYZob{}}\PY{l+s+s1}{\PYZsq{}}\PY{l+s+s1}{math score}\PY{l+s+s1}{\PYZsq{}}\PY{p}{:} \PY{l+s+s1}{\PYZsq{}}\PY{l+s+s1}{mean}\PY{l+s+s1}{\PYZsq{}}\PY{p}{,} \PY{l+s+s1}{\PYZsq{}}\PY{l+s+s1}{reading score}\PY{l+s+s1}{\PYZsq{}}\PY{p}{:} \PY{l+s+s1}{\PYZsq{}}\PY{l+s+s1}{mean}\PY{l+s+s1}{\PYZsq{}}\PY{p}{\PYZcb{}}\PY{p}{)} \PYZbs{}
             \PY{o}{.}\PY{n}{rename}\PY{p}{(}\PY{n}{columns} \PY{o}{=} \PY{p}{\PYZob{}}\PY{l+s+s1}{\PYZsq{}}\PY{l+s+s1}{math score}\PY{l+s+s1}{\PYZsq{}}\PY{p}{:} \PY{l+s+s1}{\PYZsq{}}\PY{l+s+s1}{mean\PYZus{}math\PYZus{}score}\PY{l+s+s1}{\PYZsq{}}\PY{p}{,} \PY{l+s+s1}{\PYZsq{}}\PY{l+s+s1}{reading score}\PY{l+s+s1}{\PYZsq{}}\PY{p}{:} \PY{l+s+s1}{\PYZsq{}}\PY{l+s+s1}{mean\PYZus{}reading\PYZus{}score}\PY{l+s+s1}{\PYZsq{}}\PY{p}{\PYZcb{}}\PY{p}{)}
\end{Verbatim}


\begin{Verbatim}[commandchars=\\\{\}]
{\color{outcolor}Out[{\color{outcolor}33}]:}    gender         lunch  mean\_math\_score  mean\_reading\_score
         0  female  free/reduced        55.814815           67.386243
         1  female      standard        68.124620           75.607903
         2    male  free/reduced        62.457831           61.542169
         3    male      standard        72.022152           67.537975
\end{Verbatim}
            
    \begin{Verbatim}[commandchars=\\\{\}]
{\color{incolor}In [{\color{incolor}34}]:} \PY{n}{students\PYZus{}perfomance}\PY{o}{.}\PY{n}{groupby}\PY{p}{(}\PY{p}{[}\PY{l+s+s1}{\PYZsq{}}\PY{l+s+s1}{gender}\PY{l+s+s1}{\PYZsq{}}\PY{p}{,} \PY{l+s+s1}{\PYZsq{}}\PY{l+s+s1}{lunch}\PY{l+s+s1}{\PYZsq{}}\PY{p}{]}\PY{p}{)}\PYZbs{}
             \PY{o}{.}\PY{n}{aggregate}\PY{p}{(}\PY{p}{\PYZob{}}\PY{l+s+s1}{\PYZsq{}}\PY{l+s+s1}{math score}\PY{l+s+s1}{\PYZsq{}}\PY{p}{:} \PY{l+s+s1}{\PYZsq{}}\PY{l+s+s1}{mean}\PY{l+s+s1}{\PYZsq{}}\PY{p}{,} \PY{l+s+s1}{\PYZsq{}}\PY{l+s+s1}{reading score}\PY{l+s+s1}{\PYZsq{}}\PY{p}{:} \PY{l+s+s1}{\PYZsq{}}\PY{l+s+s1}{mean}\PY{l+s+s1}{\PYZsq{}}\PY{p}{\PYZcb{}}\PY{p}{)} \PYZbs{}
             \PY{o}{.}\PY{n}{rename}\PY{p}{(}\PY{n}{columns} \PY{o}{=} \PY{p}{\PYZob{}}\PY{l+s+s1}{\PYZsq{}}\PY{l+s+s1}{math score}\PY{l+s+s1}{\PYZsq{}}\PY{p}{:} \PY{l+s+s1}{\PYZsq{}}\PY{l+s+s1}{mean\PYZus{}math\PYZus{}score}\PY{l+s+s1}{\PYZsq{}}\PY{p}{,} \PY{l+s+s1}{\PYZsq{}}\PY{l+s+s1}{reading score}\PY{l+s+s1}{\PYZsq{}}\PY{p}{:} \PY{l+s+s1}{\PYZsq{}}\PY{l+s+s1}{mean\PYZus{}reading\PYZus{}score}\PY{l+s+s1}{\PYZsq{}}\PY{p}{\PYZcb{}}\PY{p}{)}
         
         \PY{c+c1}{\PYZsh{}сложно составной индекс}
\end{Verbatim}


\begin{Verbatim}[commandchars=\\\{\}]
{\color{outcolor}Out[{\color{outcolor}34}]:}                      mean\_math\_score  mean\_reading\_score
         gender lunch                                            
         female free/reduced        55.814815           67.386243
                standard            68.124620           75.607903
         male   free/reduced        62.457831           61.542169
                standard            72.022152           67.537975
\end{Verbatim}
            
    \begin{Verbatim}[commandchars=\\\{\}]
{\color{incolor}In [{\color{incolor}35}]:} \PY{n}{mean\PYZus{}scores} \PY{o}{=} \PY{n}{students\PYZus{}perfomance}\PY{o}{.}\PY{n}{groupby}\PY{p}{(}\PY{p}{[}\PY{l+s+s1}{\PYZsq{}}\PY{l+s+s1}{gender}\PY{l+s+s1}{\PYZsq{}}\PY{p}{,} \PY{l+s+s1}{\PYZsq{}}\PY{l+s+s1}{lunch}\PY{l+s+s1}{\PYZsq{}}\PY{p}{]}\PY{p}{)}\PYZbs{}
             \PY{o}{.}\PY{n}{aggregate}\PY{p}{(}\PY{p}{\PYZob{}}\PY{l+s+s1}{\PYZsq{}}\PY{l+s+s1}{math score}\PY{l+s+s1}{\PYZsq{}}\PY{p}{:} \PY{l+s+s1}{\PYZsq{}}\PY{l+s+s1}{mean}\PY{l+s+s1}{\PYZsq{}}\PY{p}{,} \PY{l+s+s1}{\PYZsq{}}\PY{l+s+s1}{reading score}\PY{l+s+s1}{\PYZsq{}}\PY{p}{:} \PY{l+s+s1}{\PYZsq{}}\PY{l+s+s1}{mean}\PY{l+s+s1}{\PYZsq{}}\PY{p}{\PYZcb{}}\PY{p}{)} \PYZbs{}
             \PY{o}{.}\PY{n}{rename}\PY{p}{(}\PY{n}{columns} \PY{o}{=} \PY{p}{\PYZob{}}\PY{l+s+s1}{\PYZsq{}}\PY{l+s+s1}{math score}\PY{l+s+s1}{\PYZsq{}}\PY{p}{:} \PY{l+s+s1}{\PYZsq{}}\PY{l+s+s1}{mean\PYZus{}math\PYZus{}score}\PY{l+s+s1}{\PYZsq{}}\PY{p}{,} \PY{l+s+s1}{\PYZsq{}}\PY{l+s+s1}{reading score}\PY{l+s+s1}{\PYZsq{}}\PY{p}{:} \PY{l+s+s1}{\PYZsq{}}\PY{l+s+s1}{mean\PYZus{}reading\PYZus{}score}\PY{l+s+s1}{\PYZsq{}}\PY{p}{\PYZcb{}}\PY{p}{)}
\end{Verbatim}


    \begin{Verbatim}[commandchars=\\\{\}]
{\color{incolor}In [{\color{incolor}36}]:} \PY{c+c1}{\PYZsh{}мульти индекс, состаящий из нескольких уровней}
         \PY{n}{mean\PYZus{}scores}\PY{o}{.}\PY{n}{index}
\end{Verbatim}


\begin{Verbatim}[commandchars=\\\{\}]
{\color{outcolor}Out[{\color{outcolor}36}]:} MultiIndex(levels=[['female', 'male'], ['free/reduced', 'standard']],
                    labels=[[0, 0, 1, 1], [0, 1, 0, 1]],
                    names=['gender', 'lunch'])
\end{Verbatim}
            
    \begin{Verbatim}[commandchars=\\\{\}]
{\color{incolor}In [{\color{incolor}39}]:} \PY{n}{mean\PYZus{}scores}\PY{o}{.}\PY{n}{loc}\PY{p}{[}\PY{p}{[}\PY{p}{(}\PY{l+s+s1}{\PYZsq{}}\PY{l+s+s1}{female}\PY{l+s+s1}{\PYZsq{}}\PY{p}{,} \PY{l+s+s1}{\PYZsq{}}\PY{l+s+s1}{standard}\PY{l+s+s1}{\PYZsq{}}\PY{p}{)}\PY{p}{]}\PY{p}{]}
\end{Verbatim}


\begin{Verbatim}[commandchars=\\\{\}]
{\color{outcolor}Out[{\color{outcolor}39}]:}                  mean\_math\_score  mean\_reading\_score
         gender lunch                                        
         female standard         68.12462           75.607903
\end{Verbatim}
            
    Преимущество мульти индекса

    \begin{Verbatim}[commandchars=\\\{\}]
{\color{incolor}In [{\color{incolor}47}]:} \PY{c+c1}{\PYZsh{}сгруппировать по 2ув переменным}
         \PY{n}{students\PYZus{}perfomance}\PY{o}{.}\PY{n}{groupby}\PY{p}{(}\PY{p}{[}\PY{l+s+s1}{\PYZsq{}}\PY{l+s+s1}{gender}\PY{l+s+s1}{\PYZsq{}}\PY{p}{,} \PY{l+s+s1}{\PYZsq{}}\PY{l+s+s1}{race/ethnicity}\PY{l+s+s1}{\PYZsq{}}\PY{p}{]}\PY{p}{)}\PY{p}{[}\PY{l+s+s1}{\PYZsq{}}\PY{l+s+s1}{math score}\PY{l+s+s1}{\PYZsq{}}\PY{p}{]}\PY{o}{.}\PY{n}{unique}\PY{p}{(}\PY{p}{)}
\end{Verbatim}


\begin{Verbatim}[commandchars=\\\{\}]
{\color{outcolor}Out[{\color{outcolor}47}]:} gender  race/ethnicity
         female  group A           [50, 55, 41, 58, 51, 44, 71, 38, 49, 59, 47, 7{\ldots}
                 group B           [72, 90, 71, 88, 38, 65, 18, 63, 53, 47, 79, 5{\ldots}
                 group C           [69, 54, 67, 58, 66, 71, 33, 0, 60, 39, 73, 76{\ldots}
                 group D           [62, 69, 74, 50, 75, 57, 59, 58, 61, 71, 73, 6{\ldots}
                 group E           [56, 50, 82, 62, 63, 99, 42, 66, 75, 81, 83, 4{\ldots}
         male    group A           [47, 78, 73, 39, 62, 80, 50, 54, 57, 77, 72, 6{\ldots}
                 group B           [40, 69, 57, 59, 65, 67, 61, 44, 49, 79, 60, 9{\ldots}
                 group C           [76, 58, 88, 46, 70, 55, 82, 53, 61, 49, 27, 7{\ldots}
                 group D           [64, 40, 66, 44, 74, 88, 52, 58, 45, 63, 42, 6{\ldots}
                 group E           [97, 81, 53, 77, 79, 30, 72, 66, 83, 87, 70, 1{\ldots}
         Name: math score, dtype: object
\end{Verbatim}
            
    \begin{Verbatim}[commandchars=\\\{\}]
{\color{incolor}In [{\color{incolor}49}]:} \PY{c+c1}{\PYZsh{}получим топ 5 студентов по математики (male/female)}
         
         \PY{n}{students\PYZus{}perfomance}\PY{o}{.}\PY{n}{sort\PYZus{}values}\PY{p}{(}\PY{p}{[}\PY{l+s+s1}{\PYZsq{}}\PY{l+s+s1}{gender}\PY{l+s+s1}{\PYZsq{}}\PY{p}{,} \PY{l+s+s1}{\PYZsq{}}\PY{l+s+s1}{math score}\PY{l+s+s1}{\PYZsq{}}\PY{p}{]}\PY{p}{,} \PY{n}{ascending}\PY{o}{=}\PY{k+kc}{False}\PY{p}{)} \PY{c+c1}{\PYZsh{}сортировка}
         
         \PY{n}{students\PYZus{}perfomance}\PY{o}{.}\PY{n}{sort\PYZus{}values}\PY{p}{(}\PY{p}{[}\PY{l+s+s1}{\PYZsq{}}\PY{l+s+s1}{gender}\PY{l+s+s1}{\PYZsq{}}\PY{p}{,} \PY{l+s+s1}{\PYZsq{}}\PY{l+s+s1}{math score}\PY{l+s+s1}{\PYZsq{}}\PY{p}{]}\PY{p}{,} \PY{n}{ascending}\PY{o}{=}\PY{k+kc}{False}\PY{p}{)} \PYZbs{}
             \PY{o}{.}\PY{n}{groupby}\PY{p}{(}\PY{l+s+s1}{\PYZsq{}}\PY{l+s+s1}{gender}\PY{l+s+s1}{\PYZsq{}}\PY{p}{)}\PY{o}{.}\PY{n}{head}\PY{p}{(}\PY{p}{)}
\end{Verbatim}


\begin{Verbatim}[commandchars=\\\{\}]
{\color{outcolor}Out[{\color{outcolor}49}]:}      gender race/ethnicity parental level of education         lunch  \textbackslash{}
         149    male        group E          associate's degree  free/reduced   
         623    male        group A                some college      standard   
         625    male        group D                some college      standard   
         916    male        group E           bachelor's degree      standard   
         306    male        group E                some college      standard   
         451  female        group E                some college      standard   
         458  female        group E           bachelor's degree      standard   
         962  female        group E          associate's degree      standard   
         114  female        group E           bachelor's degree      standard   
         263  female        group E                 high school      standard   
         
             test preparation course  math score  reading score  writing score  
         149               completed         100            100             93  
         623               completed         100             96             86  
         625               completed         100             97             99  
         916               completed         100            100            100  
         306               completed          99             87             81  
         451                    none         100             92             97  
         458                    none         100            100            100  
         962                    none         100            100            100  
         114               completed          99            100            100  
         263                    none          99             93             90  
\end{Verbatim}
            
    \paragraph{Tips с
комментов}\label{tips-ux441-ux43aux43eux43cux43cux435ux43dux442ux43eux432}

Если хочется отсортировать, например, в порядке возрастания по одной
переменной, и в порядке убывания по другой, то достаточно лишь передать
в аргумент ascending соответствующий лист:

    \begin{Verbatim}[commandchars=\\\{\}]
{\color{incolor}In [{\color{incolor}52}]:} \PY{n}{students\PYZus{}perfomance}\PY{o}{.}\PY{n}{sort\PYZus{}values}\PY{p}{(}\PY{p}{[}\PY{l+s+s1}{\PYZsq{}}\PY{l+s+s1}{reading score}\PY{l+s+s1}{\PYZsq{}}\PY{p}{,} \PY{l+s+s1}{\PYZsq{}}\PY{l+s+s1}{math score}\PY{l+s+s1}{\PYZsq{}}\PY{p}{]}\PY{p}{,} \PY{n}{ascending}\PY{o}{=}\PY{p}{[}\PY{k+kc}{True}\PY{p}{,} \PY{k+kc}{False}\PY{p}{]}\PY{p}{)}\PY{o}{.}\PY{n}{head}\PY{p}{(}\PY{p}{)}
\end{Verbatim}


\begin{Verbatim}[commandchars=\\\{\}]
{\color{outcolor}Out[{\color{outcolor}52}]:}      gender race/ethnicity parental level of education         lunch  \textbackslash{}
         59   female        group C            some high school  free/reduced   
         327    male        group A                some college  free/reduced   
         596    male        group B                 high school  free/reduced   
         980  female        group B                 high school  free/reduced   
         76     male        group E            some high school      standard   
         
             test preparation course  math score  reading score  writing score  
         59                     none           0             17             10  
         327                    none          28             23             19  
         596                    none          30             24             15  
         980                    none           8             24             23  
         76                     none          30             26             22  
\end{Verbatim}
            
    \paragraph{Как создавать колонки в
Pandas}\label{ux43aux430ux43a-ux441ux43eux437ux434ux430ux432ux430ux442ux44c-ux43aux43eux43bux43eux43dux43aux438-ux432-pandas}

Есть несколько способов

    \begin{Verbatim}[commandchars=\\\{\}]
{\color{incolor}In [{\color{incolor}54}]:} \PY{n}{students\PYZus{}perfomance}\PY{p}{[}\PY{l+s+s1}{\PYZsq{}}\PY{l+s+s1}{total\PYZus{}score}\PY{l+s+s1}{\PYZsq{}}\PY{p}{]} \PY{o}{=} \PY{n}{students\PYZus{}perfomance}\PY{p}{[}\PY{l+s+s1}{\PYZsq{}}\PY{l+s+s1}{math score}\PY{l+s+s1}{\PYZsq{}}\PY{p}{]} \PY{o}{+} \PY{n}{students\PYZus{}perfomance}\PY{p}{[}\PY{l+s+s1}{\PYZsq{}}\PY{l+s+s1}{reading score}\PY{l+s+s1}{\PYZsq{}}\PY{p}{]} \PY{o}{+} \PY{n}{students\PYZus{}perfomance}\PY{p}{[}\PY{l+s+s1}{\PYZsq{}}\PY{l+s+s1}{writing score}\PY{l+s+s1}{\PYZsq{}}\PY{p}{]}
\end{Verbatim}


    \begin{Verbatim}[commandchars=\\\{\}]
{\color{incolor}In [{\color{incolor}55}]:} \PY{n}{students\PYZus{}perfomance}\PY{o}{.}\PY{n}{head}\PY{p}{(}\PY{p}{)}
\end{Verbatim}


\begin{Verbatim}[commandchars=\\\{\}]
{\color{outcolor}Out[{\color{outcolor}55}]:}    gender race/ethnicity parental level of education         lunch  \textbackslash{}
         0  female        group B           bachelor's degree      standard   
         1  female        group C                some college      standard   
         2  female        group B             master's degree      standard   
         3    male        group A          associate's degree  free/reduced   
         4    male        group C                some college      standard   
         
           test preparation course  math score  reading score  writing score  \textbackslash{}
         0                    none          72             72             74   
         1               completed          69             90             88   
         2                    none          90             95             93   
         3                    none          47             57             44   
         4                    none          76             78             75   
         
            total\_score  
         0          218  
         1          247  
         2          278  
         3          148  
         4          229  
\end{Verbatim}
            
    \begin{Verbatim}[commandchars=\\\{\}]
{\color{incolor}In [{\color{incolor}57}]:} \PY{c+c1}{\PYZsh{}еще один способ, сделать несколько колонок}
         \PY{n}{students\PYZus{}perfomance} \PY{o}{=} \PY{n}{students\PYZus{}perfomance}\PY{o}{.}\PY{n}{assign}\PY{p}{(}\PY{n}{total\PYZus{}score\PYZus{}log} \PY{o}{=} \PY{n}{np}\PY{o}{.}\PY{n}{log}\PY{p}{(}\PY{n}{students\PYZus{}perfomance}\PY{o}{.}\PY{n}{total\PYZus{}score}\PY{p}{)}\PY{p}{)}
         \PY{n}{students\PYZus{}perfomance}\PY{o}{.}\PY{n}{head}\PY{p}{(}\PY{p}{)}
\end{Verbatim}


\begin{Verbatim}[commandchars=\\\{\}]
{\color{outcolor}Out[{\color{outcolor}57}]:}    gender race/ethnicity parental level of education         lunch  \textbackslash{}
         0  female        group B           bachelor's degree      standard   
         1  female        group C                some college      standard   
         2  female        group B             master's degree      standard   
         3    male        group A          associate's degree  free/reduced   
         4    male        group C                some college      standard   
         
           test preparation course  math score  reading score  writing score  \textbackslash{}
         0                    none          72             72             74   
         1               completed          69             90             88   
         2                    none          90             95             93   
         3                    none          47             57             44   
         4                    none          76             78             75   
         
            total\_score  total\_score\_log  
         0          218         5.384495  
         1          247         5.509388  
         2          278         5.627621  
         3          148         4.997212  
         4          229         5.433722  
\end{Verbatim}
            
    \begin{Verbatim}[commandchars=\\\{\}]
{\color{incolor}In [{\color{incolor}60}]:} \PY{c+c1}{\PYZsh{}удалим колонки или строки}
         \PY{n}{students\PYZus{}perfomance}\PY{o}{.}\PY{n}{drop}\PY{p}{(}\PY{p}{[}\PY{l+s+s1}{\PYZsq{}}\PY{l+s+s1}{total\PYZus{}score}\PY{l+s+s1}{\PYZsq{}}\PY{p}{,} \PY{l+s+s1}{\PYZsq{}}\PY{l+s+s1}{lunch}\PY{l+s+s1}{\PYZsq{}}\PY{p}{]}\PY{p}{,} \PY{n}{axis}\PY{o}{=}\PY{l+m+mi}{1}\PY{p}{)}\PY{o}{.}\PY{n}{head}\PY{p}{(}\PY{p}{)}\PY{c+c1}{\PYZsh{}asix=1 \PYZhy{} по колонкам, axis=0 \PYZhy{} по строкам}
\end{Verbatim}


\begin{Verbatim}[commandchars=\\\{\}]
{\color{outcolor}Out[{\color{outcolor}60}]:}    gender race/ethnicity parental level of education test preparation course  \textbackslash{}
         0  female        group B           bachelor's degree                    none   
         1  female        group C                some college               completed   
         2  female        group B             master's degree                    none   
         3    male        group A          associate's degree                    none   
         4    male        group C                some college                    none   
         
            math score  reading score  writing score  total\_score\_log  
         0          72             72             74         5.384495  
         1          69             90             88         5.509388  
         2          90             95             93         5.627621  
         3          47             57             44         4.997212  
         4          76             78             75         5.433722  
\end{Verbatim}
            
    \subsubsection{Задача из модуля (шаг -
5)}\label{ux437ux430ux434ux430ux447ux430-ux438ux437-ux43cux43eux434ux443ux43bux44f-ux448ux430ux433---5}

Пересчитаем число ног у героев игры Dota2! Сгруппируйте героев из
\href{https://stepik.org/media/attachments/course/4852/dota_hero_stats.csv}{датасэта}
по числу их ног (колонка \emph{legs}), и заполните их число в задании
ниже.

Данные взяты \href{https://api.opendota.com/api/heroes}{отсюда}, на этом
же \href{https://www.opendota.com/}{сайте} можно найти больше
разнообразных данных по Dota2.

    \begin{Verbatim}[commandchars=\\\{\}]
{\color{incolor}In [{\color{incolor}61}]:} \PY{k+kn}{import} \PY{n+nn}{pandas} \PY{k}{as} \PY{n+nn}{pd}
         \PY{k+kn}{import} \PY{n+nn}{numpy} \PY{k}{as} \PY{n+nn}{np}
         
         \PY{n}{df} \PY{o}{=} \PY{n}{pd}\PY{o}{.}\PY{n}{read\PYZus{}csv}\PY{p}{(}\PY{l+s+s1}{\PYZsq{}}\PY{l+s+s1}{https://stepik.org/media/attachments/course/4852/dota\PYZus{}hero\PYZus{}stats.csv}\PY{l+s+s1}{\PYZsq{}}\PY{p}{)}
\end{Verbatim}


    \begin{Verbatim}[commandchars=\\\{\}]
{\color{incolor}In [{\color{incolor}62}]:} \PY{n}{df}\PY{o}{.}\PY{n}{head}\PY{p}{(}\PY{p}{)}
\end{Verbatim}


\begin{Verbatim}[commandchars=\\\{\}]
{\color{outcolor}Out[{\color{outcolor}62}]:}    Unnamed: 0 attack\_type  id  legs  localized\_name  \textbackslash{}
         0           0       Melee   1     2       Anti-Mage   
         1           1       Melee   2     2             Axe   
         2           2      Ranged   3     4            Bane   
         3           3       Melee   4     2     Bloodseeker   
         4           4      Ranged   5     2  Crystal Maiden   
         
                                    name primary\_attr  \textbackslash{}
         0        npc\_dota\_hero\_antimage          agi   
         1             npc\_dota\_hero\_axe          str   
         2            npc\_dota\_hero\_bane          int   
         3     npc\_dota\_hero\_bloodseeker          agi   
         4  npc\_dota\_hero\_crystal\_maiden          int   
         
                                                        roles  
         0                       ['Carry', 'Escape', 'Nuker']  
         1    ['Initiator', 'Durable', 'Disabler', 'Jungler']  
         2        ['Support', 'Disabler', 'Nuker', 'Durable']  
         3  ['Carry', 'Disabler', 'Jungler', 'Nuker', 'Ini{\ldots}  
         4        ['Support', 'Disabler', 'Nuker', 'Jungler']  
\end{Verbatim}
            
    \begin{Verbatim}[commandchars=\\\{\}]
{\color{incolor}In [{\color{incolor}76}]:} \PY{n}{df}\PY{o}{.}\PY{n}{groupby}\PY{p}{(}\PY{l+s+s1}{\PYZsq{}}\PY{l+s+s1}{legs}\PY{l+s+s1}{\PYZsq{}}\PY{p}{,} \PY{n}{as\PYZus{}index}\PY{o}{=}\PY{k+kc}{False}\PY{p}{)}\PY{o}{.}\PY{n}{count}\PY{p}{(}\PY{p}{)}
\end{Verbatim}


\begin{Verbatim}[commandchars=\\\{\}]
{\color{outcolor}Out[{\color{outcolor}76}]:}    legs  Unnamed: 0  attack\_type  id  localized\_name  name  primary\_attr  \textbackslash{}
         0     0          11           11  11              11    11            11   
         1     2          95           95  95              95    95            95   
         2     4           7            7   7               7     7             7   
         3     6           3            3   3               3     3             3   
         4     8           1            1   1               1     1             1   
         
            roles  
         0     11  
         1     95  
         2      7  
         3      3  
         4      1  
\end{Verbatim}
            
    \subsubsection{Задача модуля (шаг -
6)}\label{ux437ux430ux434ux430ux447ux430-ux43cux43eux434ux443ux43bux44f-ux448ux430ux433---6}

К нам поступили
\href{https://stepik.org/media/attachments/course/4852/accountancy.csv}{данные}
из бухгалтерии о заработках Лупы и Пупы за разные задачи! Посмотрите у
кого из них больше средний заработок в различных категориях (колонка
\textbf{Type}) и заполните таблицу, указывая исполнителя с большим
заработком в каждой из категорий.

    \begin{Verbatim}[commandchars=\\\{\}]
{\color{incolor}In [{\color{incolor}77}]:} \PY{k+kn}{import} \PY{n+nn}{pandas} \PY{k}{as} \PY{n+nn}{pd}
         \PY{k+kn}{import} \PY{n+nn}{numpy} \PY{k}{as} \PY{n+nn}{np}
         
         \PY{n}{df} \PY{o}{=} \PY{n}{pd}\PY{o}{.}\PY{n}{read\PYZus{}csv}\PY{p}{(}\PY{l+s+s1}{\PYZsq{}}\PY{l+s+s1}{https://stepik.org/media/attachments/course/4852/accountancy.csv}\PY{l+s+s1}{\PYZsq{}}\PY{p}{)}
\end{Verbatim}


    \begin{Verbatim}[commandchars=\\\{\}]
{\color{incolor}In [{\color{incolor}78}]:} \PY{n}{df}\PY{o}{.}\PY{n}{head}\PY{p}{(}\PY{p}{)}
\end{Verbatim}


\begin{Verbatim}[commandchars=\\\{\}]
{\color{outcolor}Out[{\color{outcolor}78}]:}    Unnamed: 0 Executor Type  Salary
         0           0     Pupa    D      63
         1           1     Pupa    A     158
         2           2     Pupa    D     194
         3           3     Pupa    E     109
         4           4    Loopa    E     184
\end{Verbatim}
            
    \begin{Verbatim}[commandchars=\\\{\}]
{\color{incolor}In [{\color{incolor}82}]:} \PY{n}{df}\PY{o}{.}\PY{n}{groupby}\PY{p}{(}\PY{p}{[}\PY{l+s+s1}{\PYZsq{}}\PY{l+s+s1}{Executor}\PY{l+s+s1}{\PYZsq{}}\PY{p}{,}\PY{l+s+s1}{\PYZsq{}}\PY{l+s+s1}{Type}\PY{l+s+s1}{\PYZsq{}}\PY{p}{]}\PY{p}{)}\PY{o}{.}\PY{n}{aggregate}\PY{p}{(}\PY{p}{\PYZob{}}\PY{l+s+s1}{\PYZsq{}}\PY{l+s+s1}{Salary}\PY{l+s+s1}{\PYZsq{}}\PY{p}{:} \PY{l+s+s1}{\PYZsq{}}\PY{l+s+s1}{mean}\PY{l+s+s1}{\PYZsq{}}\PY{p}{\PYZcb{}}\PY{p}{)}
\end{Verbatim}


\begin{Verbatim}[commandchars=\\\{\}]
{\color{outcolor}Out[{\color{outcolor}82}]:}                    Salary
         Executor Type            
         Loopa    A      58.000000
                  B     145.166667
                  C     154.333333
                  D     137.714286
                  E     164.000000
                  F     238.000000
         Pupa     A     160.833333
                  B      77.000000
                  C      74.500000
                  D     146.500000
                  E     131.200000
                  F     136.250000
\end{Verbatim}
            
    \subsubsection{Задача из модуля (шаг -
7)}\label{ux437ux430ux434ux430ux447ux430-ux438ux437-ux43cux43eux434ux443ux43bux44f-ux448ux430ux433---7}

\textgreater{} Пересчитаем число ног у героев игры Dota2! Сгруппируйте
героев из
\href{https://stepik.org/media/attachments/course/4852/dota_hero_stats.csv}{датасэта}
по числу их ног (колонка \emph{legs}), и заполните их число в задании
ниже.

\textgreater{} Данные взяты
\href{https://api.opendota.com/api/heroes}{отсюда}, на этом же
\href{https://www.opendota.com/}{сайте} можно найти больше разнообразных
данных по Dota2.

Продолжим исследование героев Dota2. Сгруппируйте по колонкам
\textbf{attack\_type} и \textbf{primary\_attr} и выберите самый
распространённый набор характеристик.

    \begin{Verbatim}[commandchars=\\\{\}]
{\color{incolor}In [{\color{incolor}83}]:} \PY{k+kn}{import} \PY{n+nn}{pandas} \PY{k}{as} \PY{n+nn}{pd}
         \PY{k+kn}{import} \PY{n+nn}{numpy} \PY{k}{as} \PY{n+nn}{np}
         
         \PY{n}{df} \PY{o}{=} \PY{n}{pd}\PY{o}{.}\PY{n}{read\PYZus{}csv}\PY{p}{(}\PY{l+s+s1}{\PYZsq{}}\PY{l+s+s1}{https://stepik.org/media/attachments/course/4852/dota\PYZus{}hero\PYZus{}stats.csv}\PY{l+s+s1}{\PYZsq{}}\PY{p}{)}
\end{Verbatim}


    \begin{Verbatim}[commandchars=\\\{\}]
{\color{incolor}In [{\color{incolor}84}]:} \PY{n}{df}\PY{o}{.}\PY{n}{head}\PY{p}{(}\PY{p}{)}
\end{Verbatim}


\begin{Verbatim}[commandchars=\\\{\}]
{\color{outcolor}Out[{\color{outcolor}84}]:}    Unnamed: 0 attack\_type  id  legs  localized\_name  \textbackslash{}
         0           0       Melee   1     2       Anti-Mage   
         1           1       Melee   2     2             Axe   
         2           2      Ranged   3     4            Bane   
         3           3       Melee   4     2     Bloodseeker   
         4           4      Ranged   5     2  Crystal Maiden   
         
                                    name primary\_attr  \textbackslash{}
         0        npc\_dota\_hero\_antimage          agi   
         1             npc\_dota\_hero\_axe          str   
         2            npc\_dota\_hero\_bane          int   
         3     npc\_dota\_hero\_bloodseeker          agi   
         4  npc\_dota\_hero\_crystal\_maiden          int   
         
                                                        roles  
         0                       ['Carry', 'Escape', 'Nuker']  
         1    ['Initiator', 'Durable', 'Disabler', 'Jungler']  
         2        ['Support', 'Disabler', 'Nuker', 'Durable']  
         3  ['Carry', 'Disabler', 'Jungler', 'Nuker', 'Ini{\ldots}  
         4        ['Support', 'Disabler', 'Nuker', 'Jungler']  
\end{Verbatim}
            
    \begin{Verbatim}[commandchars=\\\{\}]
{\color{incolor}In [{\color{incolor}86}]:} \PY{n}{df}\PY{o}{.}\PY{n}{groupby}\PY{p}{(}\PY{p}{[}\PY{l+s+s1}{\PYZsq{}}\PY{l+s+s1}{attack\PYZus{}type}\PY{l+s+s1}{\PYZsq{}}\PY{p}{,} \PY{l+s+s1}{\PYZsq{}}\PY{l+s+s1}{primary\PYZus{}attr}\PY{l+s+s1}{\PYZsq{}}\PY{p}{]}\PY{p}{)}\PY{o}{.}\PY{n}{count}\PY{p}{(}\PY{p}{)}
\end{Verbatim}


\begin{Verbatim}[commandchars=\\\{\}]
{\color{outcolor}Out[{\color{outcolor}86}]:}                           Unnamed: 0  id  legs  localized\_name  name  roles
         attack\_type primary\_attr                                                   
         Melee       agi                   19  19    19              19    19     19
                     int                    2   2     2               2     2      2
                     str                   35  35    35              35    35     35
         Ranged      agi                   18  18    18              18    18     18
                     int                   40  40    40              40    40     40
                     str                    3   3     3               3     3      3
\end{Verbatim}
            
    \subsubsection{Задача из модуля (шаг -
8)}\label{ux437ux430ux434ux430ux447ux430-ux438ux437-ux43cux43eux434ux443ux43bux44f-ux448ux430ux433---8}

Аспирант Ростислав изучает метаболом водорослей и получил такую
\href{http://stepik.org/media/attachments/course/4852/algae.csv}{табличку}.
В ней он записал вид каждой водоросли, её род (группа, объединяющая
близкие виды), группа (ещё одно объединение водорослей в крупные
фракции) и концентрации анализируемых веществ.

Помогите Ростиславу найти среднюю концентрацию каждого из веществ в
каждом из родов (колонка \textbf{genus})! Для этого проведите
группировку датафрэйма, сохранённого в переменной
\textbf{concentrations}, и примените метод, сохранив результат в
переменной \textbf{mean\_concentrations}.

    \begin{Verbatim}[commandchars=\\\{\}]
{\color{incolor}In [{\color{incolor}87}]:} \PY{k+kn}{import} \PY{n+nn}{pandas} \PY{k}{as} \PY{n+nn}{pd}
         
         \PY{n}{df} \PY{o}{=} \PY{n}{pd}\PY{o}{.}\PY{n}{read\PYZus{}csv}\PY{p}{(}\PY{l+s+s1}{\PYZsq{}}\PY{l+s+s1}{http://stepik.org/media/attachments/course/4852/algae.csv}\PY{l+s+s1}{\PYZsq{}}\PY{p}{)}
\end{Verbatim}


    \begin{Verbatim}[commandchars=\\\{\}]
{\color{incolor}In [{\color{incolor}88}]:} \PY{n}{df}\PY{o}{.}\PY{n}{head}\PY{p}{(}\PY{p}{)}
\end{Verbatim}


\begin{Verbatim}[commandchars=\\\{\}]
{\color{outcolor}Out[{\color{outcolor}88}]:}                species       genus  group   sucrose    alanin   citrate  \textbackslash{}
         0    Fucus\_vesiculosus       Fucus  brown  3.001472  3.711498  5.004262   
         1  Saccharina\_japonica  Saccharina  brown  6.731070  1.255251  5.621499   
         2       Fucus\_serratus       Fucus  brown  3.276870  0.346431  1.216767   
         3      Fucus\_distichus       Fucus  brown  6.786996  6.641303  6.423606   
         4    Cladophora\_fracta  Cladophora  green  3.861470  1.648450  6.940588   
         
             glucose  oleic\_acid  
         0  2.548459    6.405165  
         1  6.013219    4.156700  
         2  3.623225    0.304573  
         3  2.272724    3.393203  
         4  2.316955    2.528886  
\end{Verbatim}
            
    \begin{Verbatim}[commandchars=\\\{\}]
{\color{incolor}In [{\color{incolor}89}]:} \PY{n}{df}\PY{o}{.}\PY{n}{groupby}\PY{p}{(}\PY{p}{[}\PY{l+s+s1}{\PYZsq{}}\PY{l+s+s1}{species}\PY{l+s+s1}{\PYZsq{}}\PY{p}{,} \PY{l+s+s1}{\PYZsq{}}\PY{l+s+s1}{genus}\PY{l+s+s1}{\PYZsq{}}\PY{p}{]}\PY{p}{)}\PY{o}{.}\PY{n}{mean}\PY{p}{(}\PY{p}{)}
\end{Verbatim}


\begin{Verbatim}[commandchars=\\\{\}]
{\color{outcolor}Out[{\color{outcolor}89}]:}                                    sucrose    alanin   citrate   glucose  \textbackslash{}
         species              genus                                                 
         Ascophyllum\_nodosum  Ascophyllum  6.825467  0.875429  5.253527  3.414961   
         Cladophora\_compacta  Cladophora   5.712284  3.461692  3.082826  3.343707   
         Cladophora\_fracta    Cladophora   3.861470  1.648450  6.940588  2.316955   
         Cladophora\_gracilis  Cladophora   2.452623  6.881024  5.841520  2.740165   
         Fucus\_distichus      Fucus        6.786996  6.641303  6.423606  2.272724   
         Fucus\_serratus       Fucus        3.276870  0.346431  1.216767  3.623225   
         Fucus\_vesiculosus    Fucus        3.001472  3.711498  5.004262  2.548459   
         Palmaria\_palmata     Palmaria     0.704580  3.176440  5.573905  3.242090   
         Saccharina\_japonica  Saccharina   6.731070  1.255251  5.621499  6.013219   
         Saccharina\_latissima Saccharina   1.636122  5.793163  1.071920  3.947968   
         
                                           oleic\_acid  
         species              genus                    
         Ascophyllum\_nodosum  Ascophyllum    2.432526  
         Cladophora\_compacta  Cladophora     1.432514  
         Cladophora\_fracta    Cladophora     2.528886  
         Cladophora\_gracilis  Cladophora     2.829016  
         Fucus\_distichus      Fucus          3.393203  
         Fucus\_serratus       Fucus          0.304573  
         Fucus\_vesiculosus    Fucus          6.405165  
         Palmaria\_palmata     Palmaria       2.245538  
         Saccharina\_japonica  Saccharina     4.156700  
         Saccharina\_latissima Saccharina     4.817804  
\end{Verbatim}
            
    \subsubsection{Задача из модуля (шаг -
9)}\label{ux437ux430ux434ux430ux447ux430-ux438ux437-ux43cux43eux434ux443ux43bux44f-ux448ux430ux433---9}

\textgreater{}Аспирант Ростислав изучает метаболом водорослей и получил
такую
\href{http://stepik.org/media/attachments/course/4852/algae.csv}{табличку}.
В ней он записал вид каждой водоросли, её род (группа, объединяющая
близкие виды), группа (ещё одно объединение водорослей в крупные
фракции) и концентрации анализируемых веществ.

\textgreater{}Помогите Ростиславу найти среднюю концентрацию каждого из
веществ в каждом из родов (колонка \textbf{genus})! Для этого проведите
группировку датафрэйма, сохранённого в переменной
\textbf{concentrations}, и примените метод, сохранив результат в
переменной \textbf{mean\_concentrations}.

Пользуясь предыдущими
\href{http://stepik.org/media/attachments/course/4852/algae.csv}{данными},
укажите через пробел (без запятых) чему равны минимальная, средняя и
максимальная концентрации аланина (\textbf{alanin}) среди видов рода
Fucus. Округлите до 2-ого знака, десятичным разделителем является точка.

Формат ответа: 0.55 6.77 7.48

    \begin{Verbatim}[commandchars=\\\{\}]
{\color{incolor}In [{\color{incolor}92}]:} \PY{k+kn}{import} \PY{n+nn}{pandas} \PY{k}{as} \PY{n+nn}{pd}
         
         \PY{n}{df} \PY{o}{=} \PY{n}{pd}\PY{o}{.}\PY{n}{read\PYZus{}csv}\PY{p}{(}\PY{l+s+s1}{\PYZsq{}}\PY{l+s+s1}{http://stepik.org/media/attachments/course/4852/algae.csv}\PY{l+s+s1}{\PYZsq{}}\PY{p}{)}
\end{Verbatim}


    \begin{Verbatim}[commandchars=\\\{\}]
{\color{incolor}In [{\color{incolor}93}]:} \PY{n}{df}\PY{o}{.}\PY{n}{head}\PY{p}{(}\PY{p}{)}
\end{Verbatim}


\begin{Verbatim}[commandchars=\\\{\}]
{\color{outcolor}Out[{\color{outcolor}93}]:}                species       genus  group   sucrose    alanin   citrate  \textbackslash{}
         0    Fucus\_vesiculosus       Fucus  brown  3.001472  3.711498  5.004262   
         1  Saccharina\_japonica  Saccharina  brown  6.731070  1.255251  5.621499   
         2       Fucus\_serratus       Fucus  brown  3.276870  0.346431  1.216767   
         3      Fucus\_distichus       Fucus  brown  6.786996  6.641303  6.423606   
         4    Cladophora\_fracta  Cladophora  green  3.861470  1.648450  6.940588   
         
             glucose  oleic\_acid  
         0  2.548459    6.405165  
         1  6.013219    4.156700  
         2  3.623225    0.304573  
         3  2.272724    3.393203  
         4  2.316955    2.528886  
\end{Verbatim}
            
    \begin{Verbatim}[commandchars=\\\{\}]
{\color{incolor}In [{\color{incolor}95}]:} \PY{n}{df}\PY{o}{.}\PY{n}{groupby}\PY{p}{(}\PY{l+s+s1}{\PYZsq{}}\PY{l+s+s1}{genus}\PY{l+s+s1}{\PYZsq{}}\PY{p}{)}\PY{o}{.}\PY{n}{agg}\PY{p}{(}\PY{p}{\PYZob{}}\PY{l+s+s1}{\PYZsq{}}\PY{l+s+s1}{alanin}\PY{l+s+s1}{\PYZsq{}}\PY{p}{:} \PY{p}{[}\PY{l+s+s1}{\PYZsq{}}\PY{l+s+s1}{min}\PY{l+s+s1}{\PYZsq{}}\PY{p}{,} \PY{l+s+s1}{\PYZsq{}}\PY{l+s+s1}{mean}\PY{l+s+s1}{\PYZsq{}}\PY{p}{,} \PY{l+s+s1}{\PYZsq{}}\PY{l+s+s1}{max}\PY{l+s+s1}{\PYZsq{}}\PY{p}{]}\PY{p}{\PYZcb{}}\PY{p}{)}
\end{Verbatim}


\begin{Verbatim}[commandchars=\\\{\}]
{\color{outcolor}Out[{\color{outcolor}95}]:}                alanin                    
                           min      mean       max
         genus                                    
         Ascophyllum  0.875429  0.875429  0.875429
         Cladophora   1.648450  3.997055  6.881024
         Fucus        0.346431  3.566411  6.641303
         Palmaria     3.176440  3.176440  3.176440
         Saccharina   1.255251  3.524207  5.793163
\end{Verbatim}
            
    \begin{Verbatim}[commandchars=\\\{\}]
{\color{incolor}In [{\color{incolor}96}]:} \PY{n}{df}\PY{o}{.}\PY{n}{groupby}\PY{p}{(}\PY{l+s+s1}{\PYZsq{}}\PY{l+s+s1}{genus}\PY{l+s+s1}{\PYZsq{}}\PY{p}{)}\PY{o}{.}\PY{n}{agg}\PY{p}{(}\PY{p}{[}\PY{l+s+s1}{\PYZsq{}}\PY{l+s+s1}{min}\PY{l+s+s1}{\PYZsq{}}\PY{p}{,} \PY{l+s+s1}{\PYZsq{}}\PY{l+s+s1}{mean}\PY{l+s+s1}{\PYZsq{}}\PY{p}{,} \PY{l+s+s1}{\PYZsq{}}\PY{l+s+s1}{max}\PY{l+s+s1}{\PYZsq{}}\PY{p}{]}\PY{p}{)}\PY{o}{.}\PY{n}{loc}\PY{p}{[}\PY{l+s+s1}{\PYZsq{}}\PY{l+s+s1}{Fucus}\PY{l+s+s1}{\PYZsq{}}\PY{p}{,} \PY{l+s+s1}{\PYZsq{}}\PY{l+s+s1}{alanin}\PY{l+s+s1}{\PYZsq{}}\PY{p}{]}\PY{o}{.}\PY{n}{round}\PY{p}{(}\PY{l+m+mi}{2}\PY{p}{)}
\end{Verbatim}


\begin{Verbatim}[commandchars=\\\{\}]
{\color{outcolor}Out[{\color{outcolor}96}]:} min     0.35
         mean    3.57
         max     6.64
         Name: Fucus, dtype: float64
\end{Verbatim}
            
    \subsubsection{Задача из модуля (шаг -
10)}\label{ux437ux430ux434ux430ux447ux430-ux438ux437-ux43cux43eux434ux443ux43bux44f-ux448ux430ux433---10}

Сгруппируйте
\href{http://stepik.org/media/attachments/course/4852/algae.csv}{данные}
по переменной \textbf{group} и соотнесите вопросы с ответами

    \begin{Verbatim}[commandchars=\\\{\}]
{\color{incolor}In [{\color{incolor}97}]:} \PY{k+kn}{import} \PY{n+nn}{pandas} \PY{k}{as} \PY{n+nn}{pd}
         
         \PY{n}{df} \PY{o}{=} \PY{n}{pd}\PY{o}{.}\PY{n}{read\PYZus{}csv}\PY{p}{(}\PY{l+s+s1}{\PYZsq{}}\PY{l+s+s1}{http://stepik.org/media/attachments/course/4852/algae.csv}\PY{l+s+s1}{\PYZsq{}}\PY{p}{)}
\end{Verbatim}


    \begin{Verbatim}[commandchars=\\\{\}]
{\color{incolor}In [{\color{incolor}98}]:} \PY{n}{df}\PY{o}{.}\PY{n}{head}\PY{p}{(}\PY{p}{)}
\end{Verbatim}


\begin{Verbatim}[commandchars=\\\{\}]
{\color{outcolor}Out[{\color{outcolor}98}]:}                species       genus  group   sucrose    alanin   citrate  \textbackslash{}
         0    Fucus\_vesiculosus       Fucus  brown  3.001472  3.711498  5.004262   
         1  Saccharina\_japonica  Saccharina  brown  6.731070  1.255251  5.621499   
         2       Fucus\_serratus       Fucus  brown  3.276870  0.346431  1.216767   
         3      Fucus\_distichus       Fucus  brown  6.786996  6.641303  6.423606   
         4    Cladophora\_fracta  Cladophora  green  3.861470  1.648450  6.940588   
         
             glucose  oleic\_acid  
         0  2.548459    6.405165  
         1  6.013219    4.156700  
         2  3.623225    0.304573  
         3  2.272724    3.393203  
         4  2.316955    2.528886  
\end{Verbatim}
            
    \begin{Verbatim}[commandchars=\\\{\}]
{\color{incolor}In [{\color{incolor}103}]:} \PY{n}{df}\PY{o}{.}\PY{n}{groupby}\PY{p}{(}\PY{l+s+s1}{\PYZsq{}}\PY{l+s+s1}{group}\PY{l+s+s1}{\PYZsq{}}\PY{p}{)}\PY{o}{.}\PY{n}{agg}\PY{p}{(}\PY{p}{\PYZob{}}\PY{l+s+s1}{\PYZsq{}}\PY{l+s+s1}{sucrose}\PY{l+s+s1}{\PYZsq{}}\PY{p}{:} \PY{p}{[}\PY{l+s+s1}{\PYZsq{}}\PY{l+s+s1}{max}\PY{l+s+s1}{\PYZsq{}}\PY{p}{,} \PY{l+s+s1}{\PYZsq{}}\PY{l+s+s1}{min}\PY{l+s+s1}{\PYZsq{}}\PY{p}{]}\PY{p}{\PYZcb{}}\PY{p}{)}
\end{Verbatim}


\begin{Verbatim}[commandchars=\\\{\}]
{\color{outcolor}Out[{\color{outcolor}103}]:}         sucrose          
                      max       min
          group                    
          brown  6.825467  1.636122
          green  5.712284  2.452623
          red    0.704580  0.704580
\end{Verbatim}
            
    \begin{Verbatim}[commandchars=\\\{\}]
{\color{incolor}In [{\color{incolor}104}]:} \PY{n}{df}\PY{o}{.}\PY{n}{groupby}\PY{p}{(}\PY{l+s+s1}{\PYZsq{}}\PY{l+s+s1}{group}\PY{l+s+s1}{\PYZsq{}}\PY{p}{)}\PY{o}{.}\PY{n}{count}\PY{p}{(}\PY{p}{)}
\end{Verbatim}


\begin{Verbatim}[commandchars=\\\{\}]
{\color{outcolor}Out[{\color{outcolor}104}]:}        species  genus  sucrose  alanin  citrate  glucose  oleic\_acid
          group                                                               
          brown        6      6        6       6        6        6           6
          green        3      3        3       3        3        3           3
          red          1      1        1       1        1        1           1
\end{Verbatim}
            
    \subsubsection{Модуль 1.7 Визуализация
Seaborn}\label{ux43cux43eux434ux443ux43bux44c-1.7-ux432ux438ux437ux443ux430ux43bux438ux437ux430ux446ux438ux44f-seaborn}

    \begin{Verbatim}[commandchars=\\\{\}]
{\color{incolor}In [{\color{incolor}1}]:} \PY{k+kn}{import} \PY{n+nn}{pandas} \PY{k}{as} \PY{n+nn}{pd}
        \PY{k+kn}{import} \PY{n+nn}{numpy} \PY{k}{as} \PY{n+nn}{np}
\end{Verbatim}


    \begin{Verbatim}[commandchars=\\\{\}]
{\color{incolor}In [{\color{incolor}5}]:} \PY{o}{\PYZpc{}}\PY{k}{matplotlib} inline 
        \PY{c+c1}{\PYZsh{}вывод графика в jupyter notebook, а не в отдельном окне}
        \PY{k+kn}{import} \PY{n+nn}{matplotlib}\PY{n+nn}{.}\PY{n+nn}{pyplot} \PY{k}{as} \PY{n+nn}{plt}
        \PY{k+kn}{import} \PY{n+nn}{seaborn} \PY{k}{as} \PY{n+nn}{sns}
\end{Verbatim}


    \begin{Verbatim}[commandchars=\\\{\}]
{\color{incolor}In [{\color{incolor}3}]:} \PY{n}{students\PYZus{}perfomance} \PY{o}{=} \PY{n}{pd}\PY{o}{.}\PY{n}{read\PYZus{}csv}\PY{p}{(}\PY{l+s+s2}{\PYZdq{}}\PY{l+s+s2}{https://stepik.org/media/attachments/course/4852/StudentsPerformance.csv}\PY{l+s+s2}{\PYZdq{}}\PY{p}{)}
\end{Verbatim}


    \begin{Verbatim}[commandchars=\\\{\}]
{\color{incolor}In [{\color{incolor}6}]:} \PY{n}{students\PYZus{}perfomance}\PY{o}{.}\PY{n}{head}\PY{p}{(}\PY{p}{)}
\end{Verbatim}


\begin{Verbatim}[commandchars=\\\{\}]
{\color{outcolor}Out[{\color{outcolor}6}]:}    gender race/ethnicity parental level of education         lunch  \textbackslash{}
        0  female        group B           bachelor's degree      standard   
        1  female        group C                some college      standard   
        2  female        group B             master's degree      standard   
        3    male        group A          associate's degree  free/reduced   
        4    male        group C                some college      standard   
        
          test preparation course  math score  reading score  writing score  
        0                    none          72             72             74  
        1               completed          69             90             88  
        2                    none          90             95             93  
        3                    none          47             57             44  
        4                    none          76             78             75  
\end{Verbatim}
            
    \begin{Verbatim}[commandchars=\\\{\}]
{\color{incolor}In [{\color{incolor}8}]:} \PY{n}{students\PYZus{}perfomance}\PY{p}{[}\PY{l+s+s1}{\PYZsq{}}\PY{l+s+s1}{math score}\PY{l+s+s1}{\PYZsq{}}\PY{p}{]}\PY{o}{.}\PY{n}{hist}\PY{p}{(}\PY{p}{)} \PY{c+c1}{\PYZsh{}стандартный метод из Pandas}
        \PY{c+c1}{\PYZsh{}распределение оценок по математики}
\end{Verbatim}


\begin{Verbatim}[commandchars=\\\{\}]
{\color{outcolor}Out[{\color{outcolor}8}]:} <matplotlib.axes.\_subplots.AxesSubplot at 0x7f42c8b248d0>
\end{Verbatim}
            
    \begin{center}
    \adjustimage{max size={0.9\linewidth}{0.9\paperheight}}{output_112_1.png}
    \end{center}
    { \hspace*{\fill} \\}
    
    \begin{Verbatim}[commandchars=\\\{\}]
{\color{incolor}In [{\color{incolor}11}]:} \PY{c+c1}{\PYZsh{}график корреляции между двумя переменными}
         \PY{n}{students\PYZus{}perfomance}\PY{o}{.}\PY{n}{plot}\PY{o}{.}\PY{n}{scatter}\PY{p}{(}\PY{n}{x}\PY{o}{=}\PY{l+s+s1}{\PYZsq{}}\PY{l+s+s1}{math score}\PY{l+s+s1}{\PYZsq{}}\PY{p}{,} \PY{n}{y}\PY{o}{=}\PY{l+s+s1}{\PYZsq{}}\PY{l+s+s1}{reading score}\PY{l+s+s1}{\PYZsq{}}\PY{p}{)} \PY{c+c1}{\PYZsh{}скаттер плот, по оси Х math score}
\end{Verbatim}


\begin{Verbatim}[commandchars=\\\{\}]
{\color{outcolor}Out[{\color{outcolor}11}]:} <matplotlib.axes.\_subplots.AxesSubplot at 0x7f42c87662e8>
\end{Verbatim}
            
    \begin{center}
    \adjustimage{max size={0.9\linewidth}{0.9\paperheight}}{output_113_1.png}
    \end{center}
    { \hspace*{\fill} \\}
    
    \begin{Verbatim}[commandchars=\\\{\}]
{\color{incolor}In [{\color{incolor}14}]:} \PY{c+c1}{\PYZsh{}Seaborn \PYZhy{} надстройка над matplotlib}
         \PY{c+c1}{\PYZsh{}облако точек, регрессионная прямая, которая показывает лининейную}
         \PY{c+c1}{\PYZsh{}аппроксимацию взаимосвязи двух переменных}
         \PY{n}{sns}\PY{o}{.}\PY{n}{lmplot}\PY{p}{(}\PY{n}{x}\PY{o}{=}\PY{l+s+s1}{\PYZsq{}}\PY{l+s+s1}{math score}\PY{l+s+s1}{\PYZsq{}}\PY{p}{,} \PY{n}{y}\PY{o}{=}\PY{l+s+s1}{\PYZsq{}}\PY{l+s+s1}{reading score}\PY{l+s+s1}{\PYZsq{}}\PY{p}{,} \PY{n}{data}\PY{o}{=}\PY{n}{students\PYZus{}perfomance}\PY{p}{)}
\end{Verbatim}


\begin{Verbatim}[commandchars=\\\{\}]
{\color{outcolor}Out[{\color{outcolor}14}]:} <seaborn.axisgrid.FacetGrid at 0x7f42c83bc048>
\end{Verbatim}
            
    \begin{center}
    \adjustimage{max size={0.9\linewidth}{0.9\paperheight}}{output_114_1.png}
    \end{center}
    { \hspace*{\fill} \\}
    
    \begin{Verbatim}[commandchars=\\\{\}]
{\color{incolor}In [{\color{incolor}15}]:} \PY{c+c1}{\PYZsh{}добавим группирующую переменную, и тогда}
         \PY{n}{sns}\PY{o}{.}\PY{n}{lmplot}\PY{p}{(}\PY{n}{x}\PY{o}{=}\PY{l+s+s1}{\PYZsq{}}\PY{l+s+s1}{math score}\PY{l+s+s1}{\PYZsq{}}\PY{p}{,} \PY{n}{y}\PY{o}{=}\PY{l+s+s1}{\PYZsq{}}\PY{l+s+s1}{reading score}\PY{l+s+s1}{\PYZsq{}}\PY{p}{,} \PY{n}{hue}\PY{o}{=}\PY{l+s+s1}{\PYZsq{}}\PY{l+s+s1}{gender}\PY{l+s+s1}{\PYZsq{}}\PY{p}{,} \PY{n}{data}\PY{o}{=}\PY{n}{students\PYZus{}perfomance}\PY{p}{)}
\end{Verbatim}


\begin{Verbatim}[commandchars=\\\{\}]
{\color{outcolor}Out[{\color{outcolor}15}]:} <seaborn.axisgrid.FacetGrid at 0x7f42c831f898>
\end{Verbatim}
            
    \begin{center}
    \adjustimage{max size={0.9\linewidth}{0.9\paperheight}}{output_115_1.png}
    \end{center}
    { \hspace*{\fill} \\}
    
    \begin{Verbatim}[commandchars=\\\{\}]
{\color{incolor}In [{\color{incolor}17}]:} \PY{n}{sns}\PY{o}{.}\PY{n}{lmplot}\PY{p}{(}\PY{n}{x}\PY{o}{=}\PY{l+s+s1}{\PYZsq{}}\PY{l+s+s1}{math score}\PY{l+s+s1}{\PYZsq{}}\PY{p}{,} \PY{n}{y}\PY{o}{=}\PY{l+s+s1}{\PYZsq{}}\PY{l+s+s1}{reading score}\PY{l+s+s1}{\PYZsq{}}\PY{p}{,} \PY{n}{hue}\PY{o}{=}\PY{l+s+s1}{\PYZsq{}}\PY{l+s+s1}{gender}\PY{l+s+s1}{\PYZsq{}}\PY{p}{,} \PY{n}{data}\PY{o}{=}\PY{n}{students\PYZus{}perfomance}\PY{p}{,} \PY{n}{fit\PYZus{}reg}\PY{o}{=}\PY{k+kc}{False}\PY{p}{)}
\end{Verbatim}


\begin{Verbatim}[commandchars=\\\{\}]
{\color{outcolor}Out[{\color{outcolor}17}]:} <seaborn.axisgrid.FacetGrid at 0x7f42c7fdd7f0>
\end{Verbatim}
            
    \begin{center}
    \adjustimage{max size={0.9\linewidth}{0.9\paperheight}}{output_116_1.png}
    \end{center}
    { \hspace*{\fill} \\}
    
    \begin{Verbatim}[commandchars=\\\{\}]
{\color{incolor}In [{\color{incolor}18}]:} \PY{n}{ax} \PY{o}{=} \PY{n}{sns}\PY{o}{.}\PY{n}{lmplot}\PY{p}{(}\PY{n}{x}\PY{o}{=}\PY{l+s+s1}{\PYZsq{}}\PY{l+s+s1}{math score}\PY{l+s+s1}{\PYZsq{}}\PY{p}{,} \PY{n}{y}\PY{o}{=}\PY{l+s+s1}{\PYZsq{}}\PY{l+s+s1}{reading score}\PY{l+s+s1}{\PYZsq{}}\PY{p}{,} \PY{n}{hue}\PY{o}{=}\PY{l+s+s1}{\PYZsq{}}\PY{l+s+s1}{gender}\PY{l+s+s1}{\PYZsq{}}\PY{p}{,} \PY{n}{data}\PY{o}{=}\PY{n}{students\PYZus{}perfomance}\PY{p}{,} \PY{n}{fit\PYZus{}reg}\PY{o}{=}\PY{k+kc}{False}\PY{p}{)}
         \PY{n}{ax}\PY{o}{.}\PY{n}{set\PYZus{}xlabels}\PY{p}{(}\PY{l+s+s1}{\PYZsq{}}\PY{l+s+s1}{Math score}\PY{l+s+s1}{\PYZsq{}}\PY{p}{)}
         \PY{n}{ax}\PY{o}{.}\PY{n}{set\PYZus{}ylabels}\PY{p}{(}\PY{l+s+s1}{\PYZsq{}}\PY{l+s+s1}{Reading score}\PY{l+s+s1}{\PYZsq{}}\PY{p}{)}
\end{Verbatim}


\begin{Verbatim}[commandchars=\\\{\}]
{\color{outcolor}Out[{\color{outcolor}18}]:} <seaborn.axisgrid.FacetGrid at 0x7f42c7fa1a20>
\end{Verbatim}
            
    \begin{center}
    \adjustimage{max size={0.9\linewidth}{0.9\paperheight}}{output_117_1.png}
    \end{center}
    { \hspace*{\fill} \\}
    
    \paragraph{Задача из модуля (шаг -
5)}\label{ux437ux430ux434ux430ux447ux430-ux438ux437-ux43cux43eux434ux443ux43bux44f-ux448ux430ux433---5}

Представьте, что у вас есть
\href{https://stepik.org/media/attachments/course/4852/income.csv}{датафрэйм}
\textbf{df}, хранящий данные о зарплате за месяц, со всего 1-ой колонкой
\textbf{income}.

Укажите верные способы, как отрисовать простой график зависимости
зарплаты от даты (то, как отображается дата сейчас не важно, главное сам
график)

Убедитесь, что вы используте версию seaborn \textgreater{} = 0.9.

    \begin{Verbatim}[commandchars=\\\{\}]
{\color{incolor}In [{\color{incolor}23}]:} \PY{k+kn}{import} \PY{n+nn}{pandas} \PY{k}{as} \PY{n+nn}{pd}
         \PY{k+kn}{import} \PY{n+nn}{numpy} \PY{k}{as} \PY{n+nn}{np}
         \PY{k+kn}{import} \PY{n+nn}{seaborn} \PY{k}{as} \PY{n+nn}{sns}
         \PY{k+kn}{import} \PY{n+nn}{matplotlib}\PY{n+nn}{.}\PY{n+nn}{pyplot} \PY{k}{as} \PY{n+nn}{plt}
         
         \PY{n}{df} \PY{o}{=} \PY{n}{pd}\PY{o}{.}\PY{n}{read\PYZus{}csv}\PY{p}{(}\PY{l+s+s1}{\PYZsq{}}\PY{l+s+s1}{https://stepik.org/media/attachments/course/4852/income.csv}\PY{l+s+s1}{\PYZsq{}}\PY{p}{)}
\end{Verbatim}


    \begin{Verbatim}[commandchars=\\\{\}]
{\color{incolor}In [{\color{incolor}22}]:} \PY{n}{df}\PY{o}{.}\PY{n}{head}\PY{p}{(}\PY{p}{)}
\end{Verbatim}


\begin{Verbatim}[commandchars=\\\{\}]
{\color{outcolor}Out[{\color{outcolor}22}]:}             income
         2018-11-30      20
         2018-12-31      60
         2019-01-31     180
         2019-02-28     380
         2019-03-31     660
\end{Verbatim}
            
    \begin{Verbatim}[commandchars=\\\{\}]
{\color{incolor}In [{\color{incolor}27}]:} \PY{n}{sns}\PY{o}{.}\PY{n}{lineplot}\PY{p}{(}\PY{n}{data}\PY{o}{=}\PY{n}{df}\PY{p}{)}
\end{Verbatim}


\begin{Verbatim}[commandchars=\\\{\}]
{\color{outcolor}Out[{\color{outcolor}27}]:} <matplotlib.axes.\_subplots.AxesSubplot at 0x7f42c7b04e48>
\end{Verbatim}
            
    \begin{center}
    \adjustimage{max size={0.9\linewidth}{0.9\paperheight}}{output_121_1.png}
    \end{center}
    { \hspace*{\fill} \\}
    
    \paragraph{Задача из модуля (шаг -
6)}\label{ux437ux430ux434ux430ux447ux430-ux438ux437-ux43cux43eux434ux443ux43bux44f-ux448ux430ux433---6}

Вам дан датасэт с 2-мя фичами (колонками). Постройте график
распределения точек (наблюдений) в пространстве этих 2-ух переменных
(одна из них будет x, а другая - y) и напишите число кластеров,
формируемых наблюдениями.

В ответе вы должны указать число кластеров в виде числа (например: 3).

У вас есть неограниченное число попыток. \textbf{Время одной попытки}: 5
mins

    \begin{Verbatim}[commandchars=\\\{\}]
{\color{incolor}In [{\color{incolor}31}]:} \PY{k+kn}{import} \PY{n+nn}{pandas} \PY{k}{as} \PY{n+nn}{pd}
         \PY{k+kn}{import} \PY{n+nn}{numpy} \PY{k}{as} \PY{n+nn}{np}
         \PY{k+kn}{import} \PY{n+nn}{matplotlib}\PY{n+nn}{.}\PY{n+nn}{pyplot} \PY{k}{as} \PY{n+nn}{plt}
         \PY{k+kn}{import} \PY{n+nn}{seaborn} \PY{k}{as} \PY{n+nn}{sns}
\end{Verbatim}


    \begin{Verbatim}[commandchars=\\\{\}]
{\color{incolor}In [{\color{incolor}54}]:} \PY{n}{df} \PY{o}{=} \PY{n}{pd}\PY{o}{.}\PY{n}{read\PYZus{}csv}\PY{p}{(}\PY{l+s+s1}{\PYZsq{}}\PY{l+s+s1}{datasets\PYZus{}ml\PYZus{}stepik/dataset\PYZus{}1\PYZhy{}7\PYZhy{}6.txt}\PY{l+s+s1}{\PYZsq{}}\PY{p}{,} \PY{n}{sep}\PY{o}{=}\PY{l+s+sa}{r}\PY{l+s+s2}{\PYZdq{}}\PY{l+s+s2}{\PYZbs{}}\PY{l+s+s2}{s+}\PY{l+s+s2}{\PYZdq{}}\PY{p}{)}
         \PY{n}{df}\PY{o}{.}\PY{n}{head}\PY{p}{(}\PY{p}{)}
\end{Verbatim}


\begin{Verbatim}[commandchars=\\\{\}]
{\color{outcolor}Out[{\color{outcolor}54}]:}             x            y
         0  894.927646   205.129027
         1  605.804957  1101.988823
         2  997.963934  1406.187134
         3  898.554240   200.312598
         4  595.475078  1107.213419
\end{Verbatim}
            
    \begin{Verbatim}[commandchars=\\\{\}]
{\color{incolor}In [{\color{incolor}56}]:} \PY{n}{sns}\PY{o}{.}\PY{n}{lmplot}\PY{p}{(}\PY{n}{x}\PY{o}{=}\PY{l+s+s1}{\PYZsq{}}\PY{l+s+s1}{x}\PY{l+s+s1}{\PYZsq{}}\PY{p}{,} \PY{n}{y}\PY{o}{=}\PY{l+s+s1}{\PYZsq{}}\PY{l+s+s1}{y}\PY{l+s+s1}{\PYZsq{}}\PY{p}{,} \PY{n}{data}\PY{o}{=}\PY{n}{df}\PY{p}{)}
\end{Verbatim}


\begin{Verbatim}[commandchars=\\\{\}]
{\color{outcolor}Out[{\color{outcolor}56}]:} <seaborn.axisgrid.FacetGrid at 0x7f42c7754a20>
\end{Verbatim}
            
    \begin{center}
    \adjustimage{max size={0.9\linewidth}{0.9\paperheight}}{output_125_1.png}
    \end{center}
    { \hspace*{\fill} \\}
    
    \paragraph{Задача из модуля (шаг -
7)}\label{ux437ux430ux434ux430ux447ux430-ux438ux437-ux43cux43eux434ux443ux43bux44f-ux448ux430ux433---7}

Скачайте
\href{https://stepik.org/media/attachments/course/4852/genome_matrix.csv}{данные},
представляющие геномные расстояния между видами, и постройте тепловую
карту, чтобы различия было видно наглядно. В ответ впишите, какая
картинка соответствует скачанным данным.

Чтобы график отображался как на картинках, добавьте

\begin{Shaded}
\begin{Highlighting}[]
\NormalTok{g }\OperatorTok{=} \CommentTok{# ваш код для создания теплокарты, укажите параметр cmap=viridis для той же цветовой схемы}
\NormalTok{g.xaxis.set_ticks_position(}\StringTok{'top'}\NormalTok{)}
\NormalTok{g.xaxis.set_tick_params(rotation}\OperatorTok{=}\DecValTok{90}\NormalTok{)}
\end{Highlighting}
\end{Shaded}

Вариант 1:

\begin{figure}[htbp]
\centering
\includegraphics{https://ucarecdn.com/c3a64c11-6416-401b-ad27-e43b82fd99be/}
\caption{"image1"}
\end{figure}

    \begin{Verbatim}[commandchars=\\\{\}]
{\color{incolor}In [{\color{incolor}70}]:} \PY{k+kn}{import} \PY{n+nn}{pandas} \PY{k}{as} \PY{n+nn}{pd}
         \PY{k+kn}{import} \PY{n+nn}{numpy} \PY{k}{as} \PY{n+nn}{np}
         \PY{k+kn}{import} \PY{n+nn}{seaborn} \PY{k}{as} \PY{n+nn}{sns}
         \PY{k+kn}{import} \PY{n+nn}{matplotlib}\PY{n+nn}{.}\PY{n+nn}{pyplot} \PY{k}{as} \PY{n+nn}{plt}
         
         \PY{n}{df} \PY{o}{=} \PY{n}{pd}\PY{o}{.}\PY{n}{read\PYZus{}csv}\PY{p}{(}\PY{l+s+s1}{\PYZsq{}}\PY{l+s+s1}{https://stepik.org/media/attachments/course/4852/genome\PYZus{}matrix.csv}\PY{l+s+s1}{\PYZsq{}}\PY{p}{,} \PY{n}{index\PYZus{}col}\PY{o}{=}\PY{l+m+mi}{0}\PY{p}{)}
         \PY{c+c1}{\PYZsh{}в первом столбце есть специальная ошибка, \PYZdq{}запятая\PYZdq{}}
\end{Verbatim}


    \begin{Verbatim}[commandchars=\\\{\}]
{\color{incolor}In [{\color{incolor}71}]:} \PY{n}{df}\PY{o}{.}\PY{n}{head}\PY{p}{(}\PY{p}{)}
\end{Verbatim}


\begin{Verbatim}[commandchars=\\\{\}]
{\color{outcolor}Out[{\color{outcolor}71}]:}           species0  species1  species2  species3  species4
         species0  0.536029  0.920292  0.679708  0.840606  0.430842
         species1  0.920292  0.862417  0.887593  0.769754  0.203214
         species2  0.679708  0.887593  0.595156  0.003435  0.096052
         species3  0.840606  0.769754  0.003435  0.458870  0.029251
         species4  0.430842  0.203214  0.096052  0.029251  0.642109
\end{Verbatim}
            
    \begin{Verbatim}[commandchars=\\\{\}]
{\color{incolor}In [{\color{incolor}72}]:} \PY{n}{g} \PY{o}{=} \PY{n}{sns}\PY{o}{.}\PY{n}{heatmap}\PY{p}{(}\PY{n}{data}\PY{o}{=}\PY{n}{df}\PY{p}{,} \PY{n}{cmap}\PY{o}{=}\PY{l+s+s1}{\PYZsq{}}\PY{l+s+s1}{viridis}\PY{l+s+s1}{\PYZsq{}}\PY{p}{)}
         \PY{n}{g}\PY{o}{.}\PY{n}{xaxis}\PY{o}{.}\PY{n}{set\PYZus{}ticks\PYZus{}position}\PY{p}{(}\PY{l+s+s1}{\PYZsq{}}\PY{l+s+s1}{top}\PY{l+s+s1}{\PYZsq{}}\PY{p}{)}
         \PY{n}{g}\PY{o}{.}\PY{n}{xaxis}\PY{o}{.}\PY{n}{set\PYZus{}tick\PYZus{}params}\PY{p}{(}\PY{n}{rotation}\PY{o}{=}\PY{l+m+mi}{90}\PY{p}{)}
\end{Verbatim}


    \begin{center}
    \adjustimage{max size={0.9\linewidth}{0.9\paperheight}}{output_129_0.png}
    \end{center}
    { \hspace*{\fill} \\}
    
    \paragraph{Задача из модуля (шаг -
8)}\label{ux437ux430ux434ux430ux447ux430-ux438ux437-ux43cux43eux434ux443ux43bux44f-ux448ux430ux433---8}

Пришло время узнать, \sout{кто самый главный рак} какая роль в dota
самая распространённая. Скачайте
\href{https://stepik.org/media/attachments/course/4852/dota_hero_stats.csv}{датасэт}
с данными о героях из игры dota 2 и посмотрите на распределение их
возможных ролей в игре (колонка roles). Постройте гистограмму,
отражающую скольким героям сколько ролей приписывается (по мнению Valve,
конечно) и напишите какое число ролей у большинства героев.

Это задание можно выполнить многими путями, и рисовать гистограмму
вообще говоря для этого не нужно.

Данные взяты \href{https://api.opendota.com/api/heroes}{отсюда}, на этом
же \href{https://www.opendota.com/}{сайте} можно найти больше
разнообразных данных по dota 2

    \begin{Verbatim}[commandchars=\\\{\}]
{\color{incolor}In [{\color{incolor}11}]:} \PY{k+kn}{import} \PY{n+nn}{pandas} \PY{k}{as} \PY{n+nn}{pd}
         \PY{k+kn}{import} \PY{n+nn}{numpy} \PY{k}{as} \PY{n+nn}{np}
         \PY{k+kn}{import} \PY{n+nn}{seaborn} \PY{k}{as} \PY{n+nn}{sns}
         \PY{k+kn}{import} \PY{n+nn}{matplotlib}\PY{n+nn}{.}\PY{n+nn}{pyplot} \PY{k}{as} \PY{n+nn}{plt}
         
         \PY{n}{df} \PY{o}{=} \PY{n}{pd}\PY{o}{.}\PY{n}{read\PYZus{}csv}\PY{p}{(}\PY{l+s+s1}{\PYZsq{}}\PY{l+s+s1}{https://stepik.org/media/attachments/course/4852/dota\PYZus{}hero\PYZus{}stats.csv}\PY{l+s+s1}{\PYZsq{}}\PY{p}{,} \PY{n}{index\PYZus{}col}\PY{o}{=}\PY{l+m+mi}{0}\PY{p}{)}
\end{Verbatim}


    \begin{Verbatim}[commandchars=\\\{\}]
{\color{incolor}In [{\color{incolor}12}]:} \PY{c+c1}{\PYZsh{}eval \PYZhy{} исполняет выражение, т.е. если список идет как строка, то после выполнения получаем обычный список}
         \PY{n}{df}\PY{p}{[}\PY{l+s+s1}{\PYZsq{}}\PY{l+s+s1}{roles\PYZus{}cnt}\PY{l+s+s1}{\PYZsq{}}\PY{p}{]} \PY{o}{=} \PY{n}{df}\PY{p}{[}\PY{l+s+s1}{\PYZsq{}}\PY{l+s+s1}{roles}\PY{l+s+s1}{\PYZsq{}}\PY{p}{]}\PY{o}{.}\PY{n}{apply}\PY{p}{(}\PY{k}{lambda} \PY{n}{x}\PY{p}{:} \PY{n+nb}{len}\PY{p}{(}\PY{n+nb}{eval}\PY{p}{(}\PY{n}{x}\PY{p}{)}\PY{p}{)}\PY{p}{)}
         \PY{n}{df}\PY{o}{.}\PY{n}{head}\PY{p}{(}\PY{p}{)}
\end{Verbatim}


\begin{Verbatim}[commandchars=\\\{\}]
{\color{outcolor}Out[{\color{outcolor}12}]:}   attack\_type  id  legs  localized\_name                          name  \textbackslash{}
         0       Melee   1     2       Anti-Mage        npc\_dota\_hero\_antimage   
         1       Melee   2     2             Axe             npc\_dota\_hero\_axe   
         2      Ranged   3     4            Bane            npc\_dota\_hero\_bane   
         3       Melee   4     2     Bloodseeker     npc\_dota\_hero\_bloodseeker   
         4      Ranged   5     2  Crystal Maiden  npc\_dota\_hero\_crystal\_maiden   
         
           primary\_attr                                              roles  roles\_cnt  
         0          agi                       ['Carry', 'Escape', 'Nuker']          3  
         1          str    ['Initiator', 'Durable', 'Disabler', 'Jungler']          4  
         2          int        ['Support', 'Disabler', 'Nuker', 'Durable']          4  
         3          agi  ['Carry', 'Disabler', 'Jungler', 'Nuker', 'Ini{\ldots}          5  
         4          int        ['Support', 'Disabler', 'Nuker', 'Jungler']          4  
\end{Verbatim}
            
    \begin{Verbatim}[commandchars=\\\{\}]
{\color{incolor}In [{\color{incolor}13}]:} \PY{n}{df}\PY{p}{[}\PY{l+s+s1}{\PYZsq{}}\PY{l+s+s1}{roles\PYZus{}cnt}\PY{l+s+s1}{\PYZsq{}}\PY{p}{]}\PY{o}{.}\PY{n}{hist}\PY{p}{(}\PY{p}{)}
\end{Verbatim}


\begin{Verbatim}[commandchars=\\\{\}]
{\color{outcolor}Out[{\color{outcolor}13}]:} <matplotlib.axes.\_subplots.AxesSubplot at 0x7fd1526ecc50>
\end{Verbatim}
            
    \begin{center}
    \adjustimage{max size={0.9\linewidth}{0.9\paperheight}}{output_133_1.png}
    \end{center}
    { \hspace*{\fill} \\}
    
    \paragraph{Задача из модуля (шаг -
9)}\label{ux437ux430ux434ux430ux447ux430-ux438ux437-ux43cux43eux434ux443ux43bux44f-ux448ux430ux433---9}

Теперь перейдём к
\href{https://stepik.org/media/attachments/course/4852/iris.csv}{цветочкам}.
Магистрантка Адель решила изучить какие бывают ирисы. Помогите Адель
узнать об ирисах больше - скачайте датасэт со значениями параметров
ирисов, постройте их распределения и отметьте правильные утверждения,
глядя на график.

Распределение должно быть по всем образцам, без разделения на вид. Чтобы
построить на 1-ом графике распределения для каждого из параметров, можно
воспользоваться петлёй

\begin{Shaded}
\begin{Highlighting}[]
\ControlFlowTok{for} \NormalTok{column }\OperatorTok{in} \NormalTok{df:}
    \CommentTok{# Draw distribution with that column}
\end{Highlighting}
\end{Shaded}

Ссылки для изучения:

\href{https://seaborn.pydata.org/generated/seaborn.kdeplot.html}{kdeplot}

\href{https://seaborn.pydata.org/generated/seaborn.distplot.html}{distplot}

    \begin{Verbatim}[commandchars=\\\{\}]
{\color{incolor}In [{\color{incolor}1}]:} \PY{k+kn}{import} \PY{n+nn}{pandas} \PY{k}{as} \PY{n+nn}{pd}
        \PY{k+kn}{import} \PY{n+nn}{numpy} \PY{k}{as} \PY{n+nn}{np}
        \PY{k+kn}{import} \PY{n+nn}{seaborn} \PY{k}{as} \PY{n+nn}{sns}
        \PY{k+kn}{import} \PY{n+nn}{matplotlib}\PY{n+nn}{.}\PY{n+nn}{pyplot} \PY{k}{as} \PY{n+nn}{plt}
        
        \PY{n}{df} \PY{o}{=} \PY{n}{pd}\PY{o}{.}\PY{n}{read\PYZus{}csv}\PY{p}{(}\PY{l+s+s1}{\PYZsq{}}\PY{l+s+s1}{https://stepik.org/media/attachments/course/4852/iris.csv}\PY{l+s+s1}{\PYZsq{}}\PY{p}{,} \PY{n}{index\PYZus{}col}\PY{o}{=}\PY{l+m+mi}{0}\PY{p}{)}
\end{Verbatim}


    \begin{Verbatim}[commandchars=\\\{\}]
{\color{incolor}In [{\color{incolor}2}]:} \PY{n}{df}\PY{o}{.}\PY{n}{head}\PY{p}{(}\PY{p}{)}
\end{Verbatim}


\begin{Verbatim}[commandchars=\\\{\}]
{\color{outcolor}Out[{\color{outcolor}2}]:}    sepal length  sepal width  petal length  petal width  species
        0           5.1          3.5           1.4          0.2        0
        1           4.9          3.0           1.4          0.2        0
        2           4.7          3.2           1.3          0.2        0
        3           4.6          3.1           1.5          0.2        0
        4           5.0          3.6           1.4          0.2        0
\end{Verbatim}
            
    \begin{Verbatim}[commandchars=\\\{\}]
{\color{incolor}In [{\color{incolor}17}]:} \PY{n}{df}\PY{o}{.}\PY{n}{max}\PY{p}{(}\PY{p}{)}
\end{Verbatim}


\begin{Verbatim}[commandchars=\\\{\}]
{\color{outcolor}Out[{\color{outcolor}17}]:} sepal length    7.9
         sepal width     4.4
         petal length    6.9
         petal width     2.5
         species         2.0
         dtype: float64
\end{Verbatim}
            
    \begin{Verbatim}[commandchars=\\\{\}]
{\color{incolor}In [{\color{incolor}19}]:} \PY{k}{for} \PY{n}{c} \PY{o+ow}{in} \PY{n}{df}\PY{p}{:}
             \PY{n}{sns}\PY{o}{.}\PY{n}{distplot}\PY{p}{(}\PY{n}{df}\PY{p}{[}\PY{n}{c}\PY{p}{]}\PY{p}{)}
\end{Verbatim}


    \begin{center}
    \adjustimage{max size={0.9\linewidth}{0.9\paperheight}}{output_138_0.png}
    \end{center}
    { \hspace*{\fill} \\}
    
    \begin{Verbatim}[commandchars=\\\{\}]
{\color{incolor}In [{\color{incolor}18}]:} \PY{n}{df}\PY{o}{.}\PY{n}{min}\PY{p}{(}\PY{p}{)}
\end{Verbatim}


\begin{Verbatim}[commandchars=\\\{\}]
{\color{outcolor}Out[{\color{outcolor}18}]:} sepal length    4.3
         sepal width     2.0
         petal length    1.0
         petal width     0.1
         species         0.0
         dtype: float64
\end{Verbatim}
            
    \paragraph{Задача из модуля (шаг -
10)}\label{ux437ux430ux434ux430ux447ux430-ux438ux437-ux43cux43eux434ux443ux43bux44f-ux448ux430ux433---10}

Рассмотрим длину лепестков (\textbf{petal length}) подробнее и
воспользуемся для этого violin плотом. Нарисуйте распределение длины
лепестков ирисов из
\href{https://stepik.org/media/attachments/course/4852/iris.csv}{предыдущего
датасэта с помощью} violin плота и выберите правильный (такой же)
вариант среди предложенных

\href{https://seaborn.pydata.org/generated/seaborn.violinplot.html}{Мануал
по рисованию violin плотов}

    \begin{Verbatim}[commandchars=\\\{\}]
{\color{incolor}In [{\color{incolor}12}]:} \PY{k+kn}{import} \PY{n+nn}{pandas} \PY{k}{as} \PY{n+nn}{pd}
         \PY{k+kn}{import} \PY{n+nn}{numpy} \PY{k}{as} \PY{n+nn}{np}
         \PY{k+kn}{import} \PY{n+nn}{seaborn} \PY{k}{as} \PY{n+nn}{sns}
         \PY{k+kn}{import} \PY{n+nn}{matplotlib}\PY{n+nn}{.}\PY{n+nn}{pyplot} \PY{k}{as} \PY{n+nn}{plt}
         
         \PY{n}{df} \PY{o}{=} \PY{n}{pd}\PY{o}{.}\PY{n}{read\PYZus{}csv}\PY{p}{(}\PY{l+s+s1}{\PYZsq{}}\PY{l+s+s1}{https://stepik.org/media/attachments/course/4852/iris.csv}\PY{l+s+s1}{\PYZsq{}}\PY{p}{,} \PY{n}{index\PYZus{}col}\PY{o}{=}\PY{l+m+mi}{0}\PY{p}{)}
\end{Verbatim}


    \begin{Verbatim}[commandchars=\\\{\}]
{\color{incolor}In [{\color{incolor}13}]:} \PY{n}{df}\PY{o}{.}\PY{n}{head}\PY{p}{(}\PY{p}{)}
\end{Verbatim}


\begin{Verbatim}[commandchars=\\\{\}]
{\color{outcolor}Out[{\color{outcolor}13}]:}    sepal length  sepal width  petal length  petal width  species
         0           5.1          3.5           1.4          0.2        0
         1           4.9          3.0           1.4          0.2        0
         2           4.7          3.2           1.3          0.2        0
         3           4.6          3.1           1.5          0.2        0
         4           5.0          3.6           1.4          0.2        0
\end{Verbatim}
            
    \begin{Verbatim}[commandchars=\\\{\}]
{\color{incolor}In [{\color{incolor}16}]:} \PY{n}{sns}\PY{o}{.}\PY{n}{violinplot}\PY{p}{(}\PY{n}{y}\PY{o}{=}\PY{l+s+s1}{\PYZsq{}}\PY{l+s+s1}{petal length}\PY{l+s+s1}{\PYZsq{}}\PY{p}{,} \PY{n}{data}\PY{o}{=}\PY{n}{df}\PY{p}{)}
\end{Verbatim}


\begin{Verbatim}[commandchars=\\\{\}]
{\color{outcolor}Out[{\color{outcolor}16}]:} <matplotlib.axes.\_subplots.AxesSubplot at 0x7ff6621e3080>
\end{Verbatim}
            
    \begin{center}
    \adjustimage{max size={0.9\linewidth}{0.9\paperheight}}{output_143_1.png}
    \end{center}
    { \hspace*{\fill} \\}
    
    \paragraph{Задача из модуля (шаг -
11)}\label{ux437ux430ux434ux430ux447ux430-ux438ux437-ux43cux43eux434ux443ux43bux44f-ux448ux430ux433---11}

Продолжаем изучение ирисов! Ещё один важный тип графиков - pairplot,
отражающий зависимость пар переменных друг от друга, а также
распределение каждой из переменных. Постройте его и посмотрите на
scatter плоты для каждой из пар фичей. Какая из пар навскидку имеет
наибольшую корреляцию?

Также обратите внимание, что можно разделить на группы с помощью
параметра \textbf{hue}.

Ссылки для изучения:

\href{https://seaborn.pydata.org/generated/seaborn.pairplot.html}{pairplot}

\textbf{Tips:} \textgreater{}Напоминание: если совсем просто, то
"коррелируют" == "меняются одинаково". Т.е. две величины
\emph{коррелируют} между собой, если \emph{изменение одной величины
влечёт такое же изменение другой}. Значит чем сильнее корреляция между
двумя величинами \(x\) и \(y\), тем меньше точки отклоняются от прямой
\(y=x\).

    \begin{Verbatim}[commandchars=\\\{\}]
{\color{incolor}In [{\color{incolor}17}]:} \PY{k+kn}{import} \PY{n+nn}{pandas} \PY{k}{as} \PY{n+nn}{pd}
         \PY{k+kn}{import} \PY{n+nn}{numpy} \PY{k}{as} \PY{n+nn}{np}
         \PY{k+kn}{import} \PY{n+nn}{seaborn} \PY{k}{as} \PY{n+nn}{sns}
         \PY{k+kn}{import} \PY{n+nn}{matplotlib}\PY{n+nn}{.}\PY{n+nn}{pyplot} \PY{k}{as} \PY{n+nn}{plt}
         
         \PY{n}{df} \PY{o}{=} \PY{n}{pd}\PY{o}{.}\PY{n}{read\PYZus{}csv}\PY{p}{(}\PY{l+s+s1}{\PYZsq{}}\PY{l+s+s1}{https://stepik.org/media/attachments/course/4852/iris.csv}\PY{l+s+s1}{\PYZsq{}}\PY{p}{,} \PY{n}{index\PYZus{}col}\PY{o}{=}\PY{l+m+mi}{0}\PY{p}{)}
\end{Verbatim}


    \begin{Verbatim}[commandchars=\\\{\}]
{\color{incolor}In [{\color{incolor}18}]:} \PY{n}{df}\PY{o}{.}\PY{n}{head}\PY{p}{(}\PY{p}{)}
\end{Verbatim}


\begin{Verbatim}[commandchars=\\\{\}]
{\color{outcolor}Out[{\color{outcolor}18}]:}    sepal length  sepal width  petal length  petal width  species
         0           5.1          3.5           1.4          0.2        0
         1           4.9          3.0           1.4          0.2        0
         2           4.7          3.2           1.3          0.2        0
         3           4.6          3.1           1.5          0.2        0
         4           5.0          3.6           1.4          0.2        0
\end{Verbatim}
            
    \begin{Verbatim}[commandchars=\\\{\}]
{\color{incolor}In [{\color{incolor}25}]:} \PY{n}{sns}\PY{o}{.}\PY{n}{pairplot}\PY{p}{(}\PY{n}{df}\PY{p}{,} \PY{n}{hue}\PY{o}{=}\PY{l+s+s1}{\PYZsq{}}\PY{l+s+s1}{species}\PY{l+s+s1}{\PYZsq{}}\PY{p}{)}
         \PY{c+c1}{\PYZsh{}результат petal length and petal width}
\end{Verbatim}


\begin{Verbatim}[commandchars=\\\{\}]
{\color{outcolor}Out[{\color{outcolor}25}]:} <seaborn.axisgrid.PairGrid at 0x7ff63cfaaf28>
\end{Verbatim}
            
    \begin{center}
    \adjustimage{max size={0.9\linewidth}{0.9\paperheight}}{output_147_1.png}
    \end{center}
    { \hspace*{\fill} \\}
    
    \subsection{Практические задания
Pandas}\label{ux43fux440ux430ux43aux442ux438ux447ux435ux441ux43aux438ux435-ux437ux430ux434ux430ux43dux438ux44f-pandas}

    \begin{Verbatim}[commandchars=\\\{\}]
{\color{incolor}In [{\color{incolor}26}]:} \PY{k+kn}{import} \PY{n+nn}{pandas} \PY{k}{as} \PY{n+nn}{pd}
         
         \PY{n}{df} \PY{o}{=} \PY{n}{pd}\PY{o}{.}\PY{n}{read\PYZus{}csv}\PY{p}{(}\PY{l+s+s1}{\PYZsq{}}\PY{l+s+s1}{https://stepik.org/media/attachments/course/4852/my\PYZus{}stat.csv}\PY{l+s+s1}{\PYZsq{}}\PY{p}{)}
\end{Verbatim}


    \begin{Verbatim}[commandchars=\\\{\}]
{\color{incolor}In [{\color{incolor}27}]:} \PY{n}{df}\PY{o}{.}\PY{n}{head}\PY{p}{(}\PY{p}{)}
\end{Verbatim}


\begin{Verbatim}[commandchars=\\\{\}]
{\color{outcolor}Out[{\color{outcolor}27}]:}    V1    V2 V3   V4
         0   2  13.0  B  1.0
         1  -1   7.0  B  1.0
         2   0  11.0  A  0.0
         3   2  11.0  A  2.0
         4   0  10.0  B -1.0
\end{Verbatim}
            
    \begin{Verbatim}[commandchars=\\\{\}]
{\color{incolor}In [{\color{incolor}28}]:} \PY{n}{subset\PYZus{}1} \PY{o}{=} \PY{n}{df}\PY{o}{.}\PY{n}{iloc}\PY{p}{[}\PY{p}{:}\PY{l+m+mi}{10}\PY{p}{,} \PY{p}{[}\PY{l+m+mi}{0}\PY{p}{,} \PY{l+m+mi}{2}\PY{p}{]}\PY{p}{]}
         \PY{n}{subset\PYZus{}1}
\end{Verbatim}


\begin{Verbatim}[commandchars=\\\{\}]
{\color{outcolor}Out[{\color{outcolor}28}]:}    V1 V3
         0   2  B
         1  -1  B
         2   0  A
         3   2  A
         4   0  B
         5   2  A
         6   0  A
         7   1  A
         8   0  B
         9  -1  A
\end{Verbatim}
            
    \begin{Verbatim}[commandchars=\\\{\}]
{\color{incolor}In [{\color{incolor}37}]:} \PY{n}{not\PYZus{}subset\PYZus{}2} \PY{o}{=} \PY{n}{df}\PY{o}{.}\PY{n}{index}\PY{o}{.}\PY{n}{isin}\PY{p}{(}\PY{p}{[}\PY{l+m+mi}{1}\PY{p}{,} \PY{l+m+mi}{4}\PY{p}{]}\PY{p}{)}
         \PY{n}{subset\PYZus{}2} \PY{o}{=} \PY{n}{df}\PY{o}{.}\PY{n}{iloc}\PY{p}{[}\PY{o}{\PYZti{}}\PY{n}{not\PYZus{}subset\PYZus{}2}\PY{p}{,} \PY{p}{[}\PY{l+m+mi}{1}\PY{p}{,}\PY{l+m+mi}{3}\PY{p}{]}\PY{p}{]}
         \PY{n}{subset\PYZus{}2}\PY{o}{.}\PY{n}{head}\PY{p}{(}\PY{p}{)}
\end{Verbatim}


\begin{Verbatim}[commandchars=\\\{\}]
{\color{outcolor}Out[{\color{outcolor}37}]:}      V2   V4
         0  13.0  1.0
         2  11.0  0.0
         3  11.0  2.0
         5  11.0  0.0
         6   9.0  2.0
\end{Verbatim}
            
    \begin{Verbatim}[commandchars=\\\{\}]
{\color{incolor}In [{\color{incolor}38}]:} \PY{k+kn}{import} \PY{n+nn}{pandas} \PY{k}{as} \PY{n+nn}{pd}
         
         \PY{n}{df} \PY{o}{=} \PY{n}{pd}\PY{o}{.}\PY{n}{read\PYZus{}csv}\PY{p}{(}\PY{l+s+s1}{\PYZsq{}}\PY{l+s+s1}{https://stepik.org/media/attachments/course/4852/my\PYZus{}stat\PYZus{}1.csv}\PY{l+s+s1}{\PYZsq{}}\PY{p}{)}
\end{Verbatim}


    \begin{Verbatim}[commandchars=\\\{\}]
{\color{incolor}In [{\color{incolor}39}]:} \PY{n}{df}\PY{o}{.}\PY{n}{head}\PY{p}{(}\PY{p}{)}
\end{Verbatim}


\begin{Verbatim}[commandchars=\\\{\}]
{\color{outcolor}Out[{\color{outcolor}39}]:}    session\_value  time group  n\_users
         0            0.0    12     B        9
         1            NaN    11     A       -1
         2            1.0     8     A        1
         3            2.0     9     B        3
         4            2.0    10     B        9
\end{Verbatim}
            
    \begin{Verbatim}[commandchars=\\\{\}]
{\color{incolor}In [{\color{incolor}57}]:} \PY{n}{df2} \PY{o}{=} \PY{n}{df}\PY{o}{.}\PY{n}{fillna}\PY{p}{(}\PY{l+m+mi}{0}\PY{p}{)}
\end{Verbatim}


    \begin{Verbatim}[commandchars=\\\{\}]
{\color{incolor}In [{\color{incolor}58}]:} \PY{n}{df2}\PY{o}{.}\PY{n}{head}\PY{p}{(}\PY{p}{)}
\end{Verbatim}


\begin{Verbatim}[commandchars=\\\{\}]
{\color{outcolor}Out[{\color{outcolor}58}]:}    session\_value  time group  n\_users
         0            0.0    12     B        9
         1            0.0    11     A       -1
         2            1.0     8     A        1
         3            2.0     9     B        3
         4            2.0    10     B        9
\end{Verbatim}
            
    \begin{Verbatim}[commandchars=\\\{\}]
{\color{incolor}In [{\color{incolor}59}]:} \PY{n}{mean\PYZus{}positive\PYZus{}n\PYZus{}users} \PY{o}{=} \PY{n}{df2}\PY{o}{.}\PY{n}{query}\PY{p}{(}\PY{l+s+s1}{\PYZsq{}}\PY{l+s+s1}{n\PYZus{}users \PYZgt{}= 0}\PY{l+s+s1}{\PYZsq{}}\PY{p}{)}\PY{p}{[}\PY{l+s+s1}{\PYZsq{}}\PY{l+s+s1}{n\PYZus{}users}\PY{l+s+s1}{\PYZsq{}}\PY{p}{]}\PY{o}{.}\PY{n}{median}\PY{p}{(}\PY{p}{)}
         \PY{n}{mean\PYZus{}positive\PYZus{}n\PYZus{}users}
\end{Verbatim}


\begin{Verbatim}[commandchars=\\\{\}]
{\color{outcolor}Out[{\color{outcolor}59}]:} 5.0
\end{Verbatim}
            
    \begin{Verbatim}[commandchars=\\\{\}]
{\color{incolor}In [{\color{incolor}60}]:} \PY{n}{df2}\PY{o}{.}\PY{n}{loc}\PY{p}{[}\PY{n}{df2}\PY{p}{[}\PY{l+s+s1}{\PYZsq{}}\PY{l+s+s1}{n\PYZus{}users}\PY{l+s+s1}{\PYZsq{}}\PY{p}{]} \PY{o}{\PYZlt{}} \PY{l+m+mi}{0}\PY{p}{,} \PY{l+s+s1}{\PYZsq{}}\PY{l+s+s1}{n\PYZus{}users}\PY{l+s+s1}{\PYZsq{}}\PY{p}{]} \PY{o}{=} \PY{n}{mean\PYZus{}positive\PYZus{}n\PYZus{}users}
         \PY{n}{df2}
\end{Verbatim}


\begin{Verbatim}[commandchars=\\\{\}]
{\color{outcolor}Out[{\color{outcolor}60}]:}    session\_value  time group  n\_users
         0            0.0    12     B      9.0
         1            0.0    11     A      5.0
         2            1.0     8     A      1.0
         3            2.0     9     B      3.0
         4            2.0    10     B      9.0
         5            3.0     8     A      8.0
         6            0.0    11     B      5.0
         7            0.0     9     A      4.0
         8            0.0     8     B      5.0
         9            0.0     9     A      5.0
\end{Verbatim}
            
    \subsubsection{Stepik ML-Contest}\label{stepik-ml-contest}

    Практическим проектом нашего курса будет анализ активности студентов
онлайн курса \href{https://stepik.org/course/129/syllabus}{Введение в
анализ данных в R}, спасибо команде stepik, что предоставили
анонимизированные данные.

В этом модуле, мы разберемся с задачей, начнем исследовать данные, а об
условиях соревнований, призах и тайных стэпах раскажу во втором модуле!

Описание данных:

\href{https://stepik.org/media/attachments/course/4852/event_data_train.zip}{events\_train.csv}
- данные о действиях, которые совершают студенты со стэпами

\begin{enumerate}
\def\labelenumi{\arabic{enumi}.}
\tightlist
\item
  \textbf{step\_id} - id стэпа
\item
  \textbf{user\_id} - анонимизированный id юзера
\item
  \textbf{timestamp} - время наступления события в формате unix date
\item
  \textbf{action} - событие, возможные значения:
\end{enumerate}

\begin{itemize}
\tightlist
\item
  discovered - пользователь перешел на стэп
\item
  viewed - просмотр шага,
\item
  started\_attempt - начало попытки решить шаг, ранее нужно было явно
  нажать на кнопку - начать решение, перед тем как приступить к решению
  практического шага
\item
  passed - удачное решение практического шага
\end{itemize}

\href{https://stepik.org/media/attachments/course/4852/submissions_data_train.zip}{submissions\_train.csv}
- данные о времени и статусах сабмитов к практическим заданиям

\begin{enumerate}
\def\labelenumi{\arabic{enumi}.}
\tightlist
\item
  \textbf{step\_id} - id стэпа
\item
  \textbf{timestamp} - время отправки решения в формате unix date
\item
  \textbf{submission\_status} - статус решения
\item
  \textbf{user\_id} - анонимизированный id юзера
\end{enumerate}

    \begin{Verbatim}[commandchars=\\\{\}]
{\color{incolor}In [{\color{incolor}1}]:} \PY{o}{\PYZpc{}}\PY{k}{matplotlib} inline
        \PY{k+kn}{import} \PY{n+nn}{pandas} \PY{k}{as} \PY{n+nn}{pd}
        \PY{k+kn}{import} \PY{n+nn}{numpy} \PY{k}{as} \PY{n+nn}{np}
        \PY{k+kn}{import} \PY{n+nn}{matplotlib}\PY{n+nn}{.}\PY{n+nn}{pyplot} \PY{k}{as} \PY{n+nn}{plt}
        \PY{k+kn}{import} \PY{n+nn}{seaborn} \PY{k}{as} \PY{n+nn}{sns}
\end{Verbatim}


    \begin{Verbatim}[commandchars=\\\{\}]
{\color{incolor}In [{\color{incolor}2}]:} \PY{c+c1}{\PYZsh{}чтение zip архива}
        \PY{n}{events\PYZus{}data} \PY{o}{=} \PY{n}{pd}\PY{o}{.}\PY{n}{read\PYZus{}csv}\PY{p}{(}\PY{l+s+s1}{\PYZsq{}}\PY{l+s+s1}{datasets\PYZus{}ml\PYZus{}stepik/event\PYZus{}data\PYZus{}train.zip}\PY{l+s+s1}{\PYZsq{}}\PY{p}{,} \PY{n}{compression} \PY{o}{=}\PY{l+s+s1}{\PYZsq{}}\PY{l+s+s1}{zip}\PY{l+s+s1}{\PYZsq{}}\PY{p}{)}
\end{Verbatim}


    \begin{Verbatim}[commandchars=\\\{\}]
{\color{incolor}In [{\color{incolor}3}]:} \PY{n}{events\PYZus{}data}\PY{o}{.}\PY{n}{head}\PY{p}{(}\PY{p}{)}
\end{Verbatim}


\begin{Verbatim}[commandchars=\\\{\}]
{\color{outcolor}Out[{\color{outcolor}3}]:}    step\_id   timestamp      action  user\_id
        0    32815  1434340848      viewed    17632
        1    32815  1434340848      passed    17632
        2    32815  1434340848  discovered    17632
        3    32811  1434340895  discovered    17632
        4    32811  1434340895      viewed    17632
\end{Verbatim}
            
    \begin{Verbatim}[commandchars=\\\{\}]
{\color{incolor}In [{\color{incolor}4}]:} \PY{c+c1}{\PYZsh{}timestamp \PYZhy{} это Unix time. Секунды от 01.01.1970}
\end{Verbatim}


    \begin{Verbatim}[commandchars=\\\{\}]
{\color{incolor}In [{\color{incolor}5}]:} \PY{c+c1}{\PYZsh{}самый важный этап, проверить что данные валидные}
        \PY{c+c1}{\PYZsh{}например посмотрим какие вообще есть значения в action}
        
        \PY{n}{events\PYZus{}data}\PY{p}{[}\PY{l+s+s1}{\PYZsq{}}\PY{l+s+s1}{action}\PY{l+s+s1}{\PYZsq{}}\PY{p}{]}\PY{o}{.}\PY{n}{unique}\PY{p}{(}\PY{p}{)}
\end{Verbatim}


\begin{Verbatim}[commandchars=\\\{\}]
{\color{outcolor}Out[{\color{outcolor}5}]:} array(['viewed', 'passed', 'discovered', 'started\_attempt'], dtype=object)
\end{Verbatim}
            
    \begin{Verbatim}[commandchars=\\\{\}]
{\color{incolor}In [{\color{incolor}6}]:} \PY{c+c1}{\PYZsh{}преобразуем данные так, чтобы видеть всю картину}
        
        \PY{c+c1}{\PYZsh{}переведем timestamp в обычный вид}
        \PY{n}{events\PYZus{}data}\PY{p}{[}\PY{l+s+s1}{\PYZsq{}}\PY{l+s+s1}{date}\PY{l+s+s1}{\PYZsq{}}\PY{p}{]} \PY{o}{=} \PY{n}{pd}\PY{o}{.}\PY{n}{to\PYZus{}datetime}\PY{p}{(}\PY{n}{events\PYZus{}data}\PY{p}{[}\PY{l+s+s1}{\PYZsq{}}\PY{l+s+s1}{timestamp}\PY{l+s+s1}{\PYZsq{}}\PY{p}{]}\PY{p}{,} \PY{n}{unit}\PY{o}{=}\PY{l+s+s1}{\PYZsq{}}\PY{l+s+s1}{s}\PY{l+s+s1}{\PYZsq{}}\PY{p}{)}
\end{Verbatim}


    \begin{Verbatim}[commandchars=\\\{\}]
{\color{incolor}In [{\color{incolor}7}]:} \PY{n}{events\PYZus{}data}\PY{o}{.}\PY{n}{head}\PY{p}{(}\PY{p}{)}
\end{Verbatim}


\begin{Verbatim}[commandchars=\\\{\}]
{\color{outcolor}Out[{\color{outcolor}7}]:}    step\_id   timestamp      action  user\_id                date
        0    32815  1434340848      viewed    17632 2015-06-15 04:00:48
        1    32815  1434340848      passed    17632 2015-06-15 04:00:48
        2    32815  1434340848  discovered    17632 2015-06-15 04:00:48
        3    32811  1434340895  discovered    17632 2015-06-15 04:01:35
        4    32811  1434340895      viewed    17632 2015-06-15 04:01:35
\end{Verbatim}
            
    \begin{Verbatim}[commandchars=\\\{\}]
{\color{incolor}In [{\color{incolor}8}]:} \PY{n}{events\PYZus{}data}\PY{o}{.}\PY{n}{dtypes}
\end{Verbatim}


\begin{Verbatim}[commandchars=\\\{\}]
{\color{outcolor}Out[{\color{outcolor}8}]:} step\_id               int64
        timestamp             int64
        action               object
        user\_id               int64
        date         datetime64[ns]
        dtype: object
\end{Verbatim}
            
    \begin{Verbatim}[commandchars=\\\{\}]
{\color{incolor}In [{\color{incolor}9}]:} \PY{n}{events\PYZus{}data}\PY{p}{[}\PY{l+s+s1}{\PYZsq{}}\PY{l+s+s1}{date}\PY{l+s+s1}{\PYZsq{}}\PY{p}{]}\PY{o}{.}\PY{n}{min}\PY{p}{(}\PY{p}{)}
\end{Verbatim}


\begin{Verbatim}[commandchars=\\\{\}]
{\color{outcolor}Out[{\color{outcolor}9}]:} Timestamp('2015-06-15 04:00:48')
\end{Verbatim}
            
    \begin{Verbatim}[commandchars=\\\{\}]
{\color{incolor}In [{\color{incolor}10}]:} \PY{n}{events\PYZus{}data}\PY{p}{[}\PY{l+s+s1}{\PYZsq{}}\PY{l+s+s1}{date}\PY{l+s+s1}{\PYZsq{}}\PY{p}{]}\PY{o}{.}\PY{n}{max}\PY{p}{(}\PY{p}{)}
\end{Verbatim}


\begin{Verbatim}[commandchars=\\\{\}]
{\color{outcolor}Out[{\color{outcolor}10}]:} Timestamp('2018-05-19 23:33:31')
\end{Verbatim}
            
    \begin{Verbatim}[commandchars=\\\{\}]
{\color{incolor}In [{\color{incolor}11}]:} \PY{n}{events\PYZus{}data}\PY{p}{[}\PY{l+s+s1}{\PYZsq{}}\PY{l+s+s1}{day}\PY{l+s+s1}{\PYZsq{}}\PY{p}{]} \PY{o}{=} \PY{n}{events\PYZus{}data}\PY{p}{[}\PY{l+s+s1}{\PYZsq{}}\PY{l+s+s1}{date}\PY{l+s+s1}{\PYZsq{}}\PY{p}{]}\PY{o}{.}\PY{n}{dt}\PY{o}{.}\PY{n}{date}
\end{Verbatim}


    \begin{Verbatim}[commandchars=\\\{\}]
{\color{incolor}In [{\color{incolor}12}]:} \PY{n}{events\PYZus{}data}\PY{o}{.}\PY{n}{head}\PY{p}{(}\PY{p}{)}
\end{Verbatim}


\begin{Verbatim}[commandchars=\\\{\}]
{\color{outcolor}Out[{\color{outcolor}12}]:}    step\_id   timestamp      action  user\_id                date         day
         0    32815  1434340848      viewed    17632 2015-06-15 04:00:48  2015-06-15
         1    32815  1434340848      passed    17632 2015-06-15 04:00:48  2015-06-15
         2    32815  1434340848  discovered    17632 2015-06-15 04:00:48  2015-06-15
         3    32811  1434340895  discovered    17632 2015-06-15 04:01:35  2015-06-15
         4    32811  1434340895      viewed    17632 2015-06-15 04:01:35  2015-06-15
\end{Verbatim}
            
    \begin{Verbatim}[commandchars=\\\{\}]
{\color{incolor}In [{\color{incolor}13}]:} \PY{n}{events\PYZus{}data}\PY{o}{.}\PY{n}{groupby}\PY{p}{(}\PY{l+s+s1}{\PYZsq{}}\PY{l+s+s1}{day}\PY{l+s+s1}{\PYZsq{}}\PY{p}{)}\PYZbs{}
             \PY{o}{.}\PY{n}{user\PYZus{}id}\PY{o}{.}\PY{n}{nunique}\PY{p}{(}\PY{p}{)}\PY{o}{.}\PY{n}{plot}\PY{p}{(}\PY{p}{)}
\end{Verbatim}


\begin{Verbatim}[commandchars=\\\{\}]
{\color{outcolor}Out[{\color{outcolor}13}]:} <matplotlib.axes.\_subplots.AxesSubplot at 0x7f10ef3221d0>
\end{Verbatim}
            
    \begin{center}
    \adjustimage{max size={0.9\linewidth}{0.9\paperheight}}{output_173_1.png}
    \end{center}
    { \hspace*{\fill} \\}
    
    \begin{Verbatim}[commandchars=\\\{\}]
{\color{incolor}In [{\color{incolor}14}]:} \PY{c+c1}{\PYZsh{}увеличим график, чтобы даты непересикались}
         
         \PY{n}{sns}\PY{o}{.}\PY{n}{set}\PY{p}{(}\PY{n}{rc}\PY{o}{=}\PY{p}{\PYZob{}}\PY{l+s+s1}{\PYZsq{}}\PY{l+s+s1}{figure.figsize}\PY{l+s+s1}{\PYZsq{}}\PY{p}{:} \PY{p}{(}\PY{l+m+mi}{9}\PY{p}{,} \PY{l+m+mi}{6}\PY{p}{)}\PY{p}{\PYZcb{}}\PY{p}{)}
\end{Verbatim}


    \begin{Verbatim}[commandchars=\\\{\}]
{\color{incolor}In [{\color{incolor}15}]:} \PY{n}{events\PYZus{}data}\PY{o}{.}\PY{n}{groupby}\PY{p}{(}\PY{l+s+s1}{\PYZsq{}}\PY{l+s+s1}{day}\PY{l+s+s1}{\PYZsq{}}\PY{p}{)}\PYZbs{}
             \PY{o}{.}\PY{n}{user\PYZus{}id}\PY{o}{.}\PY{n}{nunique}\PY{p}{(}\PY{p}{)}\PY{o}{.}\PY{n}{plot}\PY{p}{(}\PY{p}{)}
\end{Verbatim}


\begin{Verbatim}[commandchars=\\\{\}]
{\color{outcolor}Out[{\color{outcolor}15}]:} <matplotlib.axes.\_subplots.AxesSubplot at 0x7f10eefd10f0>
\end{Verbatim}
            
    \begin{center}
    \adjustimage{max size={0.9\linewidth}{0.9\paperheight}}{output_175_1.png}
    \end{center}
    { \hspace*{\fill} \\}
    
    \begin{Verbatim}[commandchars=\\\{\}]
{\color{incolor}In [{\color{incolor}16}]:} \PY{c+c1}{\PYZsh{}посмотрим как распределены по количеству баллов которые они набрали}
         
         \PY{c+c1}{\PYZsh{}НЕПРАВИЛЬНОЕ решение}
         \PY{n}{events\PYZus{}data}\PY{p}{[}\PY{n}{events\PYZus{}data}\PY{o}{.}\PY{n}{action} \PY{o}{==} \PY{l+s+s1}{\PYZsq{}}\PY{l+s+s1}{passed}\PY{l+s+s1}{\PYZsq{}}\PY{p}{]} \PYZbs{}
             \PY{o}{.}\PY{n}{groupby}\PY{p}{(}\PY{l+s+s1}{\PYZsq{}}\PY{l+s+s1}{user\PYZus{}id}\PY{l+s+s1}{\PYZsq{}}\PY{p}{,} \PY{n}{as\PYZus{}index}\PY{o}{=}\PY{k+kc}{False}\PY{p}{)} \PYZbs{}
             \PY{o}{.}\PY{n}{agg}\PY{p}{(}\PY{p}{\PYZob{}}\PY{l+s+s1}{\PYZsq{}}\PY{l+s+s1}{step\PYZus{}id}\PY{l+s+s1}{\PYZsq{}}\PY{p}{:} \PY{l+s+s1}{\PYZsq{}}\PY{l+s+s1}{count}\PY{l+s+s1}{\PYZsq{}}\PY{p}{\PYZcb{}}\PY{p}{)}\PYZbs{}
             \PY{o}{.}\PY{n}{rename}\PY{p}{(}\PY{n}{columns}\PY{o}{=}\PY{p}{\PYZob{}}\PY{l+s+s1}{\PYZsq{}}\PY{l+s+s1}{step\PYZus{}id}\PY{l+s+s1}{\PYZsq{}}\PY{p}{:} \PY{l+s+s1}{\PYZsq{}}\PY{l+s+s1}{passed\PYZus{}steps}\PY{l+s+s1}{\PYZsq{}}\PY{p}{\PYZcb{}}\PY{p}{)}\PY{o}{.}\PY{n}{passed\PYZus{}steps}\PY{o}{.}\PY{n}{hist}\PY{p}{(}\PY{p}{)}
\end{Verbatim}


\begin{Verbatim}[commandchars=\\\{\}]
{\color{outcolor}Out[{\color{outcolor}16}]:} <matplotlib.axes.\_subplots.AxesSubplot at 0x7f10eef4cd68>
\end{Verbatim}
            
    \begin{center}
    \adjustimage{max size={0.9\linewidth}{0.9\paperheight}}{output_176_1.png}
    \end{center}
    { \hspace*{\fill} \\}
    
    \begin{Verbatim}[commandchars=\\\{\}]
{\color{incolor}In [{\color{incolor}17}]:} \PY{c+c1}{\PYZsh{}Что не так с нашими расчетами сумарного количества пройденных стэпов?}
         
         \PY{c+c1}{\PYZsh{}Ответ \PYZhy{} Пользователи, у которых нет ни одного passed стэпа вообще не попадут в результат}
         
         \PY{c+c1}{\PYZsh{}Проверять данные \PYZhy{} к примеру количество пользователей}
         \PY{c+c1}{\PYZsh{}Правильное решение}
         
         \PY{n}{events\PYZus{}data}\PY{o}{.}\PY{n}{pivot\PYZus{}table}\PY{p}{(}\PY{n}{index}\PY{o}{=}\PY{l+s+s1}{\PYZsq{}}\PY{l+s+s1}{user\PYZus{}id}\PY{l+s+s1}{\PYZsq{}}\PY{p}{,} \PY{n}{columns}\PY{o}{=}\PY{l+s+s1}{\PYZsq{}}\PY{l+s+s1}{action}\PY{l+s+s1}{\PYZsq{}}\PY{p}{,} \PY{n}{values}\PY{o}{=}\PY{l+s+s1}{\PYZsq{}}\PY{l+s+s1}{step\PYZus{}id}\PY{l+s+s1}{\PYZsq{}}\PY{p}{,} \PY{n}{aggfunc}\PY{o}{=}\PY{l+s+s1}{\PYZsq{}}\PY{l+s+s1}{count}\PY{l+s+s1}{\PYZsq{}}\PY{p}{,} \PY{n}{fill\PYZus{}value}\PY{o}{=}\PY{l+m+mi}{0}\PY{p}{)}\PYZbs{}
             \PY{o}{.}\PY{n}{reset\PYZus{}index}\PY{p}{(}\PY{p}{)}\PY{o}{.}\PY{n}{head}\PY{p}{(}\PY{p}{)}
\end{Verbatim}


\begin{Verbatim}[commandchars=\\\{\}]
{\color{outcolor}Out[{\color{outcolor}17}]:} action  user\_id  discovered  passed  started\_attempt  viewed
         0             1           1       0                0       1
         1             2           9       9                2      10
         2             3          91      87               30     192
         3             5          11      11                4      12
         4             7           1       1                0       1
\end{Verbatim}
            
    \begin{Verbatim}[commandchars=\\\{\}]
{\color{incolor}In [{\color{incolor}18}]:} \PY{n}{events\PYZus{}data\PYZus{}table} \PY{o}{=} \PY{n}{events\PYZus{}data}\PY{o}{.}\PY{n}{pivot\PYZus{}table}\PY{p}{(}\PY{n}{index}\PY{o}{=}\PY{l+s+s1}{\PYZsq{}}\PY{l+s+s1}{user\PYZus{}id}\PY{l+s+s1}{\PYZsq{}}\PY{p}{,} \PY{n}{columns}\PY{o}{=}\PY{l+s+s1}{\PYZsq{}}\PY{l+s+s1}{action}\PY{l+s+s1}{\PYZsq{}}\PY{p}{,} \PY{n}{values}\PY{o}{=}\PY{l+s+s1}{\PYZsq{}}\PY{l+s+s1}{step\PYZus{}id}\PY{l+s+s1}{\PYZsq{}}\PY{p}{,} \PY{n}{aggfunc}\PY{o}{=}\PY{l+s+s1}{\PYZsq{}}\PY{l+s+s1}{count}\PY{l+s+s1}{\PYZsq{}}\PY{p}{,} \PY{n}{fill\PYZus{}value}\PY{o}{=}\PY{l+m+mi}{0}\PY{p}{)}\PYZbs{}
             \PY{o}{.}\PY{n}{reset\PYZus{}index}\PY{p}{(}\PY{p}{)}
\end{Verbatim}


    \begin{Verbatim}[commandchars=\\\{\}]
{\color{incolor}In [{\color{incolor}19}]:} \PY{n}{events\PYZus{}data\PYZus{}table}\PY{o}{.}\PY{n}{discovered}\PY{o}{.}\PY{n}{hist}\PY{p}{(}\PY{p}{)}
         \PY{n}{events\PYZus{}data\PYZus{}table}\PY{o}{.}\PY{n}{passed}\PY{o}{.}\PY{n}{hist}\PY{p}{(}\PY{p}{)}
\end{Verbatim}


\begin{Verbatim}[commandchars=\\\{\}]
{\color{outcolor}Out[{\color{outcolor}19}]:} <matplotlib.axes.\_subplots.AxesSubplot at 0x7f10eef290f0>
\end{Verbatim}
            
    \begin{center}
    \adjustimage{max size={0.9\linewidth}{0.9\paperheight}}{output_179_1.png}
    \end{center}
    { \hspace*{\fill} \\}
    
    \begin{Verbatim}[commandchars=\\\{\}]
{\color{incolor}In [{\color{incolor}21}]:} \PY{c+c1}{\PYZsh{}число пройденных степов можем посмотреть по другому}
         
         \PY{n}{submissions\PYZus{}data} \PY{o}{=} \PY{n}{pd}\PY{o}{.}\PY{n}{read\PYZus{}csv}\PY{p}{(}\PY{l+s+s1}{\PYZsq{}}\PY{l+s+s1}{datasets\PYZus{}ml\PYZus{}stepik/submissions\PYZus{}data\PYZus{}train.zip}\PY{l+s+s1}{\PYZsq{}}\PY{p}{)}
\end{Verbatim}


    \begin{Verbatim}[commandchars=\\\{\}]
{\color{incolor}In [{\color{incolor}23}]:} \PY{n}{submissions\PYZus{}data}\PY{o}{.}\PY{n}{head}\PY{p}{(}\PY{p}{)}
\end{Verbatim}


\begin{Verbatim}[commandchars=\\\{\}]
{\color{outcolor}Out[{\color{outcolor}23}]:}    step\_id   timestamp submission\_status  user\_id
         0    31971  1434349275           correct    15853
         1    31972  1434348300           correct    15853
         2    31972  1478852149             wrong    15853
         3    31972  1478852164           correct    15853
         4    31976  1434348123             wrong    15853
\end{Verbatim}
            
    \begin{Verbatim}[commandchars=\\\{\}]
{\color{incolor}In [{\color{incolor}24}]:} \PY{n}{submissions\PYZus{}data}\PY{p}{[}\PY{l+s+s1}{\PYZsq{}}\PY{l+s+s1}{date}\PY{l+s+s1}{\PYZsq{}}\PY{p}{]} \PY{o}{=} \PY{n}{pd}\PY{o}{.}\PY{n}{to\PYZus{}datetime}\PY{p}{(}\PY{n}{submissions\PYZus{}data}\PY{p}{[}\PY{l+s+s1}{\PYZsq{}}\PY{l+s+s1}{timestamp}\PY{l+s+s1}{\PYZsq{}}\PY{p}{]}\PY{p}{,} \PY{n}{unit}\PY{o}{=}\PY{l+s+s1}{\PYZsq{}}\PY{l+s+s1}{s}\PY{l+s+s1}{\PYZsq{}}\PY{p}{)}
\end{Verbatim}


    \begin{Verbatim}[commandchars=\\\{\}]
{\color{incolor}In [{\color{incolor}27}]:} \PY{n}{submissions\PYZus{}data}\PY{p}{[}\PY{l+s+s1}{\PYZsq{}}\PY{l+s+s1}{day}\PY{l+s+s1}{\PYZsq{}}\PY{p}{]} \PY{o}{=} \PY{n}{submissions\PYZus{}data}\PY{p}{[}\PY{l+s+s1}{\PYZsq{}}\PY{l+s+s1}{date}\PY{l+s+s1}{\PYZsq{}}\PY{p}{]}\PY{o}{.}\PY{n}{dt}\PY{o}{.}\PY{n}{date}
\end{Verbatim}


    \begin{Verbatim}[commandchars=\\\{\}]
{\color{incolor}In [{\color{incolor}28}]:} \PY{n}{submissions\PYZus{}data}\PY{o}{.}\PY{n}{head}\PY{p}{(}\PY{p}{)}
\end{Verbatim}


\begin{Verbatim}[commandchars=\\\{\}]
{\color{outcolor}Out[{\color{outcolor}28}]:}    step\_id   timestamp submission\_status  user\_id                date  \textbackslash{}
         0    31971  1434349275           correct    15853 2015-06-15 06:21:15   
         1    31972  1434348300           correct    15853 2015-06-15 06:05:00   
         2    31972  1478852149             wrong    15853 2016-11-11 08:15:49   
         3    31972  1478852164           correct    15853 2016-11-11 08:16:04   
         4    31976  1434348123             wrong    15853 2015-06-15 06:02:03   
         
                   day  
         0  2015-06-15  
         1  2015-06-15  
         2  2016-11-11  
         3  2016-11-11  
         4  2015-06-15  
\end{Verbatim}
            
    \begin{Verbatim}[commandchars=\\\{\}]
{\color{incolor}In [{\color{incolor}73}]:} \PY{c+c1}{\PYZsh{}для каждого user посчитаем сколько у него было correct submit}
         \PY{n}{users\PYZus{}scores} \PY{o}{=} \PY{n}{submissions\PYZus{}data}\PY{o}{.}\PY{n}{pivot\PYZus{}table}\PY{p}{(}\PY{n}{index}\PY{o}{=}\PY{l+s+s1}{\PYZsq{}}\PY{l+s+s1}{user\PYZus{}id}\PY{l+s+s1}{\PYZsq{}}\PY{p}{,}
                                     \PY{n}{columns}\PY{o}{=}\PY{l+s+s1}{\PYZsq{}}\PY{l+s+s1}{submission\PYZus{}status}\PY{l+s+s1}{\PYZsq{}}\PY{p}{,}
                                     \PY{n}{values}\PY{o}{=}\PY{l+s+s1}{\PYZsq{}}\PY{l+s+s1}{step\PYZus{}id}\PY{l+s+s1}{\PYZsq{}}\PY{p}{,}
                                     \PY{n}{aggfunc}\PY{o}{=}\PY{l+s+s1}{\PYZsq{}}\PY{l+s+s1}{count}\PY{l+s+s1}{\PYZsq{}}\PY{p}{,}
                                     \PY{n}{fill\PYZus{}value}\PY{o}{=}\PY{l+m+mi}{0}\PY{p}{)}\PY{o}{.}\PY{n}{reset\PYZus{}index}\PY{p}{(}\PY{p}{)}\PY{o}{.}\PY{n}{head}\PY{p}{(}\PY{p}{)}
\end{Verbatim}


    \begin{Verbatim}[commandchars=\\\{\}]
{\color{incolor}In [{\color{incolor}40}]:} \PY{c+c1}{\PYZsh{}посмотрим распределение перерывов у пользователей}
         
         \PY{n}{gap\PYZus{}data} \PY{o}{=} \PY{n}{events\PYZus{}data}\PY{p}{[}\PY{p}{[}\PY{l+s+s1}{\PYZsq{}}\PY{l+s+s1}{user\PYZus{}id}\PY{l+s+s1}{\PYZsq{}}\PY{p}{,} \PY{l+s+s1}{\PYZsq{}}\PY{l+s+s1}{day}\PY{l+s+s1}{\PYZsq{}}\PY{p}{,} \PY{l+s+s1}{\PYZsq{}}\PY{l+s+s1}{timestamp}\PY{l+s+s1}{\PYZsq{}}\PY{p}{]}\PY{p}{]}\PY{o}{.}\PY{n}{drop\PYZus{}duplicates}\PY{p}{(}\PY{n}{subset}\PY{o}{=}\PY{p}{[}\PY{l+s+s1}{\PYZsq{}}\PY{l+s+s1}{user\PYZus{}id}\PY{l+s+s1}{\PYZsq{}}\PY{p}{,} \PY{l+s+s1}{\PYZsq{}}\PY{l+s+s1}{day}\PY{l+s+s1}{\PYZsq{}}\PY{p}{]}\PY{p}{)} \PYZbs{}
             \PY{o}{.}\PY{n}{groupby}\PY{p}{(}\PY{l+s+s1}{\PYZsq{}}\PY{l+s+s1}{user\PYZus{}id}\PY{l+s+s1}{\PYZsq{}}\PY{p}{)}\PY{p}{[}\PY{l+s+s1}{\PYZsq{}}\PY{l+s+s1}{timestamp}\PY{l+s+s1}{\PYZsq{}}\PY{p}{]}\PY{o}{.}\PY{n}{apply}\PY{p}{(}\PY{n+nb}{list}\PY{p}{)} \PYZbs{}
             \PY{o}{.}\PY{n}{apply}\PY{p}{(}\PY{n}{np}\PY{o}{.}\PY{n}{diff}\PY{p}{)}\PY{o}{.}\PY{n}{values}
\end{Verbatim}


    \begin{Verbatim}[commandchars=\\\{\}]
{\color{incolor}In [{\color{incolor}42}]:} \PY{n}{gap\PYZus{}data} \PY{o}{=} \PY{n}{pd}\PY{o}{.}\PY{n}{Series}\PY{p}{(}\PY{n}{np}\PY{o}{.}\PY{n}{concatenate}\PY{p}{(}\PY{n}{gap\PYZus{}data}\PY{p}{,} \PY{n}{axis}\PY{o}{=}\PY{l+m+mi}{0}\PY{p}{)}\PY{p}{)}
\end{Verbatim}


    \begin{Verbatim}[commandchars=\\\{\}]
{\color{incolor}In [{\color{incolor}45}]:} \PY{n}{gap\PYZus{}data} \PY{o}{=} \PY{n}{gap\PYZus{}data} \PY{o}{/} \PY{p}{(}\PY{l+m+mi}{24} \PY{o}{*} \PY{l+m+mi}{60} \PY{o}{*} \PY{l+m+mi}{60}\PY{p}{)}
\end{Verbatim}


    \begin{Verbatim}[commandchars=\\\{\}]
{\color{incolor}In [{\color{incolor}48}]:} \PY{n}{gap\PYZus{}data}\PY{p}{[}\PY{n}{gap\PYZus{}data} \PY{o}{\PYZlt{}} \PY{l+m+mi}{200}\PY{p}{]}\PY{o}{.}\PY{n}{hist}\PY{p}{(}\PY{p}{)}
         \PY{c+c1}{\PYZsh{}большая часть пользователей вкладывается в промежуток 0\PYZhy{}25 дней}
\end{Verbatim}


\begin{Verbatim}[commandchars=\\\{\}]
{\color{outcolor}Out[{\color{outcolor}48}]:} <matplotlib.axes.\_subplots.AxesSubplot at 0x7f10e884f2e8>
\end{Verbatim}
            
    \begin{center}
    \adjustimage{max size={0.9\linewidth}{0.9\paperheight}}{output_189_1.png}
    \end{center}
    { \hspace*{\fill} \\}
    
    \begin{Verbatim}[commandchars=\\\{\}]
{\color{incolor}In [{\color{incolor}50}]:} \PY{n}{gap\PYZus{}data}\PY{o}{.}\PY{n}{quantile}\PY{p}{(}\PY{l+m+mf}{0.90}\PY{p}{)} \PY{c+c1}{\PYZsh{}только 10\PYZpc{} пользователей возвращаются на курс после перерыва в 2 месяца.}
\end{Verbatim}


\begin{Verbatim}[commandchars=\\\{\}]
{\color{outcolor}Out[{\color{outcolor}50}]:} 18.325995370370403
\end{Verbatim}
            
    \begin{Verbatim}[commandchars=\\\{\}]
{\color{incolor}In [{\color{incolor}69}]:} \PY{n}{now} \PY{o}{=} \PY{l+m+mi}{1526772811}
         \PY{n}{drop\PYZus{}out\PYZus{}threshold} \PY{o}{=} \PY{l+m+mi}{259200} \PY{c+c1}{\PYZsh{}30 дней}
\end{Verbatim}


    \begin{Verbatim}[commandchars=\\\{\}]
{\color{incolor}In [{\color{incolor}67}]:} \PY{n}{users\PYZus{}data} \PY{o}{=} \PY{n}{events\PYZus{}data}\PY{o}{.}\PY{n}{groupby}\PY{p}{(}\PY{l+s+s1}{\PYZsq{}}\PY{l+s+s1}{user\PYZus{}id}\PY{l+s+s1}{\PYZsq{}}\PY{p}{,} \PY{n}{as\PYZus{}index}\PY{o}{=}\PY{k+kc}{False}\PY{p}{)} \PYZbs{}
             \PY{o}{.}\PY{n}{agg}\PY{p}{(}\PY{p}{\PYZob{}}\PY{l+s+s1}{\PYZsq{}}\PY{l+s+s1}{timestamp}\PY{l+s+s1}{\PYZsq{}}\PY{p}{:} \PY{l+s+s1}{\PYZsq{}}\PY{l+s+s1}{max}\PY{l+s+s1}{\PYZsq{}}\PY{p}{\PYZcb{}}\PY{p}{)}\PY{o}{.}\PY{n}{rename}\PY{p}{(}\PY{n}{columns}\PY{o}{=}\PY{p}{\PYZob{}}\PY{l+s+s1}{\PYZsq{}}\PY{l+s+s1}{timestamp}\PY{l+s+s1}{\PYZsq{}}\PY{p}{:} \PY{l+s+s1}{\PYZsq{}}\PY{l+s+s1}{last\PYZus{}timestamp}\PY{l+s+s1}{\PYZsq{}}\PY{p}{\PYZcb{}}\PY{p}{)}
\end{Verbatim}


    \begin{Verbatim}[commandchars=\\\{\}]
{\color{incolor}In [{\color{incolor}70}]:} \PY{n}{users\PYZus{}data}\PY{p}{[}\PY{l+s+s1}{\PYZsq{}}\PY{l+s+s1}{is\PYZus{}gone\PYZus{}user}\PY{l+s+s1}{\PYZsq{}}\PY{p}{]} \PY{o}{=} \PY{p}{(}\PY{n}{now} \PY{o}{\PYZhy{}} \PY{n}{users\PYZus{}data}\PY{o}{.}\PY{n}{last\PYZus{}timestamp}\PY{p}{)} \PY{o}{\PYZgt{}} \PY{n}{drop\PYZus{}out\PYZus{}threshold}
\end{Verbatim}


    \begin{Verbatim}[commandchars=\\\{\}]
{\color{incolor}In [{\color{incolor}71}]:} \PY{n}{users\PYZus{}data}\PY{o}{.}\PY{n}{head}\PY{p}{(}\PY{p}{)}
\end{Verbatim}


\begin{Verbatim}[commandchars=\\\{\}]
{\color{outcolor}Out[{\color{outcolor}71}]:}    user\_id  last\_timestamp  is\_gone\_user
         0        1      1472827464          True
         1        2      1519226966          True
         2        3      1444581588          True
         3        5      1499859939          True
         4        7      1521634660          True
\end{Verbatim}
            
    \begin{Verbatim}[commandchars=\\\{\}]
{\color{incolor}In [{\color{incolor}76}]:} \PY{n}{users\PYZus{}data} \PY{o}{=} \PY{n}{users\PYZus{}data}\PY{o}{.}\PY{n}{merge}\PY{p}{(}\PY{n}{users\PYZus{}scores}\PY{p}{,} \PY{n}{on}\PY{o}{=}\PY{l+s+s1}{\PYZsq{}}\PY{l+s+s1}{user\PYZus{}id}\PY{l+s+s1}{\PYZsq{}}\PY{p}{,} \PY{n}{how}\PY{o}{=}\PY{l+s+s1}{\PYZsq{}}\PY{l+s+s1}{outer}\PY{l+s+s1}{\PYZsq{}}\PY{p}{)}\PY{o}{.}\PY{n}{fillna}\PY{p}{(}\PY{l+m+mi}{0}\PY{p}{)}
\end{Verbatim}


    \begin{Verbatim}[commandchars=\\\{\}]
{\color{incolor}In [{\color{incolor}78}]:} \PY{n}{users\PYZus{}data} \PY{o}{=} \PY{n}{users\PYZus{}data}\PY{o}{.}\PY{n}{merge}\PY{p}{(}\PY{n}{events\PYZus{}data\PYZus{}table}\PY{p}{,} \PY{n}{how}\PY{o}{=}\PY{l+s+s1}{\PYZsq{}}\PY{l+s+s1}{outer}\PY{l+s+s1}{\PYZsq{}}\PY{p}{,} \PY{n}{on}\PY{o}{=}\PY{l+s+s1}{\PYZsq{}}\PY{l+s+s1}{user\PYZus{}id}\PY{l+s+s1}{\PYZsq{}}\PY{p}{)}
\end{Verbatim}


    \begin{Verbatim}[commandchars=\\\{\}]
{\color{incolor}In [{\color{incolor}79}]:} \PY{n}{users\PYZus{}data}\PY{o}{.}\PY{n}{head}\PY{p}{(}\PY{p}{)}
\end{Verbatim}


\begin{Verbatim}[commandchars=\\\{\}]
{\color{outcolor}Out[{\color{outcolor}79}]:}    user\_id  last\_timestamp  is\_gone\_user  correct  wrong  discovered  passed  \textbackslash{}
         0        1      1472827464          True      0.0    0.0           1       0   
         1        2      1519226966          True      2.0    0.0           9       9   
         2        3      1444581588          True     29.0   23.0          91      87   
         3        5      1499859939          True      2.0    2.0          11      11   
         4        7      1521634660          True      0.0    0.0           1       1   
         
            started\_attempt  viewed  
         0                0       1  
         1                2      10  
         2               30     192  
         3                4      12  
         4                0       1  
\end{Verbatim}
            
    \begin{Verbatim}[commandchars=\\\{\}]
{\color{incolor}In [{\color{incolor}81}]:} \PY{n}{users\PYZus{}days} \PY{o}{=} \PY{n}{events\PYZus{}data}\PY{o}{.}\PY{n}{groupby}\PY{p}{(}\PY{l+s+s1}{\PYZsq{}}\PY{l+s+s1}{user\PYZus{}id}\PY{l+s+s1}{\PYZsq{}}\PY{p}{)}\PY{o}{.}\PY{n}{day}\PY{o}{.}\PY{n}{nunique}\PY{p}{(}\PY{p}{)}\PY{o}{.}\PY{n}{to\PYZus{}frame}\PY{p}{(}\PY{p}{)}\PY{o}{.}\PY{n}{reset\PYZus{}index}\PY{p}{(}\PY{p}{)}
\end{Verbatim}


    \begin{Verbatim}[commandchars=\\\{\}]
{\color{incolor}In [{\color{incolor}84}]:} \PY{n}{users\PYZus{}data} \PY{o}{=} \PY{n}{users\PYZus{}data}\PY{o}{.}\PY{n}{merge}\PY{p}{(}\PY{n}{users\PYZus{}days}\PY{p}{,} \PY{n}{how}\PY{o}{=}\PY{l+s+s1}{\PYZsq{}}\PY{l+s+s1}{outer}\PY{l+s+s1}{\PYZsq{}}\PY{p}{,} \PY{n}{on}\PY{o}{=}\PY{l+s+s1}{\PYZsq{}}\PY{l+s+s1}{user\PYZus{}id}\PY{l+s+s1}{\PYZsq{}}\PY{p}{)}
\end{Verbatim}


    \begin{Verbatim}[commandchars=\\\{\}]
{\color{incolor}In [{\color{incolor}88}]:} \PY{n}{users\PYZus{}data}\PY{o}{.}\PY{n}{head}\PY{p}{(}\PY{p}{)}
\end{Verbatim}


\begin{Verbatim}[commandchars=\\\{\}]
{\color{outcolor}Out[{\color{outcolor}88}]:}    user\_id  last\_timestamp  is\_gone\_user  correct  wrong  discovered  passed  \textbackslash{}
         0        1      1472827464          True      0.0    0.0           1       0   
         1        2      1519226966          True      2.0    0.0           9       9   
         2        3      1444581588          True     29.0   23.0          91      87   
         3        5      1499859939          True      2.0    2.0          11      11   
         4        7      1521634660          True      0.0    0.0           1       1   
         
            started\_attempt  viewed  day  
         0                0       1    1  
         1                2      10    2  
         2               30     192    7  
         3                4      12    2  
         4                0       1    1  
\end{Verbatim}
            
    \begin{Verbatim}[commandchars=\\\{\}]
{\color{incolor}In [{\color{incolor}86}]:} \PY{n}{users\PYZus{}data}\PY{o}{.}\PY{n}{user\PYZus{}id}\PY{o}{.}\PY{n}{nunique}\PY{p}{(}\PY{p}{)}
\end{Verbatim}


\begin{Verbatim}[commandchars=\\\{\}]
{\color{outcolor}Out[{\color{outcolor}86}]:} 19234
\end{Verbatim}
            
    \begin{Verbatim}[commandchars=\\\{\}]
{\color{incolor}In [{\color{incolor}87}]:} \PY{n}{events\PYZus{}data}\PY{o}{.}\PY{n}{user\PYZus{}id}\PY{o}{.}\PY{n}{nunique}\PY{p}{(}\PY{p}{)}
\end{Verbatim}


\begin{Verbatim}[commandchars=\\\{\}]
{\color{outcolor}Out[{\color{outcolor}87}]:} 19234
\end{Verbatim}
            
    \begin{Verbatim}[commandchars=\\\{\}]
{\color{incolor}In [{\color{incolor}89}]:} \PY{n}{users\PYZus{}data}\PY{p}{[}\PY{l+s+s1}{\PYZsq{}}\PY{l+s+s1}{passed\PYZus{}corse}\PY{l+s+s1}{\PYZsq{}}\PY{p}{]} \PY{o}{=} \PY{n}{users\PYZus{}data}\PY{o}{.}\PY{n}{passed} \PY{o}{\PYZgt{}} \PY{l+m+mi}{170}
\end{Verbatim}


    \begin{Verbatim}[commandchars=\\\{\}]
{\color{incolor}In [{\color{incolor}90}]:} \PY{n}{users\PYZus{}data}\PY{o}{.}\PY{n}{head}\PY{p}{(}\PY{p}{)}
\end{Verbatim}


\begin{Verbatim}[commandchars=\\\{\}]
{\color{outcolor}Out[{\color{outcolor}90}]:}    user\_id  last\_timestamp  is\_gone\_user  correct  wrong  discovered  passed  \textbackslash{}
         0        1      1472827464          True      0.0    0.0           1       0   
         1        2      1519226966          True      2.0    0.0           9       9   
         2        3      1444581588          True     29.0   23.0          91      87   
         3        5      1499859939          True      2.0    2.0          11      11   
         4        7      1521634660          True      0.0    0.0           1       1   
         
            started\_attempt  viewed  day  passed\_corse  
         0                0       1    1         False  
         1                2      10    2         False  
         2               30     192    7         False  
         3                4      12    2         False  
         4                0       1    1         False  
\end{Verbatim}
            
    \begin{Verbatim}[commandchars=\\\{\}]
{\color{incolor}In [{\color{incolor}91}]:} \PY{n}{users\PYZus{}data}\PY{o}{.}\PY{n}{groupby}\PY{p}{(}\PY{l+s+s1}{\PYZsq{}}\PY{l+s+s1}{passed\PYZus{}corse}\PY{l+s+s1}{\PYZsq{}}\PY{p}{)}\PY{o}{.}\PY{n}{count}\PY{p}{(}\PY{p}{)}
\end{Verbatim}


\begin{Verbatim}[commandchars=\\\{\}]
{\color{outcolor}Out[{\color{outcolor}91}]:}               user\_id  last\_timestamp  is\_gone\_user  correct  wrong  \textbackslash{}
         passed\_corse                                                          
         False           17809           17809         17809    17809  17809   
         True             1425            1425          1425     1425   1425   
         
                       discovered  passed  started\_attempt  viewed    day  
         passed\_corse                                                      
         False              17809   17809            17809   17809  17809  
         True                1425    1425             1425    1425   1425  
\end{Verbatim}
            
    \begin{Verbatim}[commandchars=\\\{\}]
{\color{incolor}In [{\color{incolor}93}]:} \PY{p}{(}\PY{l+m+mi}{100} \PY{o}{*} \PY{l+m+mi}{1425}\PY{p}{)} \PY{o}{/} \PY{p}{(}\PY{l+m+mi}{1425} \PY{o}{+} \PY{l+m+mi}{17809}\PY{p}{)} \PY{c+c1}{\PYZsh{}столько процентов пользователей прошло курс}
\end{Verbatim}


\begin{Verbatim}[commandchars=\\\{\}]
{\color{outcolor}Out[{\color{outcolor}93}]:} 7.408755329104711
\end{Verbatim}
            
    \begin{Verbatim}[commandchars=\\\{\}]
{\color{incolor}In [{\color{incolor}94}]:} \PY{n}{events\PYZus{}data}\PY{o}{.}
\end{Verbatim}


\begin{Verbatim}[commandchars=\\\{\}]
{\color{outcolor}Out[{\color{outcolor}94}]:}    step\_id   timestamp      action  user\_id                date         day
         0    32815  1434340848      viewed    17632 2015-06-15 04:00:48  2015-06-15
         1    32815  1434340848      passed    17632 2015-06-15 04:00:48  2015-06-15
         2    32815  1434340848  discovered    17632 2015-06-15 04:00:48  2015-06-15
         3    32811  1434340895  discovered    17632 2015-06-15 04:01:35  2015-06-15
         4    32811  1434340895      viewed    17632 2015-06-15 04:01:35  2015-06-15
\end{Verbatim}
            
    \begin{Verbatim}[commandchars=\\\{\}]
{\color{incolor}In [{\color{incolor}62}]:} \PY{c+c1}{\PYZsh{}id \PYZhy{} Карпова}
         \PY{n}{events\PYZus{}data}\PY{o}{.}\PY{n}{pivot\PYZus{}table}\PY{p}{(}\PY{n}{index}\PY{o}{=}\PY{l+s+s1}{\PYZsq{}}\PY{l+s+s1}{user\PYZus{}id}\PY{l+s+s1}{\PYZsq{}}\PY{p}{,}
                                     \PY{n}{columns}\PY{o}{=}\PY{l+s+s1}{\PYZsq{}}\PY{l+s+s1}{action}\PY{l+s+s1}{\PYZsq{}}\PY{p}{,}
                                     \PY{n}{values}\PY{o}{=}\PY{l+s+s1}{\PYZsq{}}\PY{l+s+s1}{step\PYZus{}id}\PY{l+s+s1}{\PYZsq{}}\PY{p}{,}
                                     \PY{n}{aggfunc}\PY{o}{=}\PY{l+s+s1}{\PYZsq{}}\PY{l+s+s1}{count}\PY{l+s+s1}{\PYZsq{}}\PY{p}{,}
                                     \PY{n}{fill\PYZus{}value}\PY{o}{=}\PY{l+m+mi}{0}\PY{p}{)}\PY{o}{.}\PY{n}{reset\PYZus{}index}\PY{p}{(}\PY{p}{)}\PY{o}{.}\PY{n}{sort\PYZus{}values}\PY{p}{(}\PY{l+s+s1}{\PYZsq{}}\PY{l+s+s1}{started\PYZus{}attempt}\PY{l+s+s1}{\PYZsq{}}\PY{p}{,} \PY{n}{ascending}\PY{o}{=}\PY{k+kc}{False}\PY{p}{)}\PY{o}{.}\PY{n}{head}\PY{p}{(}\PY{p}{)}
\end{Verbatim}


\begin{Verbatim}[commandchars=\\\{\}]
{\color{outcolor}Out[{\color{outcolor}62}]:} action  user\_id  discovered  passed  started\_attempt  viewed
         781        1046         128     124              721    8122
         2611       3572         194     193              550    6968
         6062       8394         131      94              337    1070
         4825       6662         198     197              285    1054
         1204       1649         194     194              281     807
\end{Verbatim}
            
    Обязательно запомните
\href{https://pandas.pydata.org/pandas-docs/stable/user_guide/merging.html}{типы
merge и join}, это справделиво не только для Pandas, но и для работы с
базами данных.

\includegraphics{https://ucarecdn.com/f03da508-b426-42c6-be37-ccd7627ca8a4/}
\includegraphics{https://ucarecdn.com/4614d509-542e-44ca-ad74-ca27b78ad787/}


    % Add a bibliography block to the postdoc
    
    
    
    \end{document}
